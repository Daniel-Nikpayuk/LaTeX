% Copyright 2021 Daniel Nikpayuk
\documentclass[twoside]{article}
\usepackage[letterpaper,left=2cm,right=2cm,top=2.5cm,bottom=2.5cm]{geometry}
\usepackage{asymptote}
\usepackage{amsfonts}
\usepackage{amssymb}
\usepackage{amsmath}
\usepackage{verbatim}
\usepackage{graphicx}
\usepackage{hyperref}

%%%%%%%%%%%%%%%%%%%%%%%%%%%%%%%%%%%%%%%%%%%%%%%%%%%%%%%%%%%%%%%%

% latex symbols:

\newcommand{\vn}{\ensuremath{\varnothing}}

\newcommand{\RA}{\Rightarrow}
\newcommand{\lra}{\longrightarrow}
\newcommand{\then}{\ensuremath{\quad\Longrightarrow\quad}}
\newcommand{\qmapsto}{\ensuremath{\quad \mapsto \quad}}
\newcommand{\mapsfrom}{\mathrel{\reflectbox{\ensuremath{\mapsto}}}}

\newcommand{\defeq}{\ensuremath{\ :=\ }}
\newcommand{\qeq}{\ensuremath{\quad =\quad}}
\newcommand{\qqeq}{\ensuremath{\qquad =\qquad}}
\newcommand{\qdefeq}{\ensuremath{\quad :=\quad}}
\newcommand{\qqdefeq}{\ensuremath{\qquad :=\qquad}}

\newcommand{\quadbar}{\ensuremath{\quad |\quad}}
\newcommand{\quadcomma}{\ensuremath{\quad ,\quad}}
\newcommand{\equals}{\ensuremath{\quad =\quad}}
\newcommand{\defequals}{\ensuremath{\quad :=\quad}}

\newcommand{\qed}{\nobreak \ifvmode \relax \else
      \ifdim\lastskip<1.5em \hskip-\lastskip
      \hskip1.5em plus0em minus0.5em \fi \nobreak
      \vrule height0.75em width0.5em depth0.25em\fi}

% latex markup:

\newcommand{\strong}[1]{{\bfseries #1}}
\newcommand{\bfmbox}[1]{\mbox{\bfseries #1}}

\newtheorem{theorem}{Theorem}
\newtheorem{lemma}[theorem]{Lemma}
\newtheorem{proposition}[theorem]{Proposition}
\newtheorem{corollary}[theorem]{Corollary}

% latex spacing:

\newcommand{\twoquad}{\ensuremath{\quad\quad}}
\newcommand{\threequad}{\ensuremath{\quad\quad\quad}}
\newcommand{\fourquad}{\ensuremath{\quad\quad\quad\quad}}

\newcommand{\twoqquad}{\ensuremath{\qquad\qquad}}
\newcommand{\threeqquad}{\ensuremath{\qquad\qquad\qquad}}
\newcommand{\fourqquad}{\ensuremath{\qquad\qquad\qquad\qquad}}

% essay symbols:

\newcommand{\bv}[1][v]{\ensuremath{\mathbf #1}}
\newcommand{\powerset}[2][P]{\ensuremath{\mathbb{#1}(#2)}}
\newcommand{\nthps}[2][P]{\ensuremath{\mathbb{#1}^{#2}}}
\newcommand{\nthus}[2][U]{\ensuremath{\mathbb{#1}^{#2}}}
\newcommand{\psunion}[2][P]{\ensuremath{\bigcup\limits_{#2\ge 0}\mathbb{#1}^{#2}}}
\newcommand{\usunion}[2][U]{\ensuremath{\bigcup\limits_{#2\ge 0}\mathbb{#1}^{#2}}}
\newcommand{\stratified}{\ensuremath{\mathbb{U}^\infty}}
\newcommand{\of}[1]{\ensuremath{(\mathcal{#1})}}
\newcommand{\ofbb}[1]{\ensuremath{(\mathbb{#1})}}

\newcommand{\readleft}{\ensuremath{\,_{_\leftarrow}\,}}
\newcommand{\readright}{\ensuremath{\,_{_\rightarrow}\,}}

% essay formatting:

\newcommand{\mss}[1]{\ensuremath{\mbox{\scriptsize #1}}}
\newcommand{\bms}[1]{\ensuremath{_{\mbox{\bfseries\tiny #1}}}}
\newcommand{\bnms}[2]{\ensuremath{\bms{#1,}{_#2}}}

\newcommand{\tab}[1][1.125cm]{\hspace{#1}}

\newcommand{\col}[1][0ex]{& \hspace{#1}}
\newcommand{\scol}{\col[0.15cm]}
\newcommand{\lcol}{\col[0.45cm]}

\newcommand{\msbox}[1]{\ensuremath{_{\mbox{\scriptsize #1}}}}
\newcommand{\cbox}[1]{\mbox{// #1}}

\newenvironment{proof}[1][Proof]{\begin{trivlist}
\item[\hskip \labelsep {\bfseries #1}]}{\end{trivlist}}
\newenvironment{definition}[1][Definition]{\begin{trivlist}
\item[\hskip \labelsep {\bfseries #1:}]}{\end{trivlist}}
\newenvironment{example}[1][Example]{\begin{trivlist}
\item[\hskip \labelsep {\bfseries #1}]}{\end{trivlist}}
\newenvironment{remark}[1][Remark]{\begin{trivlist}
\item[\hskip \labelsep {\bfseries #1}]}{\end{trivlist}}

% essay functions:

\newcommand{\subcirc}[1][m]{\ensuremath{\underset{#1}{\circ}}}
\newcommand{\lcirc}{\subcirc[\ell]}
\newcommand{\rcirc}{\subcirc[r]}

\newcommand{\subfunc}[1]{\ensuremath{\langle #1\rangle}}

\newcommand{\id}{\mbox{id}}

\newcommand{\compose}{\mbox{compose}}
\newcommand{\precompose}[1]{\ensuremath{\mbox{precompose}_{#1}}}
\newcommand{\postcompose}[1]{\ensuremath{\mbox{postcompose}_{#1}}}

\newcommand{\ndopose}{\mbox{endopose}}
\newcommand{\prendopose}[1]{\ensuremath{\mbox{preendopose}_{#1}}}
\newcommand{\postndopose}[1]{\ensuremath{\mbox{postendopose}_{#1}}}

\newcommand{\lift}{\mbox{lift}}
\newcommand{\stem}{\mbox{stem}}
\newcommand{\costem}{\mbox{costem}}
\newcommand{\distem}{\mbox{distem}}

%%%%%%%%%%%%%%%%%%%%%%%%%%%%%%%%%%%%%%%%%%%%%%%%%%%%%%%%%%%%%%%%

\begin{asydef}
// this comment prevents a compilation bug.

real direction(bool value)
{
	return value ? 1 : -1;
}

pair offset(pair current, string align, real x = 0.03)
{
	pair shifter =	(align == "E") ? (x, 0) :
			(align == "N") ? (0, x) :
			(align == "W") ? (-x, 0) :
			(align == "S") ? (0, -x) : (0,0);

	return shift(shifter)*current;
}

pair segment(pair current = (0,0), string align, real projection, real angle = 0)
{
	align = (align == "E") ? "EU" :
		(align == "N") ? "NL" :
		(align == "W") ? "WD" :
		(align == "S") ? "SR" : align;

	real base          = direction(substr(align, 0, 1) == "E" || substr(align, 0, 1) == "N");
	real adjacent      = direction(substr(align, 1, 1) == "R" || substr(align, 1, 1) == "U") * Tan(angle);
	pair initialVertex = (substr(align, 0, 1) == "E" || substr(align, 0, 1) == "W") ? (base, adjacent) : (adjacent, base);
	pair scaledVertex  = scale(projection) * initialVertex;
	pair shiftedVertex = shift(current)    * scaledVertex;

	return shiftedVertex;
}

//

void drawpath(path p)
{
	draw(p);

	for (int k=0; k < size(p); ++k)
	{
		dot(point(p, k));
	}
}

void safeLabel(picture pic = currentpicture, Label L, pair position, real width, real height,
	align align = NoAlign, pen textpen = currentpen, pen borderpen = currentpen,
	pen fillpen = white, filltype filltype = NoFill, string bordertype = "round")
{
	if (bordertype == "round")
	{
		pair w = position + (-width, 0);
		pair e = position + ( width, 0);
		pair n = position + (0, height);
		pair s = position + (0,-height);

		filldraw(pic, w{up}::n{right}::e{down}::s{left}::cycle, fillpen, borderpen);
	}
	else if (bordertype == "box")
	{
		pair sw = position + (-width,-height);
		pair se = position + (-width, height);
		pair nw = position + ( width,-height);
		pair ne = position + ( width, height);

		filldraw(pic, sw--se--ne--nw--cycle, fillpen, borderpen);
	}

	label(pic, L, position, align, textpen, filltype);
}


\end{asydef}

%%%%%%%%%%%%%%%%%%%%%%%%%%%%%%%%%%%%%%%%%%%%%%%%%%%%%%%%%%%%%%%%

\title{C++17\ \ Interoperable\\Register Machines}
\author{Daniel Nikpayuk}
\date{April 7, 2021}
\pagestyle{empty}
\begin{document}
\maketitle
\thispagestyle{empty}

\begin{figure}[h]
\centering
\includegraphics[width=1in]{../../../cc-by-nc.png}\\[0.1in]
\tiny This article is licensed under \\
\href{http://creativecommons.org/licenses/by-nc/4.0/}
{Creative Commons Attribution-NonCommercial 4.0 International.}\\[0.3in]
\end{figure}

\noindent Each \strong{atomic machine} has the following form:
$$ \def\arraystretch{1.15}
\tab[0cm] \begin{array}{l}
\texttt{template<>}									\\
\texttt{struct machine<\emph{name}>}							\\
\{											\\
\tab[1cm]\texttt{template<\emph{stack}\ldots>}						\\
\tab[1cm]\texttt{static constexpr auto result(\emph{heaps}\ldots)\ \{\ldots\}}		\\
\}
\end{array} $$
The \emph{name} allows for dispatching (template resolution), while the \emph{constexpr function}
has a single \emph{stack} made up of a variadic pack symbolically representing \emph{registers},
along with a fixed number of \emph{heaps} which are also made up of variadic packs, but which
are \emph{cached}, and thus more expensive in general.

We then build \strong{compound machines} by chaining them together with a \strong{controller}.
Such machines and controllers are organized into a \strong{hierarchy} of machine orders using
a \emph{monadic} narrative design: The idea is that the atomics of a higher level are constructed
from the compounds of lower levels which---assuming self-similarity propogates throughout---then
allows this pattern to scale:

\newpage

	%\begin{comment}
	%%%%%%%%%%%%%%%%%%%%%%%%%%%%%%%%%%%%%%%%%%%%%%%%%%%%%%%%%%%%%%%%%%%%%
                                          %
                                          %
	\begin{center}
	\begin{asy}
	unitsize(1cm);
	import fontsize;

	pair p       = (0, 0);
	pair pl      = segment(p, "SL", 1.5, 50);
	pair pr      = segment(p, "SR", 1.5, 50);

	pair pld     = segment(pl, "S", 2);
	pair prd     = segment(pr, "S", 2);

	pair prdu    = segment(prd, "EU", 1.50, 35);
	pair prdd    = segment(prd, "ED", 1.50, 35);

	pair prdur   = segment(prdu, "E", 1);
	pair prddr   = segment(prdd, "E", 1);

	pair plda    = segment(pld, "E", 1.25);
	pair pldau   = segment(plda, "N", 0.5);
	pair pldad   = segment(plda, "S", 0.5);

	pair prda    = segment(prd, "W", 1.25);
	pair prdau   = segment(prda, "N", 0.5);
	pair prdad   = segment(prda, "S", 0.5);

	//

	pair q       = segment(p, "E", 10);
	pair ql      = segment(q, "SL", 1.5, 50);
	pair qr      = segment(q, "SR", 1.5, 50);

	pair qld     = segment(ql, "S", 2);
	pair qrd     = segment(qr, "S", 2);

	pair qlda    = segment(qld, "E", 1.25);
	pair qldau   = segment(qlda, "N", 0.5);
	pair qldad   = segment(qlda, "S", 0.5);

	pair qrda    = segment(qrd, "W", 1.25);
	pair qrdau   = segment(qrda, "N", 0.5);
	pair qrdad   = segment(qrda, "S", 0.5);

	//

	pair mu      = segment(p, "E", 5.5);
	pair md      = segment(mu, "S", 5.5);

	//

	draw(pldau--prdau, gray, MidArrow(size = 4));
	draw( plda-- prda, gray, MidArrow(size = 4));
	draw(pldad--prdad, gray, MidArrow(size = 4));

	draw(qldau--qrdau, gray, MidArrow(size = 4));
	draw( qlda-- qrda, gray, MidArrow(size = 4));
	draw(qldad--qrdad, gray, MidArrow(size = 4));

	draw(mu--md, gray + dashed);

	//

	safeLabel("lower Order", p, 1.35, 0.30, bordertype = "box");
	safeLabel("higher Order", q, 1.40, 0.30, bordertype = "box");

	safeLabel("atomics", pld, 1.00, 2.00, textpen = heavygray + fontsize(8), borderpen = gray);
	safeLabel("atomics", qld, 1.00, 2.00, textpen = heavygray + fontsize(8), borderpen = gray);

	safeLabel("compounds", prd, 1.00, 2.00, textpen = darkgray + fontsize(8), borderpen = gray);
	safeLabel("compounds", qrd, 1.00, 2.00, textpen = darkgray + fontsize(8), borderpen = gray);

	safeLabel("\}", prdu, 0.20, 1.00, textpen = gray + fontsize(36), borderpen = white, bordertype = "box");
	safeLabel("\}", prdd, 0.20, 1.00, textpen = gray + fontsize(36), borderpen = white, bordertype = "box");

	safeLabel("halters", prdur, 0.70, 0.30, textpen = gray, borderpen = white, bordertype = "box");
	safeLabel("passers", prddr, 1.25, 0.20, textpen = gray, borderpen = white, bordertype = "box");

	draw(segment(prddr, "E", 0.85)--segment(prddr, "E", 2.75), gray, Arrow);

	//

	dot(segment(pld, "NL", 0.5, 20), gray);
	dot(segment(pld, "NL", 1.5, 10), gray);
	dot(segment(pld, "NR", 1, 30), gray);
	dot(segment(pld, "SL", 0.65, 40), gray);
	dot(segment(pld, "SR", 1.05, 25), gray);

	dot(segment(qld, "NL", 0.5, 20), gray);
	dot(segment(qld, "NL", 1.5, 10), gray);
	dot(segment(qld, "NR", 1, 30), gray);
	dot(segment(qld, "SL", 0.65, 40), gray);
	dot(segment(qld, "SR", 1.05, 25), gray);

	//

	pair c0 = segment(prd, "N", 1);
	pair c1 = segment(c0, "SR", 0.2, 45);
	pair c2 = segment(c1, "SR", 0.2, 45);

	pair d0 = segment(prd, "SL", 0.5, 45);
	pair d1 = segment(d0, "SR", 0.2, 45);

	draw(c0--c2, gray);
	draw(d0--d1, gray);

	dot(c0, gray);
	dot(c1, gray);
	dot(c2, gray);

	dot(d0, gray);
	dot(d1, gray);

	//

	pair e0 = segment(qrd, "N", 1);
	pair e1 = segment(e0, "SR", 0.2, 45);
	pair e2 = segment(e1, "SR", 0.2, 45);

	pair f0 = segment(qrd, "SL", 0.5, 45);
	pair f1 = segment(f0, "SR", 0.2, 45);

	draw(e0--e2, gray);
	draw(f0--f1, gray);

	dot(e0, gray);
	dot(e1, gray);
	dot(e2, gray);

	dot(f0, gray);
	dot(f1, gray);

	//

	\end{asy}
	\end{center}
                                          %
                                          %
	%%%%%%%%%%%%%%%%%%%%%%%%%%%%%%%%%%%%%%%%%%%%%%%%%%%%%%%%%%%%%%%%%%%%%
	%\end{comment}

The idea is that with the chains of machines (at a given level) we can either end the chain with
a \emph{halting instruction} or a \emph{passing instruction}. Halters effectively become standalone functions
(with some interface hiding the chain), returning some standalone value. Passers on the other hand are intended
to continuation pass to other machines at higher orders.

This then implies a few consequences for the design of each individual machine:

\begin{itemize}
\item Each machine is required to carry its own controller, which includes required indices (as well as a nesting
depth counter), along with index iterators. Peformance and modularization design suggests such info should generally
be carried on the stack.
\item Each machine that has a higher order is required to carry the controller, indices, iterators, of the machine
it is eventually returning to. Abstraction-wise it makes the most sense to carry this info in a designated heap.
\end{itemize}

\newpage

\end{document}

