% Copyright 2015 Daniel Nikpayuk
\documentclass[twoside]{article}
\usepackage[letterpaper,left=1cm,right=1cm,top=2cm,bottom=2cm]{geometry}
\usepackage{amsmath}
\usepackage{graphicx}
\usepackage{hyperref}

\newtheorem{lemma}{Lemma}[section]

\pagestyle{empty}
\begin{document}

\begin{figure}[h]
\centering
\includegraphics[width=1in]{cc-by-nc.png}\\[0.1in]
\tiny This article is licensed under \\
\href{http://creativecommons.org/licenses/by-nc/4.0/}
{Creative Commons Attribution-NonCommercial 4.0 International.}\\[0.3in]
\end{figure}

This short article provides a quick look at recognizing possible arithmetic overflow for fixed block binary addition.

Although examples themselves are never proofs, they do make things clearer, so our prototypical addition is as follows:

$$ \begin{array}{rr}
   & 1001\ 1101 \\
 + & 1111\ 0010 \\\hline
   & 1\ 1000\ 1111
\end{array} $$

The need for such an error recognition test is primarily for computing science applications.
So in the above we're adding two 8-bit unsigned integers and we end up with a \emph{carry} of $ 1 $:
Thus an arithmetic overflow. Let's say an 8-bit block is the most our hardware (registers) can handle,
in that case if using the builtin cpu addition algorithm you can't prevent possible overflows directly.

The recognition---and potential correction---of such overflow relies specifically on the
mathematical property of integers:

\begin{lemma} $$ 0\leq a,b < m\quad\Longrightarrow\quad a+b < 2m $$
\end{lemma}

First note if $ a=0 $ or $ b=0 $ then this trivially holds. Otherwise the proof relies on the fact that
if $ d < e $ and $ f > 0 $ then $ f+d < f+e $. It's pretty basic undergrad math, but let's go over it anyway,
starting with the simple derivation:

$$ \begin{array}{lrcl}
 & a & < & m \\
 & 0 & < & b \\
\Longrightarrow & a+b & < & b+m
\end{array} $$

With the same logic, we also have:

$$ \begin{array}{lrcl}
 & b & < & m \\
 & 0 & < & m \\
\Longrightarrow & b+m & < & 2m
\end{array} $$

Putting these together we get:

$$ \begin{array}{lrcccl}
 & a+b & < & b+m & < & 2m \\
\Longrightarrow & a+b && < && 2m
\end{array} $$

The main use of this is the following upperbound:
$$ 0\leq a,b<m\quad\Longrightarrow\quad a+b-m < m $$

The error recognizer then, is as follows:

\begin{lemma} $$ 0\leq a,b < m\leq a+b\quad\Longrightarrow\quad a+b-m < a,b $$
\end{lemma}

First off, if $ m\leq a+b $, then $ 0\leq a+b-m $. Secondly from our initial lemma we know that $ a+b-m < m $:
$$ 0\leq a+b-m < m $$
I've focused on the value $ a+b-m $ for the simple reason that it is the value returned if arithmetic overflow
does occur. This is also equivalent to modular arithmetic---only more efficient than using division.

The proof is a simple contradiction. Focusing on $ a $, assume $ a+b-m\geq a $, but then follows that $ b\geq m $.
A one-step contradiction.  A symmetric argument follows for $ b $.

Our arithmetic overflow recognizer follows because if $ a+b $ does not overflow, then $ a,b\leq a+b < m $,
but if it does overflow, the equivalent of modular arithmetic by $ m $ (here $ =2^8 $) will occur as a side effect
of any cpu addition algorithm, and from our second lemma we have $ a+b < a,b < m $. It's a clear logical dichotomy.

\end{document}

