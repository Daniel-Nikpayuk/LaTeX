% Copyright 2015 Daniel Nikpayuk
\documentclass[twoside]{article}
\usepackage[letterpaper,left=1cm,right=1cm,top=2cm,bottom=2cm]{geometry}
\usepackage{amsmath}
\usepackage{amsfonts}
\usepackage{graphicx}
\usepackage{hyperref}
\usepackage{xcolor}
\usepackage{bm}

\newcommand{\st}{$ ^{\mbox{\scriptsize st}} $ }
\newcommand{\nd}{$ ^{\mbox{\scriptsize nd}} $ }
\newcommand{\rd}{$ ^{\mbox{\scriptsize rd}} $ }

\renewcommand{\th}{$ ^{\mbox{\scriptsize th}} $ }
%\th command exists, it gives the old english symbol used for `theta' IPA (voiceless)

\renewcommand{\leq}{\ensuremath{\quad\le\qquad}}

\newcommand{\seq}[1][u]{\ensuremath{<\!#1\!>}}
\newcommand{\bseq}[1][u]{\ensuremath{<\!\!\bm{#1}\!\!>}}
\newcommand{\underseq}[2][u]{\ensuremath{\underset{#2}{<\!#1\!>}}}
\newcommand{\bunderseq}[2][u]{\ensuremath{\underset{#2}{<\!\!\bm{#1}\!\!>}}}

\newcommand{\radix}[2][u]{\ensuremath{\underset{#2}{(#1)}}}
\newcommand{\radixp}[3][u]{\ensuremath{\underset{#2}{(#1)^{#3}}}}
\newcommand{\bradix}[2][u]{\ensuremath{\underset{#2}{(\bm{#1})}}}
\newcommand{\bradixp}[3][u]{\ensuremath{\underset{#2}{(\bm{#1})^{#3}}}}

\newcommand{\numer}[3][w]{\ensuremath{(\bm{#1})_{#2\le k\le #3}}}
\newcommand{\denom}[3][y]{\ensuremath{(\bm{#1})_{#2\le k <  #3}}}

\newcommand{\ldiv}[1]{\ensuremath{\raisebox{0.045em}{)}\hspace{-0.3275em}\overline{\ #1}}}

\newtheorem{theorem}{Theorem}[section]
\newtheorem{lemma}{Lemma}[section]

\newenvironment{proof}[1][Proof]{\begin{trivlist}
\item[\hskip \labelsep {\bfseries #1}]}{\end{trivlist}}
\newenvironment{definition}[1][Definition]{\begin{trivlist}
\item[\hskip \labelsep {\bfseries Definition (#1):}]}{\end{trivlist}}

\newcommand{\qed}{\nobreak \ifvmode \relax \else
      \ifdim\lastskip<1.5em \hskip-\lastskip
      \hskip1.5em plus0em minus0.5em \fi \nobreak
      \vrule height0.5em width0.5em depth0.25em\fi}

\pagestyle{empty}
\begin{document}

\begin{figure}[h]
\centering
\includegraphics[width=1in]{cc-by-nc.png}\\[0.1in]
\tiny This article is licensed under \\
\href{http://creativecommons.org/licenses/by-nc/4.0/}
{Creative Commons Attribution-NonCommercial 4.0 International.}\\[0.3in]
\end{figure}

We have a few subtleties to consider actually. To begin, only midway did we add the assumption that
$ \denom{i}{\ell}\le\numer{i}{\ell} $, but looking more closely it's possible in this case the leading digit of $ \bm(w) $ is zero.
In our intuitive example this wouldn't happen because our divisor was $ 99 $, as large as possible under the constraint of two digits.
If it had been smaller, say $ 12 $, then the first two digits of our dividend $ 20 $ would have been large enough that we wouldn't
have needed the extra step of adding the next digit $ 200 $. In this case you can think of our partial dividend not as $ 20 $ but
instead as $ 020 $. Same value, but three digits now. If you go over the steps in this derivation you will discover it does not
contradict our result.

This resolves the first subtlety in that \emph{our derived bound works for the seperate conditional case within our algorithm} as well.
The second subtlety is it is now time to make compromises on our bound. To be fair, the above bound is still incomplete as we intended
to derive a bound where both the lower and upper were entirely independent of the numerator and denominator values in the middle of
this ordered relation. It is not yet independent because we have the $ \bseq[y]_{\ell-1} $ value embedded within, and so it is here
we make our compromise: We will add the following assumption
$$ \left\lfloor\frac{b}{2}\right\rfloor\leq\bunderseq[y]{\ell-1} $$
but if we intend to use this to further our bound derivation, we need this to be in the form of a chain. Fortunately, we already
know firstly $ \bseq[y]_{\ell-1} < b $. Secondly, given our ``floor function'' lemma from before, we then have
$$ \frac{b-1}{2}\leq\bunderseq[y]{\ell-1}\leq b-1 $$
this bound we also need in the form of a multiplicative inverse
$$ \frac{1}{b-1}\leq\frac{1}{\bseq[y]_{\ell-1}}\leq\frac{2}{b-1} $$

We have what we need to begin. Starting with the lower bound:
$$ \frac{\bseq[y]_{\ell-1}b^{\ell-1}}{(b^\ell-1)^2}-\frac{b^j-1}{\bseq[y]_{\ell-1}b^{\ell-1}}-\frac{b^\ell-2}{\bseq[y]_{\ell-1}b^{\ell-1}} $$
$$ \frac{\bseq[y]_{\ell-1}b^{\ell-1}}{(b^\ell-1)^2}-\frac{b^j-1}{\bseq[y]_{\ell-1}b^{\ell-1}}+\frac{1}{b^\ell-1}-1 $$
which first simplifies to
$$ \frac{\bseq[y]_{\ell-1}b^{\ell-1}}{(b^\ell-1)^2}-\frac{b^j+b^\ell-3}{\bseq[y]_{\ell-1}b^{\ell-1}} $$
we divide and consensus:
$$ \frac{1}{b-1}\leq\frac{1}{\bseq[y]_{\ell-1}}\leq\frac{2}{b-1} $$
becomes
$$ \frac{b^j+b^\ell-3}{(b-1)b^{\ell-1}}
	\leq\frac{b^j+b^\ell-3}{\bseq[y]_{\ell-1}b^{\ell-1}}
	\leq\frac{2(b^j+b^\ell-3)}{(b-1)b^{\ell-1}} $$
which in turn becomes
$$ -\frac{2(b^j+b^\ell-3)}{(b-1)b^{\ell-1}}
	\leq-\frac{b^j+b^\ell-3}{\bseq[y]_{\ell-1}b^{\ell-1}}
	\leq-\frac{b^j+b^\ell-3}{(b-1)b^{\ell-1}} $$

$$ \frac{b^j-1}{(b-1)b^{\ell-1}}
	\leq\frac{b^j-1}{\bseq[y]_{\ell-1}b^{\ell-1}}
	\leq\frac{2(b^j-1)}{(b-1)b^{\ell-1}} $$
which in turn becomes
$$ -\frac{2(b^j-1)}{(b-1)b^{\ell-1}}
	\leq-\frac{b^j-1}{\bseq[y]_{\ell-1}b^{\ell-1}}
	\leq-\frac{b^j-1}{(b-1)b^{\ell-1}} $$


Moving on:
$$ \frac{b-1}{2}\leq\bunderseq[y]{\ell-1}\leq b-1 $$
becomes
$$ \frac{(b-1)b^{\ell-1}}{2(b^\ell-1)^2}
	\leq\frac{\bseq[y]_{\ell-1}b^{\ell-1}}{(b^\ell-1)^2}
	\leq\frac{(b-1)b^{\ell-1}}{(b^\ell-1)^2} $$
and so adding both of these together:
$$ \frac{(b-1)b^{\ell-1}}{2(b^\ell-1)^2}-\frac{2(b^j+b^\ell-3)}{(b-1)b^{\ell-1}}
	\leq\frac{\bseq[y]_{\ell-1}b^{\ell-1}}{(b^\ell-1)^2}-\frac{b^j+b^\ell-3}{\bseq[y]_{\ell-1}b^{\ell-1}}
	\leq\frac{(b-1)b^{\ell-1}}{(b^\ell-1)^2}-\frac{b^j+b^\ell-3}{(b-1)b^{\ell-1}} $$

$$ \frac{(b-1)b^{\ell-1}}{2(b^\ell-1)^2}-\frac{2(b^j-1)}{(b-1)b^{\ell-1}}+\frac{1}{b^\ell-1}-1
	\leq\frac{\bseq[y]_{\ell-1}b^{\ell-1}}{(b^\ell-1)^2}-\frac{b^j+b^\ell-3}{\bseq[y]_{\ell-1}b^{\ell-1}}
	\leq\frac{(b-1)b^{\ell-1}}{(b^\ell-1)^2}-\frac{b^j+b^\ell-3}{(b-1)b^{\ell-1}} $$

The upper bound is a tiny bit simpler:
$$ \frac{(b^{\ell+1}-1)b^j}{\bseq[y]_{\ell-1}^2b^{2\ell-2}}-\frac{1}{b^\ell-1}+1 $$
first noting
$$ \frac{1}{(b-1)^2}\leq\frac{1}{\bseq[y]_{\ell-1}^2}\leq\frac{4}{(b-1)^2} $$
since
$$ \frac{1}{b-1}\leq\frac{1}{\bseq[y]_{\ell-1}}\leq\frac{2}{b-1} $$
and so
$$ \frac{(b^{\ell+1}-1)b^j}{(b-1)^2b^{2\ell-2}}
	\leq\frac{(b^{\ell+1}-1)b^j}{\bseq[y]_{\ell-1}^2b^{2\ell-2}}
	\leq\frac{4(b^{\ell+1}-1)b^j}{(b-1)^2b^{2\ell-2}} $$
along with
$$ \frac{(b^{\ell+1}-1)b^j}{(b-1)^2b^{2\ell-2}}-\frac{1}{b^\ell-1}+1
	\leq\frac{(b^{\ell+1}-1)b^j}{\bseq[y]_{\ell-1}^2b^{2\ell-2}}-\frac{1}{b^\ell-1}+1
	\leq\frac{4(b^{\ell+1}-1)b^j}{(b-1)^2b^{2\ell-2}}-\frac{1}{b^\ell-1}+1 $$

Finally, we have our refined---if not ugly--bound:
$$ \frac{2}{b^\ell-1}+\frac{(b-1)b^{\ell-1}}{2(b^\ell-1)^2}-\frac{2b^j}{(b-1)b^{\ell-1}}-1
	\leq\left\lfloor\frac{\numer{j}{\ell}}{\denom{j}{\ell}}\right\rfloor
		-\left\lfloor\frac{\numer{i}{\ell}}{\denom{i}{\ell}}\right\rfloor
	\leq\frac{4(b^{\ell+1}-1)b^j}{(b-1)^2b^{2\ell-2}}-\frac{1}{b^\ell-1}+1 $$

for the experienced inequality manipulator, you should also be able to verify this bound can be further simplified as:
$$ -1\ \le\ \left\lfloor\frac{\numer{j}{\ell}}{\denom{j}{\ell}}\right\rfloor
		-\left\lfloor\frac{\numer{i}{\ell}}{\denom{i}{\ell}}\right\rfloor
	\ \le\ 1 $$
keep in mind as we are assuming our divisor has two or more digits we are using the additional assumption that $ j\le\ell+2 $.

In the special case of $ i=0 $ the above simplifies further to our quotient of interest,
$$ -1\ \le\ \left\lfloor\frac{\numer{j}{\ell}}{\denom{j}{\ell}}\right\rfloor-q\ \le\ 1 $$
which of course can be rearranged to put the center of attention on said quotient:
$$ \left\lfloor\frac{\numer{j}{\ell}}{\denom{j}{\ell}}\right\rfloor-1
	\le\ \! q\ \le\ \left\lfloor\frac{\numer{j}{\ell}}{\denom{j}{\ell}}\right\rfloor+1 $$

To reiterate: This bound allows us to narrow down in advance our calculation of the quotient for arbitrary precision digit divisors
(two or more digits) to one of three possible cases. At that point, it's guess and check, but a constant bound is far better than
logarithmic or even linear.

And yet, we can still do just a little bit better.


Admittedly we are only interested in the lower end of this particular bound
$$ \frac{\bseq[y]_{\ell-1}b^{\ell-1}}{(b^\ell-1)^2}-\frac{b^j-1}{\bseq[y]_{\ell-1}b^{\ell-1}}-\frac{b^\ell-2}{\bseq[y]_{\ell-1}b^{\ell-1}}
	\leq\frac{\numer{j}{\ell}+1}{\denom{j}{\ell}}-1-\frac{\numer{i}{\ell}}{\denom{i}{\ell}} $$
as we have yet to derive the upper bound separately. In fact this is how such inequality manipulations work: maintain the chain
until the final moment where you usually only need the upper or lower bound but not both.  You have to maintain the chain because
you will likely be performing additive and multiplicative inversions, and complete information is needed for such manipulations.

We are now interested in the upper half of our original bound:
$$ \frac{\numer{j}{\ell}}{\denom{j}{\ell}}+1-\frac{\numer{i}{\ell}+1}{\denom{i}{\ell}} $$
for which we have the identity:
$$ \frac{\numer{j}{\ell}}{\denom{j}{\ell}}+1-\frac{\numer{i}{\ell}+1}{\denom{i}{\ell}}
	\quad=\qquad\bradix[w]{j\le k\le\ell}\left(\frac{1}{\denom{j}{\ell}}-\frac{1}{\denom{i}{\ell}}\right)
	+\frac{\numer{j}{\ell}-\numer{i}{\ell}-1}{\denom{i}{\ell}}+1 $$

Actually, for the curious, I'd like to make a {\bfseries side note} here. As it is, one can also derive the alternate identity:
$$ \frac{\numer{j}{\ell}}{\denom{j}{\ell}}+1-\frac{\numer{i}{\ell}+1}{\denom{i}{\ell}}
	\quad=\qquad\frac{\denom{i}{\ell}\numer{j}{\ell}+\denom{i}{\ell}\denom{j}{\ell}-\denom{j}{\ell}[\numer{i}{\ell}+1]}
		{\denom{i}{\ell}\denom{j}{\ell}} $$
which privileges multiplication as the \emph{root} operator rather than addition. I've pointed out in the \emph{inequality.pdf}
article that multiplicative forms can provide improved bounds over additive forms of an identity, the problem here is some of the
bounds derived in the process of creating a chain here don't satisfy the assumptions needed for the rules of inequality manipulation.
Otherwise I would have likely used this identity instead.

Getting back to it: We now focus on the left side similar to the previous derivation:
$$ \bradix[w]{j\le k\le\ell}\left(\frac{1}{\denom{j}{\ell}}-\frac{1}{\denom{i}{\ell}}\right) $$
Luckily for us, from our previous derivations we already know that
$$ \frac{1}{(b^\ell-1)^2}
	\leq\frac{1}{\denom{j}{\ell}}-\frac{1}{\denom{i}{\ell}}
	\leq\frac{b^j}{\bseq[y]_{\ell-1}^2b^{2\ell-2}} $$
and
$$ \bunderseq[y]{\ell-1}b^{\ell-1}\le\bradix[w]{j\le k\le\ell}\le b^{\ell+1}-1 $$
so we can skip the work and get straight to the result:
$$ \frac{\bseq[y]_{\ell-1}b^{\ell-1}}{(b^\ell-1)^2}
	\leq\bradix[w]{j\le k\le\ell}\left(\frac{1}{\denom{j}{\ell}}-\frac{1}{\denom{i}{\ell}}\right)
	\leq\frac{(b^{\ell+1}-1)b^j}{\bseq[y]_{\ell-1}^2b^{2\ell-2}}\qquad (l'') $$

As for our right-hand side:
$$ \frac{\numer{j}{\ell}-\numer{i}{\ell}-1}{\denom{i}{\ell}} $$
it is also similar to the previous derivations. If we replace the index of $ (4) $ we immediately have:
$$ \frac{1}{b^\ell-1}\leq\frac{1}{\denom{i}{\ell}}\leq\frac{1}{\bseq[y]_{\ell-1}b^{\ell-1}}\qquad (8) $$
and with $ (3) $ from even earlier
$$ 1\le\bradix[w]{i\le k \le\ell}+1-\bradix[w]{j\le k\le\ell}\le b^j $$
we can multiply this by $ (8) $ to obtain
$$ \frac{1}{b^\ell-1}\leq\frac{\numer{i}{\ell}+1-\numer{j}{\ell}}{\denom{i}{\ell}}\leq\frac{b^j}{\bseq[y]_{\ell-1}b^{\ell-1}} $$
which when itself is additively inverted results in the second bound of our above identity:
$$ -\frac{b^j}{\bseq[y]_{\ell-1}b^{\ell-1}}
	\leq\frac{\numer{j}{\ell}-\numer{i}{\ell}-1}{\denom{i}{\ell}}
	\leq-\frac{1}{b^\ell-1}\qquad (r'') $$
which when adding $ (l'') $ and $ (r'') $ as well as adding $ 1 $ more, we obtain:
$$ \frac{\bseq[y]_{\ell-1}b^{\ell-1}}{(b^\ell-1)^2}-\frac{b^j}{\bseq[y]_{\ell-1}b^{\ell-1}}+1
	\leq\frac{\numer{j}{\ell}}{\denom{j}{\ell}}+1-\frac{\numer{i}{\ell}+1}{\denom{i}{\ell}}
	\leq\frac{(b^{\ell+1}-1)b^j}{\bseq[y]_{\ell-1}^2b^{2\ell-2}}-\frac{1}{b^\ell-1}+1 $$
where of course we are only really interested in the upper half:
$$ \frac{\numer{j}{\ell}}{\denom{j}{\ell}}+1-\frac{\numer{i}{\ell}+1}{\denom{i}{\ell}}
	\leq\frac{(b^{\ell+1}-1)b^j}{\bseq[y]_{\ell-1}^2b^{2\ell-2}}-\frac{1}{b^\ell-1}+1 $$

We now have enough, and by using the transitive law we have possibly weaker but more independent bounds on our original relation:
$$ \frac{\bseq[y]_{\ell-1}b^{\ell-1}}{(b^\ell-1)^2}-\frac{b^j-1}{\bseq[y]_{\ell-1}b^{\ell-1}}-\frac{b^\ell-2}{\bseq[y]_{\ell-1}b^{\ell-1}}
	\leq\left\lfloor\frac{\numer{j}{\ell}}{\denom{j}{\ell}}\right\rfloor
		-\left\lfloor\frac{\numer{i}{\ell}}{\denom{i}{\ell}}\right\rfloor
	\leq\frac{(b^{\ell+1}-1)b^j}{\bseq[y]_{\ell-1}^2b^{2\ell-2}}-\frac{1}{b^\ell-1}+1 $$


$$ \frac{1}{b^\ell-1}
	\leq\frac{1}{\denom{i}{\ell}}
	\leq\frac{1}{\bseq[y]_{\ell-1}b^{\ell-1}} $$
$$ \frac{1}{b^\ell-1}-1
	\leq\frac{1}{\denom{j}{\ell}}-1
	\leq\frac{1}{\bseq[y]_{\ell-1}b^{\ell-1}}-1 $$

Finally adding $ (l') $ and $ (m') $ and $ (r') $ together we obtain the first serious bound for our approximation:

$$ \frac{\bseq[y]_{\ell-1}b^{\ell-1}}{(b^\ell-1)^2}-\frac{b^j-1}{\bseq[y]_{\ell-1}b^{\ell-1}}-\frac{b^\ell-2}{\bseq[y]_{\ell-1}b^{\ell-1}}
	\leq\frac{\numer{j}{\ell}+1}{\denom{j}{\ell}}-1-\frac{\numer{i}{\ell}}{\denom{i}{\ell}}
	\leq\frac{(b^{\ell+1}-1)b^j}{\bseq[y]_{\ell-1}^2b^{2\ell-2}}-\frac{\bseq[y]_{\ell-1}b^{\ell-1}-1}{b^\ell-1} $$

$$ \frac{\bseq[y]_{\ell-1}b^{\ell-1}}{(b^\ell-1)^2}-\frac{b^j-1}{\bseq[y]_{\ell-1}b^{\ell-1}}+\frac{1}{b^\ell-1}-1
	\leq\frac{\numer{j}{\ell}+1}{\denom{j}{\ell}}-1-\frac{\numer{i}{\ell}}{\denom{i}{\ell}}
	\leq\frac{(b^{\ell+1}-1)b^j}{\bseq[y]_{\ell-1}^2b^{2\ell-2}}+\frac{1}{\bseq[y]_{\ell-1}b^{\ell-1}}-1 $$


which is equivalent to
$$ \qquad\frac{1-\denom{j}{\ell}}{\denom{j}{\ell}} $$
$$ \bunderseq[y]{\ell-1}b^{\ell-1}-1
	\le\bradix[y]{j\le k < \ell}-1
	\le b^\ell-2 $$
$$ \frac{\bseq[y]_{\ell-1}b^{\ell-1}-1}{b^\ell-1}
	\leq\frac{\denom{j}{\ell}-1}{\denom{j}{\ell}}
	\leq\frac{b^\ell-2}{\bseq[y]_{\ell-1}b^{\ell-1}} $$
when inverted
$$ -\frac{b^\ell-2}{\bseq[y]_{\ell-1}b^{\ell-1}}
	\leq\frac{1-\denom{j}{\ell}}{\denom{j}{\ell}}
	\leq-\frac{\bseq[y]_{\ell-1}b^{\ell-1}-1}{b^\ell-1}\qquad (r') $$





$$ 0\leq\bradix[w]{j\le k\le\ell}\left(\frac{1}{\denom{i}{\ell}}-\frac{1}{\denom{j}{\ell}}\right) $$
and finally it's pretty obvious
$$ -1 \quad <\qquad \frac{1}{\denom{i}{\ell}}-1 $$
so if we add them altogether we already can deduce
$$ 0\leq\frac{\numer{i}{\ell}-\numer{j}{\ell}}{\denom{i}{\ell}}
	+\bradix[w]{j\le k\le\ell}\left(\frac{1}{\denom{i}{\ell}}-\frac{1}{\denom{j}{\ell}}\right)
	+\frac{1}{\denom{i}{\ell}}-1 $$






Changing our focus now toward the middle of our original identity:
$$ \bradix[w]{i\le k\le\ell}\left(\frac{1}{\denom{j}{\ell}}-\frac{1}{\denom{i}{\ell}}\right) $$
we first simplify fractional part:
$$ \frac{1}{\denom{j}{\ell}}-\frac{1}{\denom{i}{\ell}}=\frac{\denom{i}{j}}{\denom{j}{\ell}\denom{i}{\ell}} $$
Recalling the previous derivation $ (4) $, if we create a copy replacing the index $ j $ with $ i $ and multiply, we obtain:
multiplying $ (6) $ and $ (7) $ together we derive the next bound in our sequence:
$$ \frac{1}{(b^\ell-1)^2}
	\leq\frac{\denom{i}{j}}{\denom{j}{\ell}\denom{i}{\ell}}
	\leq\frac{b^j}{\bseq[y]_{\ell-1}^2b^{2\ell-2}} $$
multiplying this and the former together we derive our right bound
$$ \frac{\bseq[y]_{\ell-1}b^{\ell-1}}{(b^\ell-1)^2}
	\leq\bradix[w]{i\le k\le\ell}\left(\frac{1}{\denom{j}{\ell}}-\frac{1}{\denom{i}{\ell}}\right)
	\leq\frac{(b^{\ell+1}-1)b^j}{\bseq[y]_{\ell-1}^2b^{2\ell-2}}\qquad (m') $$


%%%%%%%%%%%%%%%%%%%%%%%%%%%%%%%%%%%%%5
$$ \begin{array}{rcl}
\bradix[y]{j\le k < \ell}[\bradix[w]{i\le k\le\ell}+1]
	-\bradix[y]{i\le k < \ell}\bradix[y]{j\le k < \ell}
	-\bradix[y]{i\le k < \ell}\bradix[w]{j\le k\le\ell}
 & = & \bradix[y]{j\le k < \ell}\bradix[w]{i\le k\le\ell}+\bradix[y]{j\le k < \ell}
	-\bradix[y]{i\le k < \ell}\bradix[y]{j\le k < \ell}
	-\bradix[y]{i\le k < \ell}\bradix[w]{j\le k\le\ell} \\ \\
 & = & \bradix[y]{j\le k < \ell}\bradix[w]{i\le k\le\ell}
	-\bradix[y]{j\le k < \ell}\bradix[w]{j\le k\le\ell}+\bradix[y]{j\le k < \ell}\bradix[w]{j\le k\le\ell}
	+\bradix[y]{j\le k < \ell}[1-\bradix[y]{i\le k < \ell}]
	-\bradix[y]{i\le k < \ell}\bradix[w]{j\le k\le\ell} \\ \\
 & = & \bradix[y]{j\le k < \ell}\bradix[w]{i\le k\le\ell}
	-\bradix[y]{j\le k < \ell}\bradix[w]{j\le k\le\ell}
	+\bradix[y]{j\le k < \ell}[1-\bradix[y]{i\le k < \ell}]
	+\bradix[y]{j\le k < \ell}\bradix[w]{j\le k\le\ell}-\bradix[y]{i\le k < \ell}\bradix[w]{j\le k\le\ell} \\ \\
 & = & \bradix[y]{j\le k < \ell}[\bradix[w]{i\le k\le\ell}-\bradix[w]{j\le k\le\ell}]
	+\bradix[y]{j\le k < \ell}[1-\bradix[y]{i\le k < \ell}]
	+\bradix[w]{j\le k\le\ell}[\bradix[y]{j\le k < \ell}-\bradix[y]{i\le k < \ell}] \\ \\
\end{array} $$

$$ (ab+c)b^{\ell-1}\le\bradix[y]{i\le k < \ell}\le b^\ell-1 $$
$$ (ab+c)b^{\ell-1}\le\bradix[y]{j\le k < \ell}\le b^\ell-1 $$

$$ 1\le\bradix[w]{i\le k\le\ell}-\bradix[w]{j\le k\le\ell}\le b^j-1 $$
$$ 1\le\bradix[y]{i\le k < \ell}-\bradix[y]{j\le k < \ell}\le b^j-1 $$

$$ (ab+c)b^{\ell-1}
	\leq\bradix[y]{j\le k < \ell}\left(\bradix[w]{i\le k\le\ell}-\bradix[w]{j\le k\le\ell}\right)
	\leq (b^\ell-1)(b^j-1) $$

$$ \frac{1}{(b^\ell-1)^2}
	\leq\frac{1}{\denom{i}{\ell}\denom{j}{\ell}}
	\leq\frac{1}{(ab^\ell+cb^{\ell-1})^2} $$

$$ \frac{(ab+c)b^{\ell-1}}{(b^\ell-1)^2}
	\leq\frac{\bradix[y]{j\le k < \ell}\left(\bradix[w]{i\le k\le\ell}-\bradix[w]{j\le k\le\ell}\right)}{\denom{i}{\ell}\denom{j}{\ell}}
	\leq\frac{(b^\ell-1)(b^j-1)}{(ab^\ell+cb^{\ell-1})^2} $$

$$ (ab+c)b^{\ell-1}-1
	\leq\bradix[y]{i\le k < \ell}-1
	\leq b^\ell-2 $$

$$ \left((ab+c)b^{\ell-1}-1\right)\left((ab+c)b^{\ell-1}\right)
	\leq\bradix[y]{j\le k < \ell}\left(\bradix[y]{i\le k < \ell}-1\right)
	\leq (b^\ell-1)(b^\ell-2) $$

$$ \frac{1}{(b^\ell-1)^2}
	\leq\frac{1}{\denom{i}{\ell}\denom{j}{\ell}}
	\leq\frac{1}{(ab^\ell+cb^{\ell-1})^2} $$

$$ \frac{\left((ab+c)b^{\ell-1}-1\right)\left((ab+c)b^{\ell-1}\right)}{(b^\ell-1)^2}
	\leq\frac{\bradix[y]{j\le k < \ell}\left(\bradix[y]{i\le k < \ell}-1\right)}{\denom{i}{\ell}\denom{j}{\ell}}
	\leq\frac{(b^\ell-1)(b^\ell-2)}{(ab^\ell+cb^{\ell-1})^2} $$

$$ -\frac{(b^\ell-1)(b^\ell-2)}{(ab^\ell+cb^{\ell-1})^2}
	\leq\frac{\bradix[y]{j\le k < \ell}\left(1-\bradix[y]{i\le k < \ell}\right)}{\denom{i}{\ell}\denom{j}{\ell}}
	\leq-\frac{\left((ab+c)b^{\ell-1}-1\right)\left((ab+c)b^{\ell-1}\right)}{(b^\ell-1)^2} $$

$$ 1\le\bradix[y]{i\le k < \ell}-\bradix[y]{j\le k < \ell}\le b^j-1 $$
$$ ab^\ell+cb^{\ell-1}\le\bradix[w]{j\le k\le\ell}\le b^{\ell+1}-1 $$

$$ ab^\ell+cb^{\ell-1}
	\leq\bradix[w]{j\le k\le\ell}\left(\bradix[y]{i\le k < \ell}-\bradix[y]{j\le k < \ell}\right)
	\leq (b^{\ell+1}-1)(b^j-1) $$

$$ \frac{1}{(b^\ell-1)^2}
	\leq\frac{1}{\denom{i}{\ell}\denom{j}{\ell}}
	\leq\frac{1}{(ab^\ell+cb^{\ell-1})^2} $$

$$ \frac{ab^\ell+cb^{\ell-1}}{(b^\ell-1)^2}
	\leq\frac{\bradix[w]{j\le k\le\ell}\left(\bradix[y]{i\le k < \ell}-\bradix[y]{j\le k < \ell}\right)}{\denom{i}{\ell}\denom{j}{\ell}}
	\leq\frac{(b^{\ell+1}-1)(b^j-1)}{(ab^\ell+cb^{\ell-1})^2} $$

$$ -\frac{(b^{\ell+1}-1)(b^j-1)}{(ab^\ell+cb^{\ell-1})^2}
	\leq\frac{\bradix[w]{j\le k\le\ell}\left(\bradix[y]{j\le k < \ell}-\bradix[y]{i\le k < \ell}\right)}{\denom{i}{\ell}\denom{j}{\ell}}
	\leq-\frac{ab^\ell+cb^{\ell-1}}{(b^\ell-1)^2} $$

$$ \frac{(ab+c)b^{\ell-1}}{(b^\ell-1)^2}
	\leq\frac{\bradix[y]{j\le k < \ell}\left(\bradix[w]{i\le k\le\ell}-\bradix[w]{j\le k\le\ell}\right)}{\denom{i}{\ell}\denom{j}{\ell}}
	\leq\frac{(b^\ell-1)(b^j-1)}{(ab^\ell+cb^{\ell-1})^2} $$

$$ -\frac{(b^\ell-1)(b^\ell-2)}{(ab^\ell+cb^{\ell-1})^2}
	\leq\frac{\bradix[y]{j\le k < \ell}\left(1-\bradix[y]{i\le k < \ell}\right)}{\denom{i}{\ell}\denom{j}{\ell}}
	\leq-\frac{\left((ab+c)b^{\ell-1}-1\right)\left((ab+c)b^{\ell-1}\right)}{(b^\ell-1)^2} $$

$$ -\frac{(b^{\ell+1}-1)(b^j-1)}{(ab^\ell+cb^{\ell-1})^2}
	\leq\frac{\bradix[w]{j\le k\le\ell}\left(\bradix[y]{j\le k < \ell}-\bradix[y]{i\le k < \ell}\right)}{\denom{i}{\ell}\denom{j}{\ell}}
	\leq-\frac{ab^\ell+cb^{\ell-1}}{(b^\ell-1)^2} $$

$$ \frac{(ab+c)b^{\ell-1}}{(b^\ell-1)^2}
		-\frac{(b^\ell-1)(b^\ell-2)}{(ab^\ell+cb^{\ell-1})^2}
		-\frac{(b^{\ell+1}-1)(b^j-1)}{(ab^\ell+cb^{\ell-1})^2}
	\leq\frac{\denom{j}{\ell}[\numer{i}{\ell}+1]
		-\denom{i}{\ell}\denom{j}{\ell}
		-\denom{i}{\ell}\numer{j}{\ell}}
		{\denom{i}{\ell}\denom{j}{\ell}} $$

$$ \frac{(ab+c)b^{\ell-1}}{(b^\ell-1)^2}
		-\frac{(b^\ell-1)(b^\ell-2)+(b^{\ell+1}-1)(b^j-1)}{(ab^\ell+cb^{\ell-1})^2}
	\leq\frac{\denom{j}{\ell}[\numer{i}{\ell}+1]
		-\denom{i}{\ell}\denom{j}{\ell}
		-\denom{i}{\ell}\numer{j}{\ell}}
		{\denom{i}{\ell}\denom{j}{\ell}} $$

$$ \frac{(ab+c)b^{\ell-1}}{(b^\ell-1)^2}
		-\frac{2b^{2\ell}-b^{\ell+1}-3b^\ell-b^{\ell-1}+3}{(ab^\ell+cb^{\ell-1})^2}
	\leq\frac{\denom{j}{\ell}[\numer{i}{\ell}+1]
		-\denom{i}{\ell}\denom{j}{\ell}
		-\denom{i}{\ell}\numer{j}{\ell}}
		{\denom{i}{\ell}\denom{j}{\ell}} $$





As for the next term, we know
$$ \frac{1}{b^\ell-1}
	\leq\frac{1}{\denom{h}{\ell}}
	\leq\frac{1}{ab^\ell+cb^{\ell-1}} $$
which when additively inverted leaves us with
$$ -\frac{1}{ab^\ell+cb^{\ell-1}}
	\leq-\frac{1}{\denom{h}{\ell}}
	\leq-\frac{1}{b^\ell-1} $$
and by substituting the proper indices and adding results in
$$ \frac{1}{b^\ell-1}-\frac{1}{ab^\ell+cb^{\ell-1}}
	\leq\frac{1}{\denom{j}{\ell}}-\frac{1}{\denom{i}{\ell}}
	\leq\frac{1}{ab^\ell+cb^{\ell-1}}-\frac{1}{b^\ell-1} $$
which is in the wrong order, but allows us to use \emph{multiplicatively}. This claim of course needs proving:

\newpage

\begin{lemma}
Let $ a,c $ be such that
$$ 1\le ab+c\le b-1 $$
we then have
$$ -1\leq\frac{1}{b^\ell-1}-\frac{1}{ab^\ell+cb^{\ell-1}} $$
\end{lemma}

\begin{proof}
Starting with
$$ 1\le ab+c\le b-1 $$
we derive
$$ b^{\ell-1}\le ab^\ell+cb^{\ell-1}\le b^\ell-b^{\ell-1} $$
we safely assume
$$ 1\le b^{\ell-1}\le ab^\ell+cb^{\ell-1}\le b^\ell-b^{\ell-1} $$
and so
$$ 1-\frac{1}{b^\ell}\leq 1\leq b^{\ell-1}\leq ab^\ell+cb^{\ell-1}\leq b^\ell-b^{\ell-1} $$
simplifying
$$ 1-\frac{1}{b^\ell}\leq ab^\ell+cb^{\ell-1} $$
equivalently
$$ \frac{b^\ell-1}{b^\ell}\leq ab^\ell+cb^{\ell-1} $$
this implies
$$ \frac{1}{ab^\ell+cb^{\ell-1}}\leq \frac{b^\ell}{b^\ell-1} $$
as well
$$ \frac{1}{ab^\ell+cb^{\ell-1}}\leq 1+\frac{1}{b^\ell-1} $$
finally
$$ -1\leq \frac{1}{b^\ell-1}-\frac{1}{ab^\ell+cb^{\ell-1}}\qed $$
\end{proof}

As a consequence of this lemma, in order to multiply, we need to shift right by one which admittedly complicates things a little:
$$ \frac{1}{b^\ell-1}-\frac{1}{ab^\ell+cb^{\ell-1}}+1
	\leq\frac{1}{\denom{j}{\ell}}-\frac{1}{\denom{i}{\ell}}+1
	\leq\frac{1}{ab^\ell+cb^{\ell-1}}-\frac{1}{b^\ell-1}+1 $$
recalling
$$ ab^\ell+cb^{\ell-1}\le\bradix[w]{j\le k\le\ell}\le b^{\ell+1}-1 $$
and multiplying the upper ends
$$ \bradix[w]{j\le k\le\ell}\left(\frac{1}{\denom{j}{\ell}}-\frac{1}{\denom{i}{\ell}}+1\right)
	\leq\left(b^{\ell+1}-1\right)\left(\frac{1}{ab^\ell+cb^{\ell-1}}-\frac{1}{b^\ell-1}+1\right) $$
which when additively inverted almost gets us there
$$ \left(b^{\ell+1}-1\right)\left(\frac{1}{b^\ell-1}-\frac{1}{ab^\ell+cb^{\ell-1}}-1\right)
	\leq\bradix[w]{j\le k\le\ell}\left(\frac{1}{\denom{i}{\ell}}-\frac{1}{\denom{j}{\ell}}-1\right) $$
one step further
$$ \left(b^{\ell+1}-1\right)\left(\frac{1}{b^\ell-1}-\frac{1}{ab^\ell+cb^{\ell-1}}-1\right)+\bradix[w]{j\le k\le\ell}
	\leq\bradix[w]{j\le k\le\ell}\left(\frac{1}{\denom{i}{\ell}}-\frac{1}{\denom{j}{\ell}}\right) $$
and finally
$$ \frac{b^{\ell+1}-1}{b^\ell-1}-\frac{b^{\ell+1}-1}{ab^\ell+cb^{\ell-1}}-b^{\ell+1}+ab^\ell+cb^{\ell-1}+1
	\leq\bradix[w]{j\le k\le\ell}\left(\frac{1}{\denom{i}{\ell}}-\frac{1}{\denom{j}{\ell}}\right) $$

%%%%%%%%%%%%%%%%%%%%%%%%%%%%%%%%%%%%%%%%%%%%%%%%%%%%%%%%%%%%%%%%%%%%%%%%%%%%%%%%%%%%%%%%%%%%%%%%%%%%%%%%%%%%%
in it, I do point out how equality identities don't necessarily preserve inequality bounds. This is to say,
you can reshape an error bound with nothing else than an identity substitution, such as the following:
$$ \frac{\numer{j}{\ell}+1}{\denom{j}{\ell}}-1-\frac{\numer{i}{\ell}}{\denom{i}{\ell}}
	\quad=\qquad\frac{\numer{j}{\ell}-\numer{i}{\ell}}{\denom{j}{\ell}}
	+\bradix[w]{i\le k\le\ell}\left(\frac{1}{\denom{j}{\ell}}-\frac{1}{\denom{i}{\ell}}\right)
	+\frac{1}{\denom{j}{\ell}}-1 $$
the advantage of breaking it up like this is being able to primarily focus separately on the numerators:
$$ \qquad\frac{\numer{j}{\ell}-\numer{i}{\ell}}{\denom{j}{\ell}}\qquad (l) $$
or the denominators
$$ \bradix[w]{i\le k\le\ell}\left(\frac{1}{\denom{j}{\ell}}-\frac{1}{\denom{i}{\ell}}\right)\qquad (m) $$
or leftovers
$$ \qquad\frac{1}{\denom{j}{\ell}}-1\qquad (r) $$

We start with the numerators, with the following straightforward inequality:
$$ 0\le\bradix[w]{i\le k\le\ell}-\bradix[w]{j\le k\le\ell}\le b^j-1 $$
we will also be needing the basic inequality:
$$ \bunderseq[y]{\ell-1}b^{\ell-1}\le\bradix[y]{j\le k < \ell}\le b^\ell-1 $$
which works because we know our denominator cannot equal zero: $ \denom{j}{\ell}\neq 0 $, and in
particular, we also know by assumption that our leading digit is also non-zero: $ \bseq[y]_{\ell-1}\neq 0 $.
We multiplicatively invert to obtain:
$$ \frac{1}{b^\ell-1}\leq\frac{1}{\denom{j}{\ell}}\leq\frac{1}{\bseq[y]_{\ell-1}b^{\ell-1}}\qquad (4) $$

Going back to $ (3) $, we multiply by $ (4) $ to extend our bound as follows:
$$ 0\leq\frac{\numer{i}{\ell}-\numer{j}{\ell}}{\denom{j}{\ell}}
	\leq\frac{b^j-1}{\bseq[y]_{\ell-1}b^{\ell-1}} $$
additively inverting this
$$ -\frac{b^j-1}{\bseq[y]_{\ell-1}b^{\ell-1}}
	\leq\frac{\numer{j}{\ell}-\numer{i}{\ell}}{\denom{j}{\ell}}
	\leq 0\qquad (l') $$

Changing our focus now toward the middle of our original identity:
$$ \bradix[w]{i\le k\le\ell}\left(\frac{1}{\denom{j}{\ell}}-\frac{1}{\denom{i}{\ell}}\right) $$
we first simplify fractional part:
$$ \frac{1}{\denom{j}{\ell}}-\frac{1}{\denom{i}{\ell}}=\frac{\denom{i}{j}}{\denom{j}{\ell}\denom{i}{\ell}} $$
Recalling the previous derivation $ (4) $, if we create a copy replacing the index $ j $ with $ i $ and multiply, we obtain:
$$ \frac{1}{(b^\ell-1)^2}\leq\frac{1}{\denom{j}{\ell}\denom{i}{\ell}}\leq\frac{1}{\bseq[y]_{\ell-1}^2b^{2\ell-2}}\qquad (6) $$
Furthermore we can safely assume $ \denom{i}{\ell}\neq\denom{j}{\ell} $ since we intentionally seek an
imperfect approximation---it makes no sense for equality to exist---which leaves us with
$$ 0 < \bradix[y]{i\le k < \ell}-\bradix[y]{j\le k < \ell}
	\quad\Longrightarrow\quad 1\le\bradix[y]{i\le k < j}\le b^j\qquad (7) $$
multiplying $ (6) $ and $ (7) $ together we derive the next bound in our sequence:
$$ \frac{1}{(b^\ell-1)^2}
	\leq\frac{\denom{i}{j}}{\denom{j}{\ell}\denom{i}{\ell}}
	\leq\frac{b^j}{\bseq[y]_{\ell-1}^2b^{2\ell-2}} $$
By assumption we know that $ \denom{i}{\ell}\le\numer{i}{\ell} $ and so we have
$$ \bunderseq[y]{\ell-1}b^{\ell-1}\le\bradix[w]{i\le k\le\ell}\le b^{\ell+1}-1 $$
multiplying this and the former together we derive our right bound
$$ \frac{\bseq[y]_{\ell-1}b^{\ell-1}}{(b^\ell-1)^2}
	\leq\bradix[w]{i\le k\le\ell}\left(\frac{1}{\denom{j}{\ell}}-\frac{1}{\denom{i}{\ell}}\right)
	\leq\frac{(b^{\ell+1}-1)b^j}{\bseq[y]_{\ell-1}^2b^{2\ell-2}}\qquad (m') $$

Lastly we look at the right side of this original identity:
$$ \qquad\frac{1}{\denom{j}{\ell}}-1 $$
which is equivalent to
$$ \qquad\frac{1-\denom{j}{\ell}}{\denom{j}{\ell}} $$
$$ \bunderseq[y]{\ell-1}b^{\ell-1}-1
	\le\bradix[y]{j\le k < \ell}-1
	\le b^\ell-2 $$
$$ \frac{\bseq[y]_{\ell-1}b^{\ell-1}-1}{b^\ell-1}
	\leq\frac{\denom{j}{\ell}-1}{\denom{j}{\ell}}
	\leq\frac{b^\ell-2}{\bseq[y]_{\ell-1}b^{\ell-1}} $$
when inverted
$$ -\frac{b^\ell-2}{\bseq[y]_{\ell-1}b^{\ell-1}}
	\leq\frac{1-\denom{j}{\ell}}{\denom{j}{\ell}}
	\leq-\frac{\bseq[y]_{\ell-1}b^{\ell-1}-1}{b^\ell-1}\qquad (r') $$

$$ \frac{1}{b^\ell-1}
	\leq\frac{1}{\denom{j}{\ell}}
	\leq\frac{1}{\bseq[y]_{\ell-1}b^{\ell-1}} $$
$$ \frac{1}{b^\ell-1}-1
	\leq\frac{1}{\denom{j}{\ell}}-1
	\leq\frac{1}{\bseq[y]_{\ell-1}b^{\ell-1}}-1 $$

Finally adding $ (l') $ and $ (m') $ and $ (r') $ together we obtain the first serious bound for our approximation:

$$ \frac{\bseq[y]_{\ell-1}b^{\ell-1}}{(b^\ell-1)^2}-\frac{b^j-1}{\bseq[y]_{\ell-1}b^{\ell-1}}-\frac{b^\ell-2}{\bseq[y]_{\ell-1}b^{\ell-1}}
	\leq\frac{\numer{j}{\ell}+1}{\denom{j}{\ell}}-1-\frac{\numer{i}{\ell}}{\denom{i}{\ell}}
	\leq\frac{(b^{\ell+1}-1)b^j}{\bseq[y]_{\ell-1}^2b^{2\ell-2}}-\frac{\bseq[y]_{\ell-1}b^{\ell-1}-1}{b^\ell-1} $$

$$ \frac{\bseq[y]_{\ell-1}b^{\ell-1}}{(b^\ell-1)^2}-\frac{b^j-1}{\bseq[y]_{\ell-1}b^{\ell-1}}+\frac{1}{b^\ell-1}-1
	\leq\frac{\numer{j}{\ell}+1}{\denom{j}{\ell}}-1-\frac{\numer{i}{\ell}}{\denom{i}{\ell}}
	\leq\frac{(b^{\ell+1}-1)b^j}{\bseq[y]_{\ell-1}^2b^{2\ell-2}}+\frac{1}{\bseq[y]_{\ell-1}b^{\ell-1}}-1 $$

which admittedly is a lot to take in when looking at it. Please take a moment to go over the steps thoroughly and convince yourself
of this derivation before we move forward\footnote{If you see a problem in the proof up to this point or even further along,
I would be more than happy to know. Thank you.}.

Admittedly we are only interested in the lower end of this particular bound
$$ \frac{\bseq[y]_{\ell-1}b^{\ell-1}}{(b^\ell-1)^2}-\frac{b^j-1}{\bseq[y]_{\ell-1}b^{\ell-1}}-\frac{b^\ell-2}{\bseq[y]_{\ell-1}b^{\ell-1}}
	\leq\frac{\numer{j}{\ell}+1}{\denom{j}{\ell}}-1-\frac{\numer{i}{\ell}}{\denom{i}{\ell}} $$
as we have yet to derive the upper bound separately. In fact this is how such inequality manipulations work: maintain the chain
until the final moment where you usually only need the upper or lower bound but not both.  You have to maintain the chain because
you will likely be performing additive and multiplicative inversions, and complete information is needed for such manipulations.

We are now interested in the upper half of our original bound:
$$ \frac{\numer{j}{\ell}}{\denom{j}{\ell}}+1-\frac{\numer{i}{\ell}+1}{\denom{i}{\ell}} $$
for which we have the identity:
$$ \frac{\numer{j}{\ell}}{\denom{j}{\ell}}+1-\frac{\numer{i}{\ell}+1}{\denom{i}{\ell}}
	\quad=\qquad\bradix[w]{j\le k\le\ell}\left(\frac{1}{\denom{j}{\ell}}-\frac{1}{\denom{i}{\ell}}\right)
	+\frac{\numer{j}{\ell}-\numer{i}{\ell}-1}{\denom{i}{\ell}}+1 $$

Actually, for the curious, I'd like to make a {\bfseries side note} here. As it is, one can also derive the alternate identity:
$$ \frac{\numer{j}{\ell}}{\denom{j}{\ell}}+1-\frac{\numer{i}{\ell}+1}{\denom{i}{\ell}}
	\quad=\qquad\frac{\denom{i}{\ell}\numer{j}{\ell}+\denom{i}{\ell}\denom{j}{\ell}-\denom{j}{\ell}[\numer{i}{\ell}+1]}
		{\denom{i}{\ell}\denom{j}{\ell}} $$
which privileges multiplication as the \emph{root} operator rather than addition. I've pointed out in the \emph{inequality.pdf}
article that multiplicative forms can provide improved bounds over additive forms of an identity, the problem here is some of the
bounds derived in the process of creating a chain here don't satisfy the assumptions needed for the rules of inequality manipulation.
Otherwise I would have likely used this identity instead.

Getting back to it: We now focus on the left side similar to the previous derivation:
$$ \bradix[w]{j\le k\le\ell}\left(\frac{1}{\denom{j}{\ell}}-\frac{1}{\denom{i}{\ell}}\right) $$
Luckily for us, from our previous derivations we already know that
$$ \frac{1}{(b^\ell-1)^2}
	\leq\frac{1}{\denom{j}{\ell}}-\frac{1}{\denom{i}{\ell}}
	\leq\frac{b^j}{\bseq[y]_{\ell-1}^2b^{2\ell-2}} $$
and
$$ \bunderseq[y]{\ell-1}b^{\ell-1}\le\bradix[w]{j\le k\le\ell}\le b^{\ell+1}-1 $$
so we can skip the work and get straight to the result:
$$ \frac{\bseq[y]_{\ell-1}b^{\ell-1}}{(b^\ell-1)^2}
	\leq\bradix[w]{j\le k\le\ell}\left(\frac{1}{\denom{j}{\ell}}-\frac{1}{\denom{i}{\ell}}\right)
	\leq\frac{(b^{\ell+1}-1)b^j}{\bseq[y]_{\ell-1}^2b^{2\ell-2}}\qquad (l'') $$

As for our right-hand side:
$$ \frac{\numer{j}{\ell}-\numer{i}{\ell}-1}{\denom{i}{\ell}} $$
it is also similar to the previous derivations. If we replace the index of $ (4) $ we immediately have:
$$ \frac{1}{b^\ell-1}\leq\frac{1}{\denom{i}{\ell}}\leq\frac{1}{\bseq[y]_{\ell-1}b^{\ell-1}}\qquad (8) $$
and with $ (3) $ from even earlier
$$ 1\le\bradix[w]{i\le k \le\ell}+1-\bradix[w]{j\le k\le\ell}\le b^j $$
we can multiply this by $ (8) $ to obtain
$$ \frac{1}{b^\ell-1}\leq\frac{\numer{i}{\ell}+1-\numer{j}{\ell}}{\denom{i}{\ell}}\leq\frac{b^j}{\bseq[y]_{\ell-1}b^{\ell-1}} $$
which when itself is additively inverted results in the second bound of our above identity:
$$ -\frac{b^j}{\bseq[y]_{\ell-1}b^{\ell-1}}
	\leq\frac{\numer{j}{\ell}-\numer{i}{\ell}-1}{\denom{i}{\ell}}
	\leq-\frac{1}{b^\ell-1}\qquad (r'') $$
which when adding $ (l'') $ and $ (r'') $ as well as adding $ 1 $ more, we obtain:
$$ \frac{\bseq[y]_{\ell-1}b^{\ell-1}}{(b^\ell-1)^2}-\frac{b^j}{\bseq[y]_{\ell-1}b^{\ell-1}}+1
	\leq\frac{\numer{j}{\ell}}{\denom{j}{\ell}}+1-\frac{\numer{i}{\ell}+1}{\denom{i}{\ell}}
	\leq\frac{(b^{\ell+1}-1)b^j}{\bseq[y]_{\ell-1}^2b^{2\ell-2}}-\frac{1}{b^\ell-1}+1 $$
where of course we are only really interested in the upper half:
$$ \frac{\numer{j}{\ell}}{\denom{j}{\ell}}+1-\frac{\numer{i}{\ell}+1}{\denom{i}{\ell}}
	\leq\frac{(b^{\ell+1}-1)b^j}{\bseq[y]_{\ell-1}^2b^{2\ell-2}}-\frac{1}{b^\ell-1}+1 $$

We now have enough, and by using the transitive law we have possibly weaker but more independent bounds on our original relation:
$$ \frac{\bseq[y]_{\ell-1}b^{\ell-1}}{(b^\ell-1)^2}-\frac{b^j-1}{\bseq[y]_{\ell-1}b^{\ell-1}}-\frac{b^\ell-2}{\bseq[y]_{\ell-1}b^{\ell-1}}
	\leq\left\lfloor\frac{\numer{j}{\ell}}{\denom{j}{\ell}}\right\rfloor
		-\left\lfloor\frac{\numer{i}{\ell}}{\denom{i}{\ell}}\right\rfloor
	\leq\frac{(b^{\ell+1}-1)b^j}{\bseq[y]_{\ell-1}^2b^{2\ell-2}}-\frac{1}{b^\ell-1}+1 $$

\subsection*{refinements}

We have a few subtleties to consider actually. To begin, only midway did we add the assumption that
$ \denom{i}{\ell}\le\numer{i}{\ell} $, but looking more closely it's possible in this case the leading digit of $ \bm(w) $ is zero.
In our intuitive example this wouldn't happen because our divisor was $ 99 $, as large as possible under the constraint of two digits.
If it had been smaller, say $ 12 $, then the first two digits of our dividend $ 20 $ would have been large enough that we wouldn't
have needed the extra step of adding the next digit $ 200 $. In this case you can think of our partial dividend not as $ 20 $ but
instead as $ 020 $. Same value, but three digits now. If you go over the steps in this derivation you will discover it does not
contradict our result.

This resolves the first subtlety in that \emph{our derived bound works for the seperate conditional case within our algorithm} as well.
The second subtlety is it is now time to make compromises on our bound. To be fair, the above bound is still incomplete as we intended
to derive a bound where both the lower and upper were entirely independent of the numerator and denominator values in the middle of
this ordered relation. It is not yet independent because we have the $ \bseq[y]_{\ell-1} $ value embedded within, and so it is here
we make our compromise: We will add the following assumption
$$ \left\lfloor\frac{b}{2}\right\rfloor\leq\bunderseq[y]{\ell-1} $$
but if we intend to use this to further our bound derivation, we need this to be in the form of a chain. Fortunately, we already
know firstly $ \bseq[y]_{\ell-1} < b $. Secondly, given our ``floor function'' lemma from before, we then have
$$ \frac{b-1}{2}\leq\bunderseq[y]{\ell-1}\leq b-1 $$
this bound we also need in the form of a multiplicative inverse
$$ \frac{1}{b-1}\leq\frac{1}{\bseq[y]_{\ell-1}}\leq\frac{2}{b-1} $$

We have what we need to begin. Starting with the lower bound:
$$ \frac{\bseq[y]_{\ell-1}b^{\ell-1}}{(b^\ell-1)^2}-\frac{b^j-1}{\bseq[y]_{\ell-1}b^{\ell-1}}-\frac{b^\ell-2}{\bseq[y]_{\ell-1}b^{\ell-1}} $$
$$ \frac{\bseq[y]_{\ell-1}b^{\ell-1}}{(b^\ell-1)^2}-\frac{b^j-1}{\bseq[y]_{\ell-1}b^{\ell-1}}+\frac{1}{b^\ell-1}-1 $$
which first simplifies to
$$ \frac{\bseq[y]_{\ell-1}b^{\ell-1}}{(b^\ell-1)^2}-\frac{b^j+b^\ell-3}{\bseq[y]_{\ell-1}b^{\ell-1}} $$
we divide and consensus:
$$ \frac{1}{b-1}\leq\frac{1}{\bseq[y]_{\ell-1}}\leq\frac{2}{b-1} $$
becomes
$$ \frac{b^j+b^\ell-3}{(b-1)b^{\ell-1}}
	\leq\frac{b^j+b^\ell-3}{\bseq[y]_{\ell-1}b^{\ell-1}}
	\leq\frac{2(b^j+b^\ell-3)}{(b-1)b^{\ell-1}} $$
which in turn becomes
$$ -\frac{2(b^j+b^\ell-3)}{(b-1)b^{\ell-1}}
	\leq-\frac{b^j+b^\ell-3}{\bseq[y]_{\ell-1}b^{\ell-1}}
	\leq-\frac{b^j+b^\ell-3}{(b-1)b^{\ell-1}} $$

$$ \frac{b^j-1}{(b-1)b^{\ell-1}}
	\leq\frac{b^j-1}{\bseq[y]_{\ell-1}b^{\ell-1}}
	\leq\frac{2(b^j-1)}{(b-1)b^{\ell-1}} $$
which in turn becomes
$$ -\frac{2(b^j-1)}{(b-1)b^{\ell-1}}
	\leq-\frac{b^j-1}{\bseq[y]_{\ell-1}b^{\ell-1}}
	\leq-\frac{b^j-1}{(b-1)b^{\ell-1}} $$


Moving on:
$$ \frac{b-1}{2}\leq\bunderseq[y]{\ell-1}\leq b-1 $$
becomes
$$ \frac{(b-1)b^{\ell-1}}{2(b^\ell-1)^2}
	\leq\frac{\bseq[y]_{\ell-1}b^{\ell-1}}{(b^\ell-1)^2}
	\leq\frac{(b-1)b^{\ell-1}}{(b^\ell-1)^2} $$
and so adding both of these together:
$$ \frac{(b-1)b^{\ell-1}}{2(b^\ell-1)^2}-\frac{2(b^j+b^\ell-3)}{(b-1)b^{\ell-1}}
	\leq\frac{\bseq[y]_{\ell-1}b^{\ell-1}}{(b^\ell-1)^2}-\frac{b^j+b^\ell-3}{\bseq[y]_{\ell-1}b^{\ell-1}}
	\leq\frac{(b-1)b^{\ell-1}}{(b^\ell-1)^2}-\frac{b^j+b^\ell-3}{(b-1)b^{\ell-1}} $$

$$ \frac{(b-1)b^{\ell-1}}{2(b^\ell-1)^2}-\frac{2(b^j-1)}{(b-1)b^{\ell-1}}+\frac{1}{b^\ell-1}-1
	\leq\frac{\bseq[y]_{\ell-1}b^{\ell-1}}{(b^\ell-1)^2}-\frac{b^j+b^\ell-3}{\bseq[y]_{\ell-1}b^{\ell-1}}
	\leq\frac{(b-1)b^{\ell-1}}{(b^\ell-1)^2}-\frac{b^j+b^\ell-3}{(b-1)b^{\ell-1}} $$

The upper bound is a tiny bit simpler:
$$ \frac{(b^{\ell+1}-1)b^j}{\bseq[y]_{\ell-1}^2b^{2\ell-2}}-\frac{1}{b^\ell-1}+1 $$
first noting
$$ \frac{1}{(b-1)^2}\leq\frac{1}{\bseq[y]_{\ell-1}^2}\leq\frac{4}{(b-1)^2} $$
since
$$ \frac{1}{b-1}\leq\frac{1}{\bseq[y]_{\ell-1}}\leq\frac{2}{b-1} $$
and so
$$ \frac{(b^{\ell+1}-1)b^j}{(b-1)^2b^{2\ell-2}}
	\leq\frac{(b^{\ell+1}-1)b^j}{\bseq[y]_{\ell-1}^2b^{2\ell-2}}
	\leq\frac{4(b^{\ell+1}-1)b^j}{(b-1)^2b^{2\ell-2}} $$
along with
$$ \frac{(b^{\ell+1}-1)b^j}{(b-1)^2b^{2\ell-2}}-\frac{1}{b^\ell-1}+1
	\leq\frac{(b^{\ell+1}-1)b^j}{\bseq[y]_{\ell-1}^2b^{2\ell-2}}-\frac{1}{b^\ell-1}+1
	\leq\frac{4(b^{\ell+1}-1)b^j}{(b-1)^2b^{2\ell-2}}-\frac{1}{b^\ell-1}+1 $$

Finally, we have our refined---if not ugly--bound:
$$ \frac{2}{b^\ell-1}+\frac{(b-1)b^{\ell-1}}{2(b^\ell-1)^2}-\frac{2b^j}{(b-1)b^{\ell-1}}-1
	\leq\left\lfloor\frac{\numer{j}{\ell}}{\denom{j}{\ell}}\right\rfloor
		-\left\lfloor\frac{\numer{i}{\ell}}{\denom{i}{\ell}}\right\rfloor
	\leq\frac{4(b^{\ell+1}-1)b^j}{(b-1)^2b^{2\ell-2}}-\frac{1}{b^\ell-1}+1 $$

for the experienced inequality manipulator, you should also be able to verify this bound can be further simplified as:
$$ -1\ \le\ \left\lfloor\frac{\numer{j}{\ell}}{\denom{j}{\ell}}\right\rfloor
		-\left\lfloor\frac{\numer{i}{\ell}}{\denom{i}{\ell}}\right\rfloor
	\ \le\ 1 $$
keep in mind as we are assuming our divisor has two or more digits we are using the additional assumption that $ j\le\ell+2 $.

In the special case of $ i=0 $ the above simplifies further to our quotient of interest,
$$ -1\ \le\ \left\lfloor\frac{\numer{j}{\ell}}{\denom{j}{\ell}}\right\rfloor-q\ \le\ 1 $$
which of course can be rearranged to put the center of attention on said quotient:
$$ \left\lfloor\frac{\numer{j}{\ell}}{\denom{j}{\ell}}\right\rfloor-1
	\le\ \! q\ \le\ \left\lfloor\frac{\numer{j}{\ell}}{\denom{j}{\ell}}\right\rfloor+1 $$

To reiterate: This bound allows us to narrow down in advance our calculation of the quotient for arbitrary precision digit divisors
(two or more digits) to one of three possible cases. At that point, it's guess and check, but a constant bound is far better than
logarithmic or even linear.

And yet, we can still do just a little bit better.

\subsection*{cleanup}


\newpage

%%%%%%%%%%%%%%%%%%%%%%%%%%%%%%%%%%%%%%%%%%%%%%%%%%%%%%%%%%%%%%%%%%%%%%%%%%%%%%%%%%%%%%%%%%%%%%%%%%%%%%%%%%%%%
we will be needing the basic inequality:
$$ \bunderseq[y]{\ell-1}b^{\ell-1}\le\bradix[y]{j\le k < \ell}\le b^\ell-1\qquad (3) $$
we know by assumption that our leading digit is non-zero: $ \bseq[y]_{\ell-1}\neq 0 $ (and thus $ \denom{j}{\ell}\neq 0 $).
We multiplicatively invert to obtain:
$$ \frac{1}{b^\ell-1}\leq\frac{1}{\denom{j}{\ell}}\leq\frac{1}{\bseq[y]_{\ell-1}b^{\ell-1}}\qquad (4) $$
and for our purposes will will also want the separate flavors:
$$ \bunderseq[y]{\ell-1}b^{\ell-1}\le\bradix[y]{i\le k < \ell}\le b^\ell-1 $$
$$ \frac{1}{b^\ell-1}\leq\frac{1}{\denom{i}{\ell}}\leq\frac{1}{\bseq[y]_{\ell-1}b^{\ell-1}} $$
looking at our numerator:
$$ \bunderseq[y]{\ell-1}b^{\ell-1}\le\bradix[w]{j\le k\le\ell}\le b^{\ell+1}-1 $$
which also holds under its replaced index
$$ \bunderseq[y]{\ell-1}b^{\ell-1}\le\bradix[w]{i\le k\le\ell}\le b^{\ell+1}-1 $$
we shift by one
$$ \bunderseq[y]{\ell-1}b^{\ell-1}+1\le\bradix[w]{j\le k\le\ell}+1\le b^{\ell+1} $$
multiplying together we get
$$ \frac{\bseq[y]_{\ell-1}b^{\ell-1}+1}{b^\ell-1}
	\leq\frac{\numer{j}{\ell}+1}{\denom{j}{\ell}}
	\leq\frac{b^{\ell+1}}{\bseq[y]_{\ell-1}b^{\ell-1}} $$
similarly we can do the same with the other index:
$$ \frac{\bseq[y]_{\ell-1}b^{\ell-1}}{b^\ell-1}
	\leq\frac{\numer{i}{\ell}}{\denom{i}{\ell}}
	\leq\frac{b^{\ell+1}-1}{\bseq[y]_{\ell-1}b^{\ell-1}} $$
but in this case we need to additively invert to get the other term,
$$ -\frac{b^{\ell+1}-1}{\bseq[y]_{\ell-1}b^{\ell-1}}
	\leq-\frac{\numer{i}{\ell}}{\denom{i}{\ell}}
	\leq-\frac{\bseq[y]_{\ell-1}b^{\ell-1}}{b^\ell-1} $$
and if we add these together while shifting left we get our first bound
$$ \frac{\bseq[y]_{\ell-1}b^{\ell-1}+1}{b^\ell-1}-\frac{b^{\ell+1}-1}{\bseq[y]_{\ell-1}b^{\ell-1}}-1
	\leq\frac{\numer{j}{\ell}+1}{\denom{j}{\ell}}-1-\frac{\numer{i}{\ell}}{\denom{i}{\ell}}
	\leq\frac{b^{\ell+1}}{\bseq[y]_{\ell-1}b^{\ell-1}}-\frac{\bseq[y]_{\ell-1}b^{\ell-1}}{b^\ell-1}-1 $$

Please take a moment to go over the steps thoroughly and convince yourself of this derivation before we move
forward\footnote{If you see a problem in the proof up to this point or even further along, I would be more
than happy to know. Thank you.}.

Admittedly we are only interested in the lower end of this particular bound
$$ \frac{\bseq[y]_{\ell-1}b^{\ell-1}+1}{b^\ell-1}-\frac{b^{\ell+1}-1}{\bseq[y]_{\ell-1}b^{\ell-1}}-1
	\leq\frac{\numer{j}{\ell}+1}{\denom{j}{\ell}}-1-\frac{\numer{i}{\ell}}{\denom{i}{\ell}} $$
as we have yet to derive the upper bound separately. In fact this is how such inequality manipulations work: maintain the chain
until the final moment where you usually only need the upper or lower bound but not both.  You have to maintain the chain because
you will likely be performing additive and multiplicative inversions, and complete information is needed for such manipulations.

We are now interested in the upper half of our original bound:
$$ \frac{\numer{j}{\ell}}{\denom{j}{\ell}}+1-\frac{\numer{i}{\ell}+1}{\denom{i}{\ell}} $$
for which we have the identity:
$$ \frac{\numer{j}{\ell}}{\denom{j}{\ell}}+1-\frac{\numer{i}{\ell}+1}{\denom{i}{\ell}}
	\quad=\qquad\bradix[w]{j\le k\le\ell}\left(\frac{1}{\denom{j}{\ell}}-\frac{1}{\denom{i}{\ell}}\right)
	+\frac{\numer{j}{\ell}-\numer{i}{\ell}-1}{\denom{i}{\ell}}+1 $$

Actually, for the curious, I'd like to make a {\bfseries side note} here. As it is, one can also derive the alternate identity:
$$ \frac{\numer{j}{\ell}}{\denom{j}{\ell}}+1-\frac{\numer{i}{\ell}+1}{\denom{i}{\ell}}
	\quad=\qquad\frac{\denom{i}{\ell}\numer{j}{\ell}+\denom{i}{\ell}\denom{j}{\ell}-\denom{j}{\ell}[\numer{i}{\ell}+1]}
		{\denom{i}{\ell}\denom{j}{\ell}} $$
which privileges multiplication as the \emph{root} operator rather than addition. I've pointed out in the \emph{inequality.pdf}
article that multiplicative forms can provide improved bounds over additive forms of an identity, the problem here is some of the
bounds derived in the process of creating a chain here don't satisfy the assumptions needed for the rules of inequality manipulation.
Otherwise I would have likely used this identity instead.

Getting back to it: We now focus on the left side similar to the previous derivation:
$$ \bradix[w]{j\le k\le\ell}\left(\frac{1}{\denom{j}{\ell}}-\frac{1}{\denom{i}{\ell}}\right) $$
Luckily for us, from our previous derivations we already know that
$$ \frac{1}{(b^\ell-1)^2}
	\leq\frac{1}{\denom{j}{\ell}}-\frac{1}{\denom{i}{\ell}}
	\leq\frac{b^j}{\bseq[y]_{\ell-1}^2b^{2\ell-2}} $$
and
$$ \bunderseq[y]{\ell-1}b^{\ell-1}\le\bradix[w]{j\le k\le\ell}\le b^{\ell+1}-1 $$
so we can skip the work and get straight to the result:
$$ \frac{\bseq[y]_{\ell-1}b^{\ell-1}}{(b^\ell-1)^2}
	\leq\bradix[w]{j\le k\le\ell}\left(\frac{1}{\denom{j}{\ell}}-\frac{1}{\denom{i}{\ell}}\right)
	\leq\frac{(b^{\ell+1}-1)b^j}{\bseq[y]_{\ell-1}^2b^{2\ell-2}}\qquad (l'') $$

As for our right-hand side:
$$ \frac{\numer{j}{\ell}-\numer{i}{\ell}-1}{\denom{i}{\ell}} $$
it is also similar to the previous derivations. If we replace the index of $ (4) $ we immediately have:
$$ \frac{1}{b^\ell-1}\leq\frac{1}{\denom{i}{\ell}}\leq\frac{1}{\bseq[y]_{\ell-1}b^{\ell-1}}\qquad (8) $$
and with $ (3) $ from even earlier
$$ 1\le\bradix[w]{i\le k \le\ell}+1-\bradix[w]{j\le k\le\ell}\le b^j $$
we can multiply this by $ (8) $ to obtain
$$ \frac{1}{b^\ell-1}\leq\frac{\numer{i}{\ell}+1-\numer{j}{\ell}}{\denom{i}{\ell}}\leq\frac{b^j}{\bseq[y]_{\ell-1}b^{\ell-1}} $$
which when itself is additively inverted results in the second bound of our above identity:
$$ -\frac{b^j}{\bseq[y]_{\ell-1}b^{\ell-1}}
	\leq\frac{\numer{j}{\ell}-\numer{i}{\ell}-1}{\denom{i}{\ell}}
	\leq-\frac{1}{b^\ell-1}\qquad (r'') $$
which when adding $ (l'') $ and $ (r'') $ as well as adding $ 1 $ more, we obtain:
$$ \frac{\bseq[y]_{\ell-1}b^{\ell-1}}{(b^\ell-1)^2}-\frac{b^j}{\bseq[y]_{\ell-1}b^{\ell-1}}+1
	\leq\frac{\numer{j}{\ell}}{\denom{j}{\ell}}+1-\frac{\numer{i}{\ell}+1}{\denom{i}{\ell}}
	\leq\frac{(b^{\ell+1}-1)b^j}{\bseq[y]_{\ell-1}^2b^{2\ell-2}}-\frac{1}{b^\ell-1}+1 $$
where of course we are only really interested in the upper half:
$$ \frac{\numer{j}{\ell}}{\denom{j}{\ell}}+1-\frac{\numer{i}{\ell}+1}{\denom{i}{\ell}}
	\leq\frac{(b^{\ell+1}-1)b^j}{\bseq[y]_{\ell-1}^2b^{2\ell-2}}-\frac{1}{b^\ell-1}+1 $$

We now have enough, and by using the transitive law we have possibly weaker but more independent bounds on our original relation:
$$ \frac{\bseq[y]_{\ell-1}b^{\ell-1}+1}{b^\ell-1}-\frac{b^{\ell+1}-1}{\bseq[y]_{\ell-1}b^{\ell-1}}-1
	\leq\left\lfloor\frac{\numer{j}{\ell}}{\denom{j}{\ell}}\right\rfloor
		-\left\lfloor\frac{\numer{i}{\ell}}{\denom{i}{\ell}}\right\rfloor
	\leq\frac{(b^{\ell+1}-1)b^j}{\bseq[y]_{\ell-1}^2b^{2\ell-2}}-\frac{1}{b^\ell-1}+1 $$

\subsection*{refinements}

We have a few subtleties to consider actually. To begin, only midway did we add the assumption that
$ \denom{i}{\ell}\le\numer{i}{\ell} $, but looking more closely it's possible in this case the leading digit of $ \bm(w) $ is zero.
In our intuitive example this wouldn't happen because our divisor was $ 99 $, as large as possible under the constraint of two digits.
If it had been smaller, say $ 12 $, then the first two digits of our dividend $ 20 $ would have been large enough that we wouldn't
have needed the extra step of adding the next digit $ 200 $. In this case you can think of our partial dividend not as $ 20 $ but
instead as $ 020 $. Same value, but three digits now. If you go over the steps in this derivation you will discover it does not
contradict our result.

This resolves the first subtlety in that \emph{our derived bound works for the seperate conditional case within our algorithm} as well.
The second subtlety is it is now time to make compromises on our bound. To be fair, the above bound is still incomplete as we intended
to derive a bound where both the lower and upper were entirely independent of the numerator and denominator values in the middle of
this ordered relation. It is not yet independent because we have the $ \bseq[y]_{\ell-1} $ value embedded within, and so it is here
we make our compromise: We will add the following assumption
$$ \left\lfloor\frac{b}{2}\right\rfloor\leq\bunderseq[y]{\ell-1} $$
but if we intend to use this to further our bound derivation, we need this to be in the form of a chain. Fortunately, we already
know firstly $ \bseq[y]_{\ell-1} < b $. Secondly, given our ``floor function'' lemma from before, we then have
$$ \frac{b-1}{2}\leq\bunderseq[y]{\ell-1}\leq b-1 $$
this bound we also need in the form of a multiplicative inverse
$$ \frac{1}{b-1}\leq\frac{1}{\bseq[y]_{\ell-1}}\leq\frac{2}{b-1} $$

We have what we need to begin. Starting with the lower bound:
$$ \frac{\bseq[y]_{\ell-1}b^{\ell-1}+1}{b^\ell-1}-\frac{b^{\ell+1}-1}{\bseq[y]_{\ell-1}b^{\ell-1}}-1 $$
we divide and get along:
$$ \frac{b-1}{2}\leq\bunderseq[y]{\ell-1}\leq b-1 $$
becomes
$$ \frac{(b-1)b^{\ell-1}}{2}\leq\bunderseq[y]{\ell-1}b^{\ell-1}\leq (b-1)b^{\ell-1} $$
which in turn becomes
$$ \frac{(b-1)b^{\ell-1}+2}{2}\leq\bunderseq[y]{\ell-1}b^{\ell-1}+1\leq (b-1)b^{\ell-1}+1 $$
which itself in turn becomes
$$ \frac{(b-1)b^{\ell-1}+2}{2(b^\ell-1)}
	\leq\frac{\bseq[y]_{\ell-1}b^{\ell-1}+1}{b^\ell-1}
	\leq\frac{(b-1)b^{\ell-1}+1}{b^\ell-1} $$

Moving on:
$$ \frac{1}{b-1}\leq\frac{1}{\bseq[y]_{\ell-1}}\leq\frac{2}{b-1} $$
becomes
$$ \frac{b^{\ell+1}-1}{(b-1)b^{\ell-1}}
	\leq\frac{b^{\ell+1}-1}{\bseq[y]_{\ell-1}b^{\ell-1}}
	\leq\frac{2(b^{\ell+1}-1)}{(b-1)b^{\ell-1}} $$
which in turn becomes
$$ -\frac{2(b^{\ell+1}-1)}{(b-1)b^{\ell-1}}
	\leq-\frac{b^{\ell+1}-1}{\bseq[y]_{\ell-1}b^{\ell-1}}
	\leq-\frac{b^{\ell+1}-1}{(b-1)b^{\ell-1}} $$

and so these together and again subtracting by one:
$$ \frac{(b-1)b^{\ell-1}+2}{2(b^\ell-1)}-\frac{2(b^{\ell+1}-1)}{(b-1)b^{\ell-1}}-1
	\leq\frac{\bseq[y]_{\ell-1}b^{\ell-1}+1}{b^\ell-1}-\frac{b^{\ell+1}-1}{\bseq[y]_{\ell-1}b^{\ell-1}}-1
	\leq\frac{(b-1)b^{\ell-1}+1}{b^\ell-1}-\frac{b^{\ell+1}-1}{(b-1)b^{\ell-1}}-1 $$

The upper bound is a tiny bit simpler:
$$ \frac{(b^{\ell+1}-1)b^j}{\bseq[y]_{\ell-1}^2b^{2\ell-2}}-\frac{1}{b^\ell-1}+1 $$
first noting
$$ \frac{1}{(b-1)^2}\leq\frac{1}{\bseq[y]_{\ell-1}^2}\leq\frac{4}{(b-1)^2} $$
since
$$ \frac{1}{b-1}\leq\frac{1}{\bseq[y]_{\ell-1}}\leq\frac{2}{b-1} $$
and so
$$ \frac{(b^{\ell+1}-1)b^j}{(b-1)^2b^{2\ell-2}}
	\leq\frac{(b^{\ell+1}-1)b^j}{\bseq[y]_{\ell-1}^2b^{2\ell-2}}
	\leq\frac{4(b^{\ell+1}-1)b^j}{(b-1)^2b^{2\ell-2}} $$
along with
$$ \frac{(b^{\ell+1}-1)b^j}{(b-1)^2b^{2\ell-2}}-\frac{1}{b^\ell-1}+1
	\leq\frac{(b^{\ell+1}-1)b^j}{\bseq[y]_{\ell-1}^2b^{2\ell-2}}-\frac{1}{b^\ell-1}+1
	\leq\frac{4(b^{\ell+1}-1)b^j}{(b-1)^2b^{2\ell-2}}-\frac{1}{b^\ell-1}+1 $$

Finally, we have our refined---if not ugly--bound:
$$ \frac{2}{b^\ell-1}+\frac{(b-1)b^{\ell-1}}{2(b^\ell-1)^2}-\frac{2b^j}{(b-1)b^{\ell-1}}-1
	\leq\left\lfloor\frac{\numer{j}{\ell}}{\denom{j}{\ell}}\right\rfloor
		-\left\lfloor\frac{\numer{i}{\ell}}{\denom{i}{\ell}}\right\rfloor
	\leq\frac{4(b^{\ell+1}-1)b^j}{(b-1)^2b^{2\ell-2}}-\frac{1}{b^\ell-1}+1 $$

for the experienced inequality manipulator, you should also be able to verify this bound can be further simplified as:
$$ -1\ \le\ \left\lfloor\frac{\numer{j}{\ell}}{\denom{j}{\ell}}\right\rfloor
		-\left\lfloor\frac{\numer{i}{\ell}}{\denom{i}{\ell}}\right\rfloor
	\ \le\ 1 $$
keep in mind as we are assuming our divisor has two or more digits we are using the additional assumption that $ j\le\ell+2 $.

In the special case of $ i=0 $ the above simplifies further to our quotient of interest,
$$ -1\ \le\ \left\lfloor\frac{\numer{j}{\ell}}{\denom{j}{\ell}}\right\rfloor-q\ \le\ 1 $$
which of course can be rearranged to put the center of attention on said quotient:
$$ \left\lfloor\frac{\numer{j}{\ell}}{\denom{j}{\ell}}\right\rfloor-1
	\le\ \! q\ \le\ \left\lfloor\frac{\numer{j}{\ell}}{\denom{j}{\ell}}\right\rfloor+1 $$

To reiterate: This bound allows us to narrow down in advance our calculation of the quotient for arbitrary precision digit divisors
(two or more digits) to one of three possible cases. At that point, it's guess and check, but a constant bound is far better than
logarithmic or even linear.

And yet, we can still do just a little bit better.

\subsection*{cleanup}


\newpage

%%%%%%%%%%%%%%%%%%%%%%%%%%%%%%%%%%%%%%%%%%%%%%%%%%%%%%%%%%%%%%%%%%%%%%%%%%%%%%%%%%%%%%%%%%%%%%%%%%%%%%%%%%%%%
in it, I do point out how equality identities don't necessarily preserve inequality bounds. This is to say,
you can reshape an error bound with nothing else than an identity substitution, such as the following:
$$ \frac{\numer{j}{\ell}+1}{\denom{j}{\ell}}-1-\frac{\numer{i}{\ell}}{\denom{i}{\ell}}
	\quad=\qquad\frac{\numer{j}{\ell}+1-\numer{i}{\ell}}{\denom{j}{\ell}}
	+\bradix[w]{i\le k\le\ell}\left(\frac{1}{\denom{j}{\ell}}-\frac{1}{\denom{i}{\ell}}\right)-1 $$
the advantage of breaking it up like this is being able to primarily focus separately on the numerators:
$$ \qquad\frac{\numer{j}{\ell}+1-\numer{i}{\ell}}{\denom{j}{\ell}}\qquad (l) $$
or the denominators
$$ \bradix[w]{i\le k\le\ell}\left(\frac{1}{\denom{j}{\ell}}-\frac{1}{\denom{i}{\ell}}\right)\qquad (r) $$

We start with the numerators, with the following straightforward inequality:
$$ 0\le\bradix[w]{i\le k\le\ell}-\bradix[w]{j\le k\le\ell}\le b^j-1 $$
as we need a ``+1'' term in the centre, we shift it right by one:
$$ 1\le\bradix[w]{i\le k \le\ell}+1-\bradix[w]{j\le k\le\ell}\le b^j\qquad (3) $$
we will also be needing the basic inequality:
$$ \bunderseq[y]{\ell-1}b^{\ell-1}\le\bradix[y]{j\le k < \ell}\le b^\ell-1 $$
which works because we know our denominator cannot equal zero: $ \denom{j}{\ell}\neq 0 $, and in
particular, we also know by assumption that our leading digit is also non-zero: $ \bseq[y]_{\ell-1}\neq 0 $.
We multiplicatively invert to obtain:
$$ \frac{1}{b^\ell-1}\leq\frac{1}{\denom{j}{\ell}}\leq\frac{1}{\bseq[y]_{\ell-1}b^{\ell-1}}\qquad (4) $$
and for our purposes will will also want to scale
$$ \frac{2}{b^\ell-1}\leq\frac{2}{\denom{j}{\ell}}\leq\frac{2}{\bseq[y]_{\ell-1}b^{\ell-1}} $$
as well as additively invert our bound
$$ \frac{-2}{\bseq[y]_{\ell-1}b^{\ell-1}}\leq\frac{-2}{\denom{j}{\ell}}\leq\frac{-2}{b^\ell-1}\qquad (5) $$

Going back to $ (3) $, we multiply by $ (4) $ to extend our bound as follows:
$$ \frac{1}{b^\ell-1}\leq\frac{\numer{i}{\ell}-\numer{j}{\ell}+1}{\denom{j}{\ell}}\leq\frac{b^j}{\bseq[y]_{\ell-1}b^{\ell-1}} $$
and finally we add to this our previous derivation $ (5) $ to obtain:
$$ \frac{1}{b^\ell-1}-\frac{2}{\bseq[y]_{\ell-1}b^{\ell-1}}
	\leq\frac{\numer{i}{\ell}-\numer{j}{\ell}-1}{\denom{j}{\ell}}
	\leq\frac{b^j}{\bseq[y]_{\ell-1}b^{\ell-1}}-\frac{2}{b^\ell-1} $$
which when itself is additively inverted results in the first term of our above identity $ (l) $:
$$ \frac{2}{b^\ell-1}-\frac{b^j}{\bseq[y]_{\ell-1}b^{\ell-1}}
	\leq\frac{\numer{j}{\ell}+1-\numer{i}{\ell}}{\denom{j}{\ell}}
	\leq\frac{2}{\bseq[y]_{\ell-1}b^{\ell-1}}-\frac{1}{b^\ell-1}\qquad (l') $$

Changing our focus now toward the right side of our original identity:
$$ \bradix[w]{i\le k\le\ell}\left(\frac{1}{\denom{j}{\ell}}-\frac{1}{\denom{i}{\ell}}\right)\qquad $$
we first simplify fractional part:
$$ \frac{1}{\denom{j}{\ell}}-\frac{1}{\denom{i}{\ell}}=\frac{\denom{i}{j}}{\denom{j}{\ell}\denom{i}{\ell}} $$
Recalling the previous derivation $ (4) $, if we create a copy replacing the index $ j $ with $ i $ and multiply, we obtain:
$$ \frac{1}{(b^\ell-1)^2}\leq\frac{1}{\denom{j}{\ell}\denom{i}{\ell}}\leq\frac{1}{\bseq[y]_{\ell-1}^2b^{2\ell-2}}\qquad (6) $$
Furthermore we can safely assume $ \denom{i}{\ell}\neq\denom{j}{\ell} $ since we intentionally seek an
imperfect approximation---it makes no sense for equality to exist---which leaves us with
$$ 0 < \bradix[y]{i\le k < \ell}-\bradix[y]{j\le k < \ell}
	\quad\Longrightarrow\quad 1\le\bradix[y]{i\le k < j}\le b^j\qquad (7) $$
multiplying $ (6) $ and $ (7) $ together we derive the next bound in our sequence:
$$ \frac{1}{(b^\ell-1)^2}
	\leq\frac{\denom{i}{j}}{\denom{j}{\ell}\denom{i}{\ell}}
	\leq\frac{b^j}{\bseq[y]_{\ell-1}^2b^{2\ell-2}} $$
By assumption we know that $ \denom{i}{\ell}\le\numer{i}{\ell} $ and so we have
$$ \bunderseq[y]{\ell-1}b^{\ell-1}\le\bradix[w]{i\le k\le\ell}\le b^{\ell+1}-1 $$
multiplying this and the former together we derive our right bound
$$ \frac{\bseq[y]_{\ell-1}b^{\ell-1}}{(b^\ell-1)^2}
	\leq\bradix[w]{i\le k\le\ell}\left(\frac{1}{\denom{j}{\ell}}-\frac{1}{\denom{i}{\ell}}\right)
	\leq\frac{(b^{\ell+1}-1)b^j}{\bseq[y]_{\ell-1}^2b^{2\ell-2}}\qquad (r') $$
finally adding $ (l') $ and $ (r') $ together---and subtracting $ 1 $ as it was also part of the original
identity---we obtain the first serious bound for our approximation:
$$ \frac{2}{b^\ell-1}-\frac{b^j}{\bseq[y]_{\ell-1}b^{\ell-1}}+\frac{\bseq[y]_{\ell-1}b^{\ell-1}}{(b^\ell-1)^2}-1
	\leq\frac{\numer{j}{\ell}+1}{\denom{j}{\ell}}-1-\frac{\numer{i}{\ell}}{\denom{i}{\ell}}
	\leq\frac{2}{\bseq[y]_{\ell-1}b^{\ell-1}}-\frac{1}{b^\ell-1}
		+\frac{(b^{\ell+1}-1)b^j}{\bseq[y]_{\ell-1}^2b^{2\ell-2}}-1 $$
which admittedly is a lot to take in when looking at it. Please take a moment to go over the steps thoroughly and convince yourself
of this derivation before we move forward\footnote{If you see a problem in the proof up to this point or even further along,
I would be more than happy to know. Thank you.}.

Admittedly we are only interested in the lower end of this particular bound
$$ \frac{2}{b^\ell-1}-\frac{b^j}{\bseq[y]_{\ell-1}b^{\ell-1}}+\frac{\bseq[y]_{\ell-1}b^{\ell-1}}{(b^\ell-1)^2}-1
	\leq\frac{\numer{j}{\ell}+1}{\denom{j}{\ell}}-1-\frac{\numer{i}{\ell}}{\denom{i}{\ell}} $$
as we have yet to derive the upper bound separately. In fact this is how such inequality manipulations work: maintain the chain
until the final moment where you usually only need the upper or lower bound but not both.  You have to maintain the chain because
you will likely be performing additive and multiplicative inversions, and complete information is needed for such manipulations.

We are now interested in the upper half of our original bound:
$$ \frac{\numer{j}{\ell}}{\denom{j}{\ell}}+1-\frac{\numer{i}{\ell}+1}{\denom{i}{\ell}} $$
for which we have the identity:
$$ \frac{\numer{j}{\ell}}{\denom{j}{\ell}}+1-\frac{\numer{i}{\ell}+1}{\denom{i}{\ell}}
	\quad=\qquad\bradix[w]{j\le k\le\ell}\left(\frac{1}{\denom{j}{\ell}}-\frac{1}{\denom{i}{\ell}}\right)
	+\frac{\numer{j}{\ell}-\numer{i}{\ell}-1}{\denom{i}{\ell}}+1 $$

Actually, for the curious, I'd like to make a {\bfseries side note} here. As it is, one can also derive the alternate identity:
$$ \frac{\numer{j}{\ell}}{\denom{j}{\ell}}+1-\frac{\numer{i}{\ell}+1}{\denom{i}{\ell}}
	\quad=\qquad\frac{\denom{i}{\ell}\numer{j}{\ell}+\denom{i}{\ell}\denom{j}{\ell}-\denom{j}{\ell}[\numer{i}{\ell}+1]}
		{\denom{i}{\ell}\denom{j}{\ell}} $$
which privileges multiplication as the \emph{root} operator rather than addition. I've pointed out in the \emph{inequality.pdf}
article that multiplicative forms can provide improved bounds over additive forms of an identity, the problem here is some of the
bounds derived in the process of creating a chain here don't satisfy the assumptions needed for the rules of inequality manipulation.
Otherwise I would have likely used this identity instead.

Getting back to it: We now focus on the left side similar to the previous derivation:
$$ \bradix[w]{j\le k\le\ell}\left(\frac{1}{\denom{j}{\ell}}-\frac{1}{\denom{i}{\ell}}\right) $$
Luckily for us, from our previous derivations we already know that
$$ \frac{1}{(b^\ell-1)^2}
	\leq\frac{1}{\denom{j}{\ell}}-\frac{1}{\denom{i}{\ell}}
	\leq\frac{b^j}{\bseq[y]_{\ell-1}^2b^{2\ell-2}} $$
and
$$ \bunderseq[y]{\ell-1}b^{\ell-1}\le\bradix[w]{j\le k\le\ell}\le b^{\ell+1}-1 $$
so we can skip the work and get straight to the result:
$$ \frac{\bseq[y]_{\ell-1}b^{\ell-1}}{(b^\ell-1)^2}
	\leq\bradix[w]{j\le k\le\ell}\left(\frac{1}{\denom{j}{\ell}}-\frac{1}{\denom{i}{\ell}}\right)
	\leq\frac{(b^{\ell+1}-1)b^j}{\bseq[y]_{\ell-1}^2b^{2\ell-2}}\qquad (l'') $$

As for our right-hand side:
$$ \frac{\numer{j}{\ell}-\numer{i}{\ell}-1}{\denom{i}{\ell}} $$
it is also similar to the previous derivations. If we replace the index of $ (4) $ we immediately have:
$$ \frac{1}{b^\ell-1}\leq\frac{1}{\denom{i}{\ell}}\leq\frac{1}{\bseq[y]_{\ell-1}b^{\ell-1}}\qquad (8) $$
and with $ (3) $ from even earlier
$$ 1\le\bradix[w]{i\le k \le\ell}+1-\bradix[w]{j\le k\le\ell}\le b^j $$
we can multiply this by $ (8) $ to obtain
$$ \frac{1}{b^\ell-1}\leq\frac{\numer{i}{\ell}+1-\numer{j}{\ell}}{\denom{i}{\ell}}\leq\frac{b^j}{\bseq[y]_{\ell-1}b^{\ell-1}} $$
which when itself is additively inverted results in the second bound of our above identity:
$$ -\frac{b^j}{\bseq[y]_{\ell-1}b^{\ell-1}}
	\leq\frac{\numer{j}{\ell}-\numer{i}{\ell}-1}{\denom{i}{\ell}}
	\leq-\frac{1}{b^\ell-1}\qquad (r'') $$
which when adding $ (l'') $ and $ (r'') $ as well as adding $ 1 $ more, we obtain:
$$ \frac{\bseq[y]_{\ell-1}b^{\ell-1}}{(b^\ell-1)^2}-\frac{b^j}{\bseq[y]_{\ell-1}b^{\ell-1}}+1
	\leq\frac{\numer{j}{\ell}}{\denom{j}{\ell}}+1-\frac{\numer{i}{\ell}+1}{\denom{i}{\ell}}
	\leq\frac{(b^{\ell+1}-1)b^j}{\bseq[y]_{\ell-1}^2b^{2\ell-2}}-\frac{1}{b^\ell-1}+1 $$
where of course we are only really interested in the upper half:
$$ \frac{\numer{j}{\ell}}{\denom{j}{\ell}}+1-\frac{\numer{i}{\ell}+1}{\denom{i}{\ell}}
	\leq\frac{(b^{\ell+1}-1)b^j}{\bseq[y]_{\ell-1}^2b^{2\ell-2}}-\frac{1}{b^\ell-1}+1 $$

We now have enough, and by using the transitive law we have possibly weaker but more independent bounds on our original relation:
$$ \frac{2}{b^\ell-1}-\frac{b^j}{\bseq[y]_{\ell-1}b^{\ell-1}}+\frac{\bseq[y]_{\ell-1}b^{\ell-1}}{(b^\ell-1)^2}-1
	\leq\left\lfloor\frac{\numer{j}{\ell}}{\denom{j}{\ell}}\right\rfloor
		-\left\lfloor\frac{\numer{i}{\ell}}{\denom{i}{\ell}}\right\rfloor
	\leq\frac{(b^{\ell+1}-1)b^j}{\bseq[y]_{\ell-1}^2b^{2\ell-2}}-\frac{1}{b^\ell-1}+1 $$

\subsection*{refinements}

We have a few subtleties to consider actually. To begin, only midway did we add the assumption that
$ \denom{i}{\ell}\le\numer{i}{\ell} $, but looking more closely it's possible in this case the leading digit of $ \bm(w) $ is zero.
In our intuitive example this wouldn't happen because our divisor was $ 99 $, as large as possible under the constraint of two digits.
If it had been smaller, say $ 12 $, then the first two digits of our dividend $ 20 $ would have been large enough that we wouldn't
have needed the extra step of adding the next digit $ 200 $. In this case you can think of our partial dividend not as $ 20 $ but
instead as $ 020 $. Same value, but three digits now. If you go over the steps in this derivation you will discover it does not
contradict our result.

This resolves the first subtlety in that \emph{our derived bound works for the seperate conditional case within our algorithm} as well.
The second subtlety is it is now time to make compromises on our bound. To be fair, the above bound is still incomplete as we intended
to derive a bound where both the lower and upper were entirely independent of the numerator and denominator values in the middle of
this ordered relation. It is not yet independent because we have the $ \bseq[y]_{\ell-1} $ value embedded within, and so it is here
we make our compromise: We will add the following assumption
$$ \left\lfloor\frac{b}{2}\right\rfloor\leq\bunderseq[y]{\ell-1} $$
but if we intend to use this to further our bound derivation, we need this to be in the form of a chain. Fortunately, we already
know firstly $ \bseq[y]_{\ell-1} < b $. Secondly, given our ``floor function'' lemma from before, we then have
$$ \frac{b-1}{2}\leq\bunderseq[y]{\ell-1}\leq b-1 $$
this bound we also need in the form of a multiplicative inverse
$$ \frac{1}{b-1}\leq\frac{1}{\bseq[y]_{\ell-1}}\leq\frac{2}{b-1} $$

We have what we need to begin. Starting with the lower bound:
$$ \frac{2}{b^\ell-1}-\frac{b^j}{\bseq[y]_{\ell-1}b^{\ell-1}}+\frac{\bseq[y]_{\ell-1}b^{\ell-1}}{(b^\ell-1)^2}-1 $$
we divide and consensus:
$$ \frac{1}{b-1}\leq\frac{1}{\bseq[y]_{\ell-1}}\leq\frac{2}{b-1} $$
becomes
$$ \frac{b^j}{(b-1)b^{\ell-1}}\leq\frac{b^j}{\bseq[y]_{\ell-1}b^{\ell-1}}\leq\frac{2b^j}{(b-1)b^{\ell-1}} $$
which in turn becomes
$$ -\frac{2b^j}{(b-1)b^{\ell-1}}\leq-\frac{b^j}{\bseq[y]_{\ell-1}b^{\ell-1}}\leq-\frac{b^j}{(b-1)b^{\ell-1}} $$

Moving on:
$$ \frac{b-1}{2}\leq\bunderseq[y]{\ell-1}\leq b-1 $$
becomes
$$ \frac{(b-1)b^{\ell-1}}{2(b^\ell-1)^2}\leq\frac{\bseq[y]_{\ell-1}b^{\ell-1}}{(b^\ell-1)^2}\leq\frac{(b-1)b^{\ell-1}}{(b^\ell-1)^2} $$
and so adding everything together:
$$ \frac{2}{b^\ell-1}+\frac{(b-1)b^{\ell-1}}{2(b^\ell-1)^2}-\frac{2b^j}{(b-1)b^{\ell-1}}-1
	\leq\frac{2}{b^\ell-1}-\frac{b^j}{\bseq[y]_{\ell-1}b^{\ell-1}}+\frac{\bseq[y]_{\ell-1}b^{\ell-1}}{(b^\ell-1)^2}-1
	\leq\frac{2}{b^\ell-1}+\frac{(b-1)b^{\ell-1}}{(b^\ell-1)^2}-\frac{b^j}{(b-1)b^{\ell-1}}-1 $$

The upper bound is a tiny bit simpler:
$$ \frac{(b^{\ell+1}-1)b^j}{\bseq[y]_{\ell-1}^2b^{2\ell-2}}-\frac{1}{b^\ell-1}+1 $$
first noting
$$ \frac{1}{(b-1)^2}\leq\frac{1}{\bseq[y]_{\ell-1}^2}\leq\frac{4}{(b-1)^2} $$
since
$$ \frac{1}{b-1}\leq\frac{1}{\bseq[y]_{\ell-1}}\leq\frac{2}{b-1} $$
and so
$$ \frac{(b^{\ell+1}-1)b^j}{(b-1)^2b^{2\ell-2}}
	\leq\frac{(b^{\ell+1}-1)b^j}{\bseq[y]_{\ell-1}^2b^{2\ell-2}}
	\leq\frac{4(b^{\ell+1}-1)b^j}{(b-1)^2b^{2\ell-2}} $$
along with
$$ \frac{(b^{\ell+1}-1)b^j}{(b-1)^2b^{2\ell-2}}-\frac{1}{b^\ell-1}+1
	\leq\frac{(b^{\ell+1}-1)b^j}{\bseq[y]_{\ell-1}^2b^{2\ell-2}}-\frac{1}{b^\ell-1}+1
	\leq\frac{4(b^{\ell+1}-1)b^j}{(b-1)^2b^{2\ell-2}}-\frac{1}{b^\ell-1}+1 $$

Finally, we have our refined---if not ugly--bound:
$$ \frac{2}{b^\ell-1}+\frac{(b-1)b^{\ell-1}}{2(b^\ell-1)^2}-\frac{2b^j}{(b-1)b^{\ell-1}}-1
	\leq\left\lfloor\frac{\numer{j}{\ell}}{\denom{j}{\ell}}\right\rfloor
		-\left\lfloor\frac{\numer{i}{\ell}}{\denom{i}{\ell}}\right\rfloor
	\leq\frac{4(b^{\ell+1}-1)b^j}{(b-1)^2b^{2\ell-2}}-\frac{1}{b^\ell-1}+1 $$

for the experienced inequality manipulator, you should also be able to verify this bound can be further simplified as:
$$ -1\ \le\ \left\lfloor\frac{\numer{j}{\ell}}{\denom{j}{\ell}}\right\rfloor
		-\left\lfloor\frac{\numer{i}{\ell}}{\denom{i}{\ell}}\right\rfloor
	\ \le\ 1 $$
keep in mind as we are assuming our divisor has two or more digits we are using the additional assumption that $ j\le\ell+2 $.

In the special case of $ i=0 $ the above simplifies further to our quotient of interest,
$$ -1\ \le\ \left\lfloor\frac{\numer{j}{\ell}}{\denom{j}{\ell}}\right\rfloor-q\ \le\ 1 $$
which of course can be rearranged to put the center of attention on said quotient:
$$ \left\lfloor\frac{\numer{j}{\ell}}{\denom{j}{\ell}}\right\rfloor-1
	\le\ \! q\ \le\ \left\lfloor\frac{\numer{j}{\ell}}{\denom{j}{\ell}}\right\rfloor+1 $$

To reiterate: This bound allows us to narrow down in advance our calculation of the quotient for arbitrary precision digit divisors
(two or more digits) to one of three possible cases. At that point, it's guess and check, but a constant bound is far better than
logarithmic or even linear.

And yet, we can still do just a little bit better.

\subsection*{cleanup}

\newpage

%%%%%%%%%%%%%%%%%%%%%%%%%%%%%%%%%%%%%%%%%%%%%%%%%%%%%%%%%%%%%%%%%%%%%%%%%%%%%%%%%%%%%%%%%%%%%%%%%%%%%%%%%%%%%
in it, I do point out how equality identities don't necessarily preserve inequality bounds. This is to say,
you can reshape an error bound with nothing else than an identity substitution, such as the following:
$$ \begin{array}{rcl}
\frac{\numer{j}{\ell}+1}{\denom{j}{\ell}}-1-\frac{\numer{i}{\ell}}{\denom{i}{\ell}}
 & = & \frac{\numer{j}{\ell}+1}{\denom{j}{\ell}}-\frac{\numer{i}{\ell}}{\denom{i}{\ell}}-1 \\
\\
 & = & \frac{\numer{j}{\ell}+1}{\denom{j}{\ell}}-\frac{\numer{i}{\ell}+\denom{i}{\ell}}{\denom{i}{\ell}} \\
\\
 & = & \frac{\numer{j}{\ell}+1}{\denom{j}{\ell}}
	-\frac{\numer{i}{\ell}+\denom{i}{\ell}}{\denom{j}{\ell}}
	+\frac{\numer{i}{\ell}+\denom{i}{\ell}}{\denom{j}{\ell}}
	-\frac{\numer{i}{\ell}+\denom{i}{\ell}}{\denom{i}{\ell}} \\
\\
 & = & \frac{\numer{j}{\ell}+1-[\numer{i}{\ell}+\denom{i}{\ell}]}{\denom{j}{\ell}}
	+[\bradix[w]{i\le k\le\ell}+\bradix[y]{i\le k < \ell}]\left(\frac{1}{\denom{j}{\ell}}-\frac{1}{\denom{i}{\ell}}\right) \\
\end{array} $$
the advantage of breaking it up like this is being able to primarily focus separately on the numerators:
$$ \qquad\frac{\numer{j}{\ell}+1-[\numer{i}{\ell}+\denom{i}{\ell}]}{\denom{j}{\ell}}\qquad (l) $$
or the denominators
$$ [\bradix[w]{i\le k\le\ell}+\bradix[y]{i\le k < \ell}]\left(\frac{1}{\denom{j}{\ell}}-\frac{1}{\denom{i}{\ell}}\right)\qquad (r) $$

We start with the numerators, with the following straightforward inequality:
$$ 0\le\bradix[w]{i\le k\le\ell}-\bradix[w]{j\le k\le\ell}\le b^j-1 $$
as we need a ``$ \denom{i}{\ell}+1 $'' term in the centre, we shift the whole thing right by such:
$$ \bradix[y]{i\le k < \ell}+1
	\le\bradix[w]{i\le k \le\ell}+\bradix[y]{i\le k < \ell}+1-\bradix[w]{j\le k\le\ell}
	\le b^j+\bradix[y]{i\le k < \ell}\qquad (3) $$
we will also be needing the basic inequality:
$$ \bunderseq[y]{\ell-1}b^{\ell-1}\le\bradix[y]{j\le k < \ell}\le b^\ell-1 $$
which also holds in the slightly different flavor:
$$ \bunderseq[y]{\ell-1}b^{\ell-1}\le\bradix[y]{i\le k < \ell}\le b^\ell-1 $$
which we can shift by one
$$ \bunderseq[y]{\ell-1}b^{\ell-1}+1\le\bradix[y]{i\le k < \ell}+1\le b^\ell $$
which improves the lower bound of $ (3) $:
$$ \bunderseq[y]{\ell-1}b^{\ell-1}+1
	\le\bradix[w]{i\le k \le\ell}+\bradix[y]{i\le k < \ell}+1-\bradix[w]{j\le k\le\ell}
	\le b^j+\bradix[y]{i\le k < \ell}\qquad (3a) $$
or we can shift by more
$$ \bunderseq[y]{\ell-1}b^{\ell-1}+b^j\le\bradix[y]{i\le k < \ell}+b^j\le b^\ell+b^j-1 $$
which improves the upper bound of $ (3a) $:
$$ \bunderseq[y]{\ell-1}b^{\ell-1}+1
	\le\bradix[w]{i\le k \le\ell}+\bradix[y]{i\le k < \ell}+1-\bradix[w]{j\le k\le\ell}
	\le b^\ell+b^j-1\qquad (4) $$
Looking back at
$$ \bunderseq[y]{\ell-1}b^{\ell-1}\le\bradix[y]{j\le k < \ell}\le b^\ell-1 $$
we notice our denominator does not equal zero: $ \denom{j}{\ell}\neq 0 $, and in
particular, we also know by assumption that our leading digit is also non-zero: $ \bseq[y]_{\ell-1}\neq 0 $.
We multiplicatively invert to obtain:
$$ \frac{1}{b^\ell-1}\leq\frac{1}{\denom{j}{\ell}}\leq\frac{1}{\bseq[y]_{\ell-1}b^{\ell-1}}\qquad (5) $$
and for our purposes will will also want to scale
$$ \frac{2}{b^\ell-1}\leq\frac{2}{\denom{j}{\ell}}\leq\frac{2}{\bseq[y]_{\ell-1}b^{\ell-1}} $$
as well as additively invert our bound
$$ \frac{-2}{\bseq[y]_{\ell-1}b^{\ell-1}}\leq\frac{-2}{\denom{j}{\ell}}\leq\frac{-2}{b^\ell-1}\qquad (6) $$

Going back to $ (4) $, we multiply by $ (5) $ to extend our bound as follows:
$$ \frac{\bseq[y]_{\ell-1}b^{\ell-1}+1}{b^\ell-1}
	\leq\frac{\numer{i}{\ell}+\denom{i}{\ell}+1-\numer{j}{\ell}}{\denom{j}{\ell}}
	\leq\frac{b^\ell+b^j-1}{\bseq[y]_{\ell-1}b^{\ell-1}} $$
and finally we add to this our previous derivation $ (6) $ to obtain:
$$ \frac{\bseq[y]_{\ell-1}b^{\ell-1}+1}{b^\ell-1}-\frac{2}{\bseq[y]_{\ell-1}b^{\ell-1}}
	\leq\frac{\numer{i}{\ell}+\denom{i}{\ell}-1-\numer{j}{\ell}}{\denom{j}{\ell}}
	\leq\frac{b^\ell+b^j-1}{\bseq[y]_{\ell-1}b^{\ell-1}}-\frac{2}{b^\ell-1}\qquad $$
which when itself is additively inverted results in the first term of our above identity $ (l) $:
$$ \frac{2}{b^\ell-1}-\frac{b^\ell+b^j-1}{\bseq[y]_{\ell-1}b^{\ell-1}}
	\leq\frac{\numer{j}{\ell}+1-[\numer{i}{\ell}+\denom{i}{\ell}]}{\denom{j}{\ell}}
	\leq\frac{2}{\bseq[y]_{\ell-1}b^{\ell-1}}-\frac{\bseq[y]_{\ell-1}b^{\ell-1}+1}{b^\ell-1}\qquad (l') $$

Changing our focus now toward the right side of our identity:
$$ [\bradix[w]{i\le k\le\ell}+\bradix[y]{i\le k < \ell}]\left(\frac{1}{\denom{j}{\ell}}-\frac{1}{\denom{i}{\ell}}\right) $$
we first simplify fractional part:
$$ \frac{1}{\denom{j}{\ell}}-\frac{1}{\denom{i}{\ell}}=\frac{\denom{i}{j}}{\denom{j}{\ell}\denom{i}{\ell}} $$
Recalling the previous derivation $ (5) $, if we create a copy replacing the index $ j $ with $ i $ and multiply, we obtain:
$$ \frac{1}{(b^\ell-1)^2}\leq\frac{1}{\denom{j}{\ell}\denom{i}{\ell}}\leq\frac{1}{\bseq[y]_{\ell-1}^2b^{2\ell-2}}\qquad (7) $$
Furthermore we can safely assume $ \denom{i}{\ell}\neq\denom{j}{\ell} $ since we intentionally seek an
imperfect approximation---it makes no sense for equality to exist---which leaves us with
$$ 0 < \bradix[y]{i\le k < \ell}-\bradix[y]{j\le k < \ell}
	\quad\Longrightarrow\quad 1\le\bradix[y]{i\le k < j}\le b^j\qquad (8) $$
multiplying $ (7) $ and $ (8) $ together we derive the next bound in our sequence:
$$ \frac{1}{(b^\ell-1)^2}
	\leq\frac{\denom{i}{j}}{\denom{j}{\ell}\denom{i}{\ell}}
	\leq\frac{b^j}{\bseq[y]_{\ell-1}^2b^{2\ell-2}} $$
By assumption we know that $ \denom{i}{\ell}\le\numer{i}{\ell} $ and so we have
$$ \bunderseq[y]{\ell-1}b^{\ell-1}\le\bradix[w]{i\le k\le\ell}\le b^{\ell+1}-1 $$
and again remembering
$$ \bunderseq[y]{\ell-1}b^{\ell-1}\le\bradix[y]{i\le k < \ell}\le b^\ell-1 $$
we can add to get
$$ 2\bunderseq[y]{\ell-1}b^{\ell-1}
	\le\bradix[w]{i\le k\le\ell}+\bradix[y]{i\le k < \ell}
	\le b^{\ell+1}+b^\ell-2 $$
multiplying this and the former together we derive our right bound
$$ \frac{2\bseq[y]_{\ell-1}b^{\ell-1}}{(b^\ell-1)^2}
	\leq[\bradix[w]{i\le k\le\ell}+\bradix[y]{i\le k < \ell}]\left(\frac{1}{\denom{j}{\ell}}-\frac{1}{\denom{i}{\ell}}\right)
	\leq\frac{(b^{\ell+1}+b^\ell-2)b^j}{\bseq[y]_{\ell-1}^2b^{2\ell-2}}\qquad (r') $$
finally adding $ (l') $ and $ (r') $ together we obtain the first serious bound for our approximation:
$$ \frac{2}{b^\ell-1}-\frac{b^\ell+b^j-1}{\bseq[y]_{\ell-1}b^{\ell-1}}+\frac{2\bseq[y]_{\ell-1}b^{\ell-1}}{(b^\ell-1)^2}
	\leq\frac{\numer{j}{\ell}+1}{\denom{j}{\ell}}-1-\frac{\numer{i}{\ell}}{\denom{i}{\ell}} $$
$$ \leq\frac{2}{\bseq[y]_{\ell-1}b^{\ell-1}}-\frac{\bseq[y]_{\ell-1}b^{\ell-1}+1}{b^\ell-1}
	+\frac{(b^{\ell+1}+b^\ell-2)b^j}{\bseq[y]_{\ell-1}^2b^{2\ell-2}} $$
which admittedly is a lot to take in when looking at it. Please take a moment to go over the steps thoroughly and convince yourself
of this derivation before we move forward\footnote{If you see a problem in the proof up to this point or even further along,
I would be more than happy to know. Thank you.}.

Admittedly we are only interested in the lower end of this particular bound
$$ \frac{2}{b^\ell-1}-\frac{b^\ell+b^j-1}{\bseq[y]_{\ell-1}b^{\ell-1}}+\frac{2\bseq[y]_{\ell-1}b^{\ell-1}}{(b^\ell-1)^2}
	\leq\frac{\numer{j}{\ell}+1}{\denom{j}{\ell}}-1-\frac{\numer{i}{\ell}}{\denom{i}{\ell}} $$
as we have yet to derive the upper bound separately. In fact this is how such inequality manipulations work:
maintain the chain until the final moment where you usually only need the upper or lower bound but not both
before moving onto a next step.  You have to maintain the chain because you will likely be performing additive
and multiplicative inversions, and complete information is needed for such manipulations.

We are now interested in the upper half of our original bound:
$$ \frac{\numer{j}{\ell}}{\denom{j}{\ell}}+1-\frac{\numer{i}{\ell}+1}{\denom{i}{\ell}} $$
for which we have the identity:
$$ \frac{\numer{j}{\ell}}{\denom{j}{\ell}}+1-\frac{\numer{i}{\ell}+1}{\denom{i}{\ell}}
	\quad=\qquad\bradix[w]{j\le k\le\ell}\left(\frac{1}{\denom{j}{\ell}}-\frac{1}{\denom{i}{\ell}}\right)
	+\frac{\numer{j}{\ell}-\numer{i}{\ell}-1}{\denom{i}{\ell}}+1 $$

Actually, for the curious, I'd like to make a {\bfseries side note} here. As it is, one can also derive the alternate identity:
$$ \frac{\numer{j}{\ell}}{\denom{j}{\ell}}+1-\frac{\numer{i}{\ell}+1}{\denom{i}{\ell}}
	\quad=\qquad\frac{\denom{i}{\ell}\numer{j}{\ell}+\denom{i}{\ell}\denom{j}{\ell}-\denom{j}{\ell}[\numer{i}{\ell}+1]}
		{\denom{i}{\ell}\denom{j}{\ell}} $$
which privileges multiplication as the \emph{root} operator rather than addition. I've pointed out in the \emph{inequality.pdf}
article that multiplicative forms can provide improved bounds over additive forms of an identity, the problem here is some of the
bounds derived in the process of creating a chain here don't satisfy the assumptions needed for the rules of inequality manipulation.
Otherwise I would have likely used this identity instead.

Getting back to it: We now focus on the left side similar to the previous derivation:
$$ \bradix[w]{j\le k\le\ell}\left(\frac{1}{\denom{j}{\ell}}-\frac{1}{\denom{i}{\ell}}\right) $$
Luckily for us, from our previous derivations we already know that
$$ \frac{1}{(b^\ell-1)^2}
	\leq\frac{1}{\denom{j}{\ell}}-\frac{1}{\denom{i}{\ell}}
	\leq\frac{b^j}{\bseq[y]_{\ell-1}^2b^{2\ell-2}} $$
and
$$ \bunderseq[y]{\ell-1}b^{\ell-1}\le\bradix[w]{j\le k\le\ell}\le b^{\ell+1}-1 $$
so we can skip the work and get straight to the result:
$$ \frac{\bseq[y]_{\ell-1}b^{\ell-1}}{(b^\ell-1)^2}
	\leq\bradix[w]{j\le k\le\ell}\left(\frac{1}{\denom{j}{\ell}}-\frac{1}{\denom{i}{\ell}}\right)
	\leq\frac{(b^{\ell+1}-1)b^j}{\bseq[y]_{\ell-1}^2b^{2\ell-2}}\qquad (l'') $$

As for our right-hand side:
$$ \frac{\numer{j}{\ell}-\numer{i}{\ell}-1}{\denom{i}{\ell}} $$
it is also similar to the previous derivations. If we replace the index of $ (4) $ we immediately have:
$$ \frac{1}{b^\ell-1}\leq\frac{1}{\denom{i}{\ell}}\leq\frac{1}{\bseq[y]_{\ell-1}b^{\ell-1}}\qquad (8) $$
and with $ (3) $ from even earlier
$$ 1\le\bradix[w]{i\le k \le\ell}+1-\bradix[w]{j\le k\le\ell}\le b^j $$
we can multiply this by $ (8) $ to obtain
$$ \frac{1}{b^\ell-1}\leq\frac{\numer{i}{\ell}+1-\numer{j}{\ell}}{\denom{i}{\ell}}\leq\frac{b^j}{\bseq[y]_{\ell-1}b^{\ell-1}} $$
which when itself is additively inverted results in the second bound of our above identity:
$$ -\frac{b^j}{\bseq[y]_{\ell-1}b^{\ell-1}}
	\leq\frac{\numer{j}{\ell}-\numer{i}{\ell}-1}{\denom{i}{\ell}}
	\leq-\frac{1}{b^\ell-1}\qquad (r'') $$
which when adding $ (l'') $ and $ (r'') $ as well as adding $ 1 $ more, we obtain:
$$ \frac{\bseq[y]_{\ell-1}b^{\ell-1}}{(b^\ell-1)^2}-\frac{b^j}{\bseq[y]_{\ell-1}b^{\ell-1}}+1
	\leq\frac{\numer{j}{\ell}}{\denom{j}{\ell}}+1-\frac{\numer{i}{\ell}+1}{\denom{i}{\ell}}
	\leq\frac{(b^{\ell+1}-1)b^j}{\bseq[y]_{\ell-1}^2b^{2\ell-2}}-\frac{1}{b^\ell-1}+1 $$
where of course we are only really interested in the upper half:
$$ \frac{\numer{j}{\ell}}{\denom{j}{\ell}}+1-\frac{\numer{i}{\ell}+1}{\denom{i}{\ell}}
	\leq\frac{(b^{\ell+1}-1)b^j}{\bseq[y]_{\ell-1}^2b^{2\ell-2}}-\frac{1}{b^\ell-1}+1 $$

We now have enough, and by using the transitive law we have possibly weaker but more independent bounds on our original relation:
$$ \frac{2}{b^\ell-1}-\frac{b^\ell+b^j-1}{\bseq[y]_{\ell-1}b^{\ell-1}}+\frac{2\bseq[y]_{\ell-1}b^{\ell-1}}{(b^\ell-1)^2}
	\leq\left\lfloor\frac{\numer{j}{\ell}}{\denom{j}{\ell}}\right\rfloor
		-\left\lfloor\frac{\numer{i}{\ell}}{\denom{i}{\ell}}\right\rfloor
	\leq\frac{(b^{\ell+1}-1)b^j}{\bseq[y]_{\ell-1}^2b^{2\ell-2}}-\frac{1}{b^\ell-1}+1 $$

\subsection*{refinements}

We have a few subtleties to consider actually. To begin, only midway did we add the assumption that
$ \denom{i}{\ell}\le\numer{i}{\ell} $, but looking more closely it's possible in this case the leading digit of $ \bm(w) $ is zero.
In our intuitive example this wouldn't happen because our divisor was $ 99 $, as large as possible under the constraint of two digits.
If it had been smaller, say $ 12 $, then the first two digits of our dividend $ 20 $ would have been large enough that we wouldn't
have needed the extra step of adding the next digit $ 200 $. In this case you can think of our partial dividend not as $ 20 $ but
instead as $ 020 $. Same value, but three digits now. If you go over the steps in this derivation you will discover it does not
contradict our result.

This resolves the first subtlety in that \emph{our derived bound works for the seperate conditional case within our algorithm} as well.
The second subtlety is it is now time to make compromises on our bound. To be fair, the above bound is still incomplete as we intended
to derive a bound where both the lower and upper were entirely independent of the numerator and denominator values in the middle of
this ordered relation. It is not yet independent because we have the $ \bseq[y]_{\ell-1} $ value embedded within, and so it is here
we make our compromise: We will add the following assumption
$$ \left\lfloor\frac{b}{2}\right\rfloor\leq\bunderseq[y]{\ell-1} $$
but if we intend to use this to further our bound derivation, we need this to be in the form of a chain. Fortunately, we already
know firstly $ \bseq[y]_{\ell-1} < b $. Secondly, given our ``floor function'' lemma from before, we then have
$$ \frac{b-1}{2}\leq\bunderseq[y]{\ell-1}\leq b-1 $$
this bound we also need in the form of a multiplicative inverse
$$ \frac{1}{b-1}\leq\frac{1}{\bseq[y]_{\ell-1}}\leq\frac{2}{b-1} $$

We have what we need to begin. Starting with the lower bound:
$$ \frac{2}{b^\ell-1}-\frac{b^\ell+b^j-1}{\bseq[y]_{\ell-1}b^{\ell-1}}+\frac{2\bseq[y]_{\ell-1}b^{\ell-1}}{(b^\ell-1)^2} $$
we divide and consensus:
$$ \frac{1}{b-1}\leq\frac{1}{\bseq[y]_{\ell-1}}\leq\frac{2}{b-1} $$
becomes
$$ \frac{b^\ell+b^j-1}{(b-1)b^{\ell-1}}
	\leq\frac{b^\ell+b^j-1}{\bseq[y]_{\ell-1}b^{\ell-1}}
	\leq\frac{2(b^\ell+b^j-1)}{(b-1)b^{\ell-1}} $$
which in turn becomes
$$ -\frac{2(b^\ell+b^j-1)}{(b-1)b^{\ell-1}}
	\leq-\frac{b^\ell+b^j-1}{\bseq[y]_{\ell-1}b^{\ell-1}}
	\leq-\frac{b^\ell+b^j-1}{(b-1)b^{\ell-1}} $$

Moving on:
$$ \frac{b-1}{2}\leq\bunderseq[y]{\ell-1}\leq b-1 $$
becomes
$$ \frac{(b-1)b^{\ell-1}}{(b^\ell-1)^2}
	\leq\frac{2\bseq[y]_{\ell-1}b^{\ell-1}}{(b^\ell-1)^2}
	\leq\frac{2(b-1)b^{\ell-1}}{(b^\ell-1)^2} $$
and so adding everything together:
$$ \frac{2}{b^\ell-1}-\frac{2(b^\ell+b^j-1)}{(b-1)b^{\ell-1}}+\frac{(b-1)b^{\ell-1}}{(b^\ell-1)^2}
	\leq\frac{2}{b^\ell-1}-\frac{b^\ell+b^j-1}{\bseq[y]_{\ell-1}b^{\ell-1}}+\frac{2\bseq[y]_{\ell-1}b^{\ell-1}}{(b^\ell-1)^2}
	\leq\frac{2}{b^\ell-1}-\frac{b^\ell+b^j-1}{(b-1)b^{\ell-1}}+\frac{2(b-1)b^{\ell-1}}{(b^\ell-1)^2} $$

The upper bound is a tiny bit simpler:
$$ \frac{(b^{\ell+1}-1)b^j}{\bseq[y]_{\ell-1}^2b^{2\ell-2}}-\frac{1}{b^\ell-1}+1 $$
first noting
$$ \frac{1}{(b-1)^2}\leq\frac{1}{\bseq[y]_{\ell-1}^2}\leq\frac{4}{(b-1)^2} $$
since
$$ \frac{1}{b-1}\leq\frac{1}{\bseq[y]_{\ell-1}}\leq\frac{2}{b-1} $$
and so
$$ \frac{(b^{\ell+1}-1)b^j}{(b-1)^2b^{2\ell-2}}
	\leq\frac{(b^{\ell+1}-1)b^j}{\bseq[y]_{\ell-1}^2b^{2\ell-2}}
	\leq\frac{4(b^{\ell+1}-1)b^j}{(b-1)^2b^{2\ell-2}} $$
along with
$$ \frac{(b^{\ell+1}-1)b^j}{(b-1)^2b^{2\ell-2}}-\frac{1}{b^\ell-1}+1
	\leq\frac{(b^{\ell+1}-1)b^j}{\bseq[y]_{\ell-1}^2b^{2\ell-2}}-\frac{1}{b^\ell-1}+1
	\leq\frac{4(b^{\ell+1}-1)b^j}{(b-1)^2b^{2\ell-2}}-\frac{1}{b^\ell-1}+1 $$

Finally, we have our refined---if not ugly--bound:
$$ \frac{2}{b^\ell-1}-\frac{2b^\ell+b^j-1}{(b-1)b^{\ell-1}}+\frac{(b-1)b^{\ell-1}}{(b^\ell-1)^2}
	\leq\left\lfloor\frac{\numer{j}{\ell}}{\denom{j}{\ell}}\right\rfloor
		-\left\lfloor\frac{\numer{i}{\ell}}{\denom{i}{\ell}}\right\rfloor
	\leq\frac{4(b^{\ell+1}-1)b^j}{(b-1)^2b^{2\ell-2}}-\frac{1}{b^\ell-1}+1 $$

for the experienced inequality manipulator, you should also be able to verify this bound can be further simplified as:
$$ -1\ \le\ \left\lfloor\frac{\numer{j}{\ell}}{\denom{j}{\ell}}\right\rfloor
		-\left\lfloor\frac{\numer{i}{\ell}}{\denom{i}{\ell}}\right\rfloor
	\ \le\ 1 $$
keep in mind as we are assuming our divisor has two or more digits we are using the additional assumption that $ j\le\ell+2 $.

In the special case of $ i=0 $ the above simplifies further to our quotient of interest,
$$ -1\ \le\ \left\lfloor\frac{\numer{j}{\ell}}{\denom{j}{\ell}}\right\rfloor-q\ \le\ 1 $$
which of course can be rearranged to put the center of attention on said quotient:
$$ \left\lfloor\frac{\numer{j}{\ell}}{\denom{j}{\ell}}\right\rfloor-1
	\le\ \! q\ \le\ \left\lfloor\frac{\numer{j}{\ell}}{\denom{j}{\ell}}\right\rfloor+1 $$

To reiterate: This bound allows us to narrow down in advance our calculation of the quotient for arbitrary precision digit divisors
(two or more digits) to one of three possible cases. At that point, it's guess and check, but a constant bound is far better than
logarithmic or even linear.

And yet, we can still do just a little bit better.

\subsection*{cleanup}

\end{document}

