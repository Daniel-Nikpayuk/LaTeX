% Copyright 2018 Daniel Nikpayuk
\documentclass[twoside]{article}
\usepackage[letterpaper,left=1cm,right=1cm,top=1.5cm,bottom=1.5cm]{geometry}
\usepackage{amsmath}
\usepackage{graphicx}
\usepackage{hyperref}

\newcommand{\then}{\ensuremath{\quad\Longrightarrow\quad}}
\newcommand{\equals}{\ensuremath{\quad =\quad}}

\newcommand{\fourqquad}{\ensuremath{\qquad\qquad\qquad\qquad}}
\newcommand{\alphahat}{\ensuremath{\hat{\alpha}}}

\pagestyle{empty}
\begin{document}

\begin{figure}[h]
\centering
\includegraphics[width=1in]{../../../cc-by-nc.png}\\[0.1in]
\tiny This article is licensed under \\
\href{http://creativecommons.org/licenses/by-nc/4.0/}
{Creative Commons Attribution-NonCommercial 4.0 International.}\\[0.3in]
\end{figure}

Recently, I was looking to solve the combinatorial problem of \emph{the average number of \,$ 0 $s in a binary string of length
$ n $ containing no substring \,$ 00 $} (two or more zeros in a row). We can encode this language into the generating function:
$$ P(z,u)\quad=\quad\frac{1+uz}{1-z(1+uz)} $$
and solve the \emph{mean} as the \emph{moment in} $ u $ we're looking for. This sort of thing ends up being a lot of algebraic
work by hand and is best solved by the symbolic packages of a computer, but otherwise what lessons can be learned if we had to
do it ourselves?

The object of this article is to apply the idea of \emph{interface theory}, where we start with a space, take an inventory
of its subspaces, and then for any given subspace create a compressed representation which maintains its own navigational algebra.
This is an \emph{interface layer}. Solving the moments of the above generating function offers a quality example of this process,
showing of the merits of interface design toward general problem solving.

To begin, when we're looking to solve for the individual terms of a rational generating function such as the above, it helps to decompose
it into its roots:

\begin{align*}
P(z,u)	& \equals \frac{1+uz}{1-z(1+uz)}						\\
											\\
	& \equals \frac{1+uz}{1-z-uz^2}							\\
											\\
	& \equals \frac{-1}{u} \cdot \frac{uz+1}{z^2+\frac{1}{u}z-\frac{1}{u}}		\\
											\\
	& \equals \frac{-1}{u} \cdot \frac{uz+1}{[z-\alpha(u)][z-\alphahat(u)]}		\\
\end{align*}
where
$$ \alpha(u) \equals \frac{-1+\sqrt{4u+1}}{2u} \fourqquad \alphahat(u) \equals \frac{-1-\sqrt{4u+1}}{2u} $$

The value of such a decomposition is that we can further separate our derived expression into its \emph{partial fraction}
representation which offers a simpler means to finding the terms of the generating function. The problem is that we're now
working with square roots in our $ \alpha, \alphahat $, which can get pretty ugly when manipulating our expression into
its partial fraction equivalent, and is likely prone to error given all the details we need to keep track of.

This is where interface theory comes in. We are now working with a grammar which includes square roots, but we'd prefer
to abstract a way from this. The idea then is to come up with an interface, a compressed representation which has its own
homomorphic algebra.~\footnote{When I say ``homomorphic algebra'' I'm using it in a casual intuitive sense, not a formal one.}

We already have our compressions: $ \alpha, \alphahat $, so we only need derive algebraic rules. The main thing with square
roots is we would prefer to always have them in the numerators of our rational functions, and this will inform our algebra
as well. Regardless, it is natural to run through the arithmetic operators of \emph{addition} and \emph{multiplication} first:
$$ \alpha+\alphahat \equals \frac{-1}{u} \fourqquad \alpha\alphahat \equals \frac{-1}{u} $$
these derive quite readily from the definitions.

Next, it we derive the \emph{additive} and \emph{multiplicative} inverses:
\begin{align*}
-\alpha			& \equals \frac{1}{u}+\alphahat		& \frac{1}{\alpha}	& \equals -u\alphahat	\\
														\\
-\alphahat		& \equals \frac{1}{u}+\alpha		& \frac{1}{\alphahat}	& \equals -u\alpha	\\
\end{align*}

With these, we have the basis needed to derive pretty much any other form we'd be interested in, for example:
\begin{align*}
\alpha-\alphahat		& \equals \alpha+\left(\frac{1}{u}+\alpha\right)		\\
												\\
				& \equals \frac{1}{u}+2\alpha					\\
												\\
				& \quad\mbox{or}						\\
												\\
\alpha^2			& \equals (-\alpha)(-\alpha)					\\
												\\
				& \equals -\alpha\left(\frac{1}{u}+\alphahat\right)		\\
												\\
				& \equals -\alpha\frac{1}{u}-\alpha\alphahat			\\
												\\
				& \equals \frac{1}{u}-\frac{1}{u}\alpha				\\
												\\
				& \quad\mbox{or}						\\
												\\
\frac{\alpha}{\alphahat}	& \equals -u\alpha^2						\\
												\\
				& \equals -u\left(\frac{1}{u}-\frac{1}{u}\alpha\right)		\\
												\\
				& \equals -1+\alpha						\\
\end{align*}

Finally, there is one additional trick worth knowing:
\begin{align*}
(c+\alpha)(c+\alphahat)	& \equals c^2+c(\alpha+\alphahat)+\alpha\alphahat		\\
											\\
			& \equals c^2+c\frac{-1}{u}+\frac{-1}{u}			\\
											\\
			& \equals \frac{c^2\,u-(c+1)}{u}				\\
\end{align*}
which is useful for when you want to move the irrational functions $ \alpha, \alphahat $ from a denominator to its numerator:
\begin{align*}
\frac{1}{c+\alpha}	& \equals \frac{1}{c+\alpha} \cdot \frac{c+\alphahat}{c+\alphahat}		\\
													\\
			& \equals (c+\alphahat)\,\frac{u}{c^2\,u-(c+1)}					\\
													\\
			& \equals \frac{cu}{c^2\,u-(c+1)}+\frac{u}{c^2\,u-(c+1)}\,\alphahat		\\
\end{align*}
which you might say is pretty complicated, but it's the closest form to a rational function we can achieve---it involves
only a single square root in the numerator---and rational functions are otherwise which are well known and well behaved.

Finally, for our purposes, there's one more which will be of use:
\begin{align*}
(\alpha-\alphahat)^2	& \equals \left(\frac{1}{u}+2\alpha\right)^2							\\
															\\
			& \equals \frac{1}{u^2}+\frac{4}{u}\alpha+4\alpha^2						\\
															\\
			& \equals \frac{1}{u^2}+\frac{4}{u}\alpha+4\left(\frac{1}{u}-\frac{1}{u}\alpha\right)		\\
															\\
			& \equals \frac{1}{u^2}+\frac{4}{u}\alpha+\frac{4}{u}-\frac{4}{u}\alpha				\\
															\\
			& \equals \frac{4u+1}{u^2}									\\
\end{align*}

Let's summarize all of these secondary identities as follows:
\begin{align}
\alpha-\alphahat			& \equals \frac{1}{u}+2\alpha							\\
\alpha^2				& \equals \frac{1}{u}-\frac{1}{u}\alpha						\\
\frac{\alpha}{\alphahat}		& \equals -1+\alpha								\\
\frac{1}{c+\alpha}			& \equals \frac{cu}{c^2\,u-(c+1)}+\frac{u}{c^2\,u-(c+1)}\,\alphahat		\\
\frac{1}{(\alpha-\alphahat)^2}		& \equals \frac{u^2}{4u+1}							\\
\end{align}

\begin{align*}
P(z,u)	& \equals \frac{-1}{u} \cdot \frac{uz+1}{(z-\alpha)(z-\alphahat)}		\\
\end{align*}

\begin{align*}
\frac{uz+1}{(z-\alpha)(z-\alphahat)}	& \equals \frac{A}{z-\alpha}+\frac{B}{z-\alphahat}				\\
															\\
					& \equals \frac{A(z-\alphahat)+B(z-\alpha)}{(z-\alpha)(z-\alphahat)}		\\
															\\
					& \equals \frac{(A+B)z-(A\alphahat+B\alpha)}{(z-\alpha)(z-\alphahat)}		\\
\end{align*}

\begin{align*}
A+B			& \equals u		\\
						\\
A\alphahat+B\alpha	& \equals -1		\\
\end{align*}

\begin{align*}
A\alphahat+B\alphahat	& \equals u\alphahat	\\
						\\
A\alphahat+B\alpha	& \equals -1		\\
\end{align*}

\begin{align*}
B(\alpha-\alphahat)	& \equals -1-u\alphahat											\\
																\\
B			& \equals \frac{-1-u\alphahat}{\alpha-\alphahat}							\\
																\\
			& \equals \frac{-1-u\alphahat}{\alpha-\alphahat} \cdot \frac{\alpha-\alphahat}{\alpha-\alphahat}	\\
																\\
			& \equals (-1-u\alphahat)(\frac{1}{u}+2\alpha)\,\frac{u^2}{4u+1}					\\
																\\
			& \equals (\frac{-1}{u}-2\alpha-\alphahat-2u\alpha\alphahat)\,\frac{u^2}{4u+1}				\\
																\\
			& \equals (2-\alpha)\,\frac{u^2}{4u+1}									\\
																\\
			& \equals \frac{2u^2}{4u+1}-\frac{u^2}{4u+1}\,\alpha							\\
\end{align*}

\begin{align*}
A	& \equals u-B												\\
														\\
	& \equals u-\frac{2u^2}{4u+1}+\frac{u^2}{4u+1}\,\left(\frac{-1}{u}-\alphahat\right)			\\
														\\
	& \equals \frac{2u^2+u}{4u+1}-\frac{u}{4u+1}-\frac{u^2}{4u+1}\,\alphahat				\\
														\\
	& \equals \frac{2u^2}{4u+1}-\frac{u^2}{4u+1}\,\alphahat							\\
\end{align*}

\begin{align*}
P(z,u)	& \equals \frac{-1}{u}\left(\frac{2u^2}{4u+1}-\frac{u^2}{4u+1}\,\alphahat\right)\frac{1}{z-\alpha}
                  +\frac{-1}{u}\left(\frac{2u^2}{4u+1}-\frac{u^2}{4u+1}\,\alpha\right)\frac{1}{z-\alphahat}				\\
																	\\
	& \equals \frac{-1}{u}\left(\frac{2u^2}{4u+1}-\frac{u^2}{4u+1}\,\alphahat\right)\frac{1}{\alpha}
                  \cdot \frac{1}{1-\frac{z}{\alpha}}
                  +\frac{-1}{u}\left(\frac{2u^2}{4u+1}-\frac{u^2}{4u+1}\,\alpha\right)\frac{1}{\alphahat}
                  \cdot \frac{1}{1-\frac{z}{\alphahat}}											\\
																	\\
	& \equals \left(\frac{2u^2}{4u+1}\,\alpha-\frac{u^2}{4u+1}\,\alpha^2\right) \frac{1}{1-(-u\alpha)z}
                  +\left(\frac{2u^2}{4u+1}\,\alphahat-\frac{u^2}{4u+1}\,\alphahat^2\right) \frac{1}{1-(-u\alphahat)z}		\\
\end{align*}

\begin{align*}
\frac{u^2}{4u+1}\,\alpha^2				& \equals \frac{u^2}{4u+1}\left(\frac{1}{u}-\frac{1}{u}\alpha\right)		\\
																	\\
							& \equals \frac{u}{4u+1}-\frac{u}{4u+1}\,\alpha					\\
																	\\
\frac{2u^2}{4u+1}\,\alpha-\frac{u^2}{4u+1}\,\alpha^2	& \equals \frac{2u^2}{4u+1}\,\alpha-\frac{u}{4u+1}+\frac{u}{4u+1}\,\alpha	\\
																	\\
							& \equals -\frac{u}{4u+1}+\frac{2u^2+u}{4u+1}\,\alpha				\\
\end{align*}

\begin{align*}
P(z,u)	& \equals \left(-\frac{u}{4u+1}+\frac{2u^2+u}{4u+1}\,\alpha\right) \frac{1}{1-(-u\alpha)z}
                  +\left(-\frac{u}{4u+1}+\frac{2u^2+u}{4u+1}\,\alphahat\right) \frac{1}{1-(-u\alphahat)z}		\\
															\\
[z^n]P(z,u)	& \equals \left(-\frac{u}{4u+1}+\frac{2u^2+u}{4u+1}\,\alpha\right)(-u\alpha)^n
                  +\left(-\frac{u}{4u+1}+\frac{2u^2+u}{4u+1}\,\alphahat\right)(-u\alphahat)^n				\\
\end{align*}

\begin{align*}
P(z,u)	& \equals \frac{1+uz}{1-z(1+uz)}										\\
															\\
	& \equals \left(-\frac{u}{4u+1}+\frac{2u^2+u}{4u+1}\,\alpha\right) \frac{1}{1-(-u\alpha)z}
                  +\left(-\frac{u}{4u+1}+\frac{2u^2+u}{4u+1}\,\alphahat\right) \frac{1}{1-(-u\alphahat)z}		\\
\end{align*}
where
$$ \alpha(u) \equals \frac{-1+\sqrt{4u+1}}{2u} \fourqquad \alphahat(u) \equals \frac{-1-\sqrt{4u+1}}{2u} $$

\end{document}

