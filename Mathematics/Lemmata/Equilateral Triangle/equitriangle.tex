% Copyright 2015 Daniel Nikpayuk
\documentclass[twoside]{article}
\usepackage[letterpaper,left=1cm,right=1cm,top=2cm,bottom=2cm]{geometry}
\usepackage{asymptote}
\usepackage{amsfonts}
\usepackage{graphicx}
\usepackage{hyperref}

\newtheorem{theorem}{Theorem}[section]

\pagestyle{empty}
\begin{document}

\begin{figure}[h]
\centering
\includegraphics[width=1in]{cc-by-nc.png}\\[0.1in]
\tiny This article is licensed under \\
\href{http://creativecommons.org/licenses/by-nc/4.0/}
{Creative Commons Attribution-NonCommercial 4.0 International.}\\[0.3in]
\end{figure}

\begin{asydef}
import graph;
import markers;
unitsize(10cm);

real s3=sqrt(3);
pair o=(0,0);
pair i=(1,0);

pair _z(real cosphi, real sinphi, real scaler0, real scaler1, real numerator, real denominator)
{
	return numerator*(scaler0*cosphi-scaler1*sinphi, scaler1*cosphi+scaler0*sinphi)/denominator;
}

pair z0(real phi, real theta)
{
	real tantheta=Tan(theta);

	return _z(Cos(phi), Sin(phi), 1, tantheta, 1, 1+s3*tantheta);
}

pair z1(real phi, real theta)
{
	real tantheta=Tan(theta);

	return _z(Cos(phi), Sin(phi), s3, 1, tantheta, 1+s3*tantheta);
}

pair z2(real phi, real theta)
{
	real tantheta=Tan(theta);
	real scaler0=1+s3*tantheta;
	real scaler1=s3-tantheta;

	return _z(Cos(phi), Sin(phi), scaler0, scaler1, 1, 2*scaler0);
}

real phi=15;

pair e0=z0(phi,0);
pair e1=z1(phi,0);
pair e2=z2(phi,0);

pair m0=z0(phi,60);
pair m1=z1(phi,60);
pair m2=z2(phi,60);

real theta=10;

pair t0=z0(phi, theta);
pair t1=z1(phi, theta);
pair t2=z2(phi, theta);

\end{asydef}

\noindent\hspace*{0cm}\begin{asy}
//draw:

draw(i--o--e0--e2--o);
draw(e0--m0, dotted);
draw(e1--m1, dotted);
draw(e2--m2, dotted);

draw(o--t0);
dot(t0);
label("$z_0$", t0, S);
dot(t1);
label("$z_1$", t1, SE);
dot(t2);
label("$z_2$", t2, E);

markangle("$\phi$", i, o, e0);
markangle("$\theta$", radius=50, e0, o, t0);

\end{asy}

\vspace{1cm}
\begin{theorem} Given $ \phi\in\mathbb{R} $ and $ 0\le\theta\le\frac{\pi}{3} $ in the equilateral triangle above, we have:
\end{theorem}

\begin{eqnarray*}
z_0 & = & \frac{1}{1+\sqrt{3}\tan\theta} (\cos\phi-\tan\theta\sin\phi,\ \tan\theta\cos\phi+\sin\phi) \\
z_1 & = & \frac{\tan\theta}{1+\sqrt{3}\tan\theta} (\sqrt{3}\cos\phi-\sin\phi,\ \cos\phi+\sqrt{3}\sin\phi) \\
z_2 & = & \frac{1}{2}\cdot\frac{1}{1+\sqrt{3}\tan\theta} ([1+\sqrt{3}\tan\theta]\cos\phi-[\sqrt{3}-\tan\theta]\sin\phi,
	  \ [\sqrt{3}-\tan\theta]\cos\phi+[1+\sqrt{3}\tan\theta]\sin\phi)
\end{eqnarray*}

\end{document}

%for (int k=0; k < 5; ++k)
%	label( "("+string(point(p, k).x)+", "+string(point(p, k).y)+")" , (0, -k*cm), fontsize(24pt));

%pair p=sin( (3, 2) );
%string x=string(p.x);
%string y=string(p.y);
%label( "("+x+", "+y+")" , fontsize(24pt));

