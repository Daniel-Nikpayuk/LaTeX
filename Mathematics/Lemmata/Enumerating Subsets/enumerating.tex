% Copyright 2015 Daniel Nikpayuk
\documentclass[twoside]{article}
\usepackage[letterpaper,left=1cm,right=1cm,top=2cm,bottom=2cm]{geometry}
\usepackage{amsfonts}
\usepackage{graphicx}
\usepackage{hyperref}

\newtheorem{theorem}{Theorem}[section]

\pagestyle{empty}
\begin{document}

\begin{figure}[h]
\centering
\includegraphics[width=1in]{../../../cc-by-nc.png}\\[0.1in]
\tiny This article is licensed under \\
\href{http://creativecommons.org/licenses/by-nc/4.0/}
{Creative Commons Attribution-NonCommercial 4.0 International.}\\[0.3in]
\end{figure}

Here's the deal, let's say you have a finite set: $ \{0, 1, 2, 3, 4, 5, 6\} $\\[0.1cm]

And you want to enumerate every subset of a particular size, say $ 2 $:

$$ \begin{array}{ccc}
\{0, 1\} & \{1, 3\} & \{2, 6\} \\
\{0, 2\} & \{1, 4\} & \{3, 4\} \\
\{0, 3\} & \{1, 5\} & \{3, 5\} \\
\{0, 4\} & \{1, 6\} & \{3, 6\} \\
\{0, 5\} & \{2, 3\} & \{4, 5\} \\
\{0, 6\} & \{2, 4\} & \{4, 6\} \\
\{1, 2\} & \{2, 5\} & \{5, 6\}
\end{array} $$

Simple combinatorics could have told us how many such subsets there are:
$$ {7\choose 2}=21 $$
It's even quite easy to find a bijection from the above \emph{enumerates} onto the set of numbers between $ 0-20 $:
$$ \{0, 1, 2,\ldots , 19, 20\} $$
since it's already listed out, you simply assign to each a natural number:
$$ \begin{array}{lll}
\{0, 1\} \to 0 & \{1, 3\} \to 7 & \{2, 6\} \to 14 \\
\{0, 2\} \to 1 & \{1, 4\} \to 8 & \{3, 4\} \to 15 \\
\{0, 3\} \to 2 & \{1, 5\} \to 9 & \{3, 5\} \to 16 \\
\{0, 4\} \to 3 & \{1, 6\} \to 10 & \{3, 6\} \to 17 \\
\{0, 5\} \to 4 & \{2, 3\} \to 11 & \{4, 5\} \to 18 \\
\{0, 6\} \to 5 & \{2, 4\} \to 12 & \{4, 6\} \to 19 \\
\{1, 2\} \to 6 & \{2, 5\} \to 13 & \{5, 6\} \to 20
\end{array} $$

But a slightly less obvious, lesser known result is to find a---to put it in more computing science terms---\emph{random access}
approach, which is to say a bijective map as function which acts on its input alone: A map which is not recursive with respect
to any of the other subsets as arguments.  The intention of this article is to demonstrate how to do such a convoluted thing:\\[0.1cm]

We'll be working with a case example as the mapping itself is still a bit more algorithmic than it is formulaic.
Take $ \{3, 5\} $ as our case study.

First we need to standardize our terminology. We're working with an $ n $-set, in this case $ n=7 $, and we are enumerating its
$ m $-subsets, in this case $ m=2 $.  The first thing we do is to ``normalize'' our input by translating it into a monotonically
non-decreasing sequence instead of the monotonically increasing set it's currently represented as. We do this by subtracting
the identity sequence:
$$ \langle 0, 1,\ldots, i\rangle_{i=0}^{m-1} $$
Which is basic component-wise vector substraction:

$$ \begin{array}{rc}
  & \langle 3, 5\rangle  \\
- & \langle 0, 1\rangle  \\ \hline
  & \langle 3, 4\rangle
\end{array} $$

At this point we're now working with the sequence $ \langle 3, 4\rangle  $.

The next step is to \emph{additively} partition our sequence into a kind of natural ``positional notation''---natural relative
to non-decreasing sequences. We split up our sequence into sequences which match the following pattern:
$$ 0^j|c^k\quad :=\quad\langle 0, 0, 0,\ldots\mbox{(j times)}, c, c, c,\ldots\mbox{(k times)}\rangle $$
When I say split up, I mean whatever we partition this vector into, those other vectors (of the same dimension) must add back
up to the original---with the additional constraint that each partition follows the above pattern. In our case study, this leads to:
$$ \langle 3, 4\rangle \quad =\quad\langle 3, 3\rangle +\langle 0, 1\rangle $$
And here's the final conversion, an actual formula:
$$ 0^j|c^k\quad\to\quad {n-j\choose k}-{n-j-c\choose k},\quad j+k=m $$
which in our case $ (n=7; m=2) $, we have
$$ 0^j|c^k\quad\to\quad {7-j\choose k}-{7-j-c\choose k},\quad j+k=2 $$

So for each partition, we map it with the above formula, and add all such maps back together to get our final bijection:
\begin{eqnarray*}
\langle 3, 4\rangle & = & \langle 3, 3\rangle + \langle 0, 1\rangle \\
       & = & 0^0|3^2 + 0^1|1^1 \\
       & = & ({7-0\choose 2} - {7-0-3\choose 2}) + ({7-1\choose 1} - {7-1-1\choose 1}) \\
       & = & ({7\choose 2} - {4\choose 2}) + ({6\choose 1} - {5\choose 1}) \\
       & = & (21 - 6) + (6 - 5) \\
       & = & 15 + 1 \\
       & = & 16 \\
\end{eqnarray*}

To top it all off, if you go back to our enumeration table above, and line up $ 16 $, you get: $ \{3, 5\} $, which is what
we started with.  I'll end with another quick example for clarity, assume again a $ 7 $-set, but in this case a $ 3 $-subset:
\begin{eqnarray*}
\{1, 4, 6\} & \to & \langle 1, 4, 6\rangle - \langle 0, 1, 2\rangle \\
            & = & \langle 1, 3, 4\rangle \\
            & = & \langle 1, 1, 1\rangle + \langle 0, 2, 3\rangle \\
            & = & \langle 1, 1, 1\rangle + \langle 0, 2, 2\rangle + \langle 0, 0, 1\rangle \\
            & = & 0^0|1^3 + 0^1|2^2 + 0^2|1^1 \\
            & = & ({7\choose 3} - {6\choose 3}) + ({6\choose 2} - {4\choose 2}) + ({5\choose 1} - {4\choose 1}) \\
            & = & (35 - 20) + (15 - 6) + (5 - 4) \\
            & = & 15 + 9 + 1 \\
            & = & 25 \\
\end{eqnarray*}

I do not supply the proof here as it is intended to be included in an article I'm pushing to release later. Actually I had hoped
to write the article this year, but I don't know if that'll happen---I have too many other projects keeping me busy. In anycase,
I haven't come across this result,\footnote{regardless of notation, I admit mine might be a bit strange.} and so I'd like to think
it's brand new \emph{even if it's not cutting edge} and this feeling is especially true given the day's effort I put into it,
but my experience in such things is it's probably tucked away somewhere in some old old textbook, discovered long ago. Regardless,
though it doesn't come up often, it is still useful every once in a while, especially in combinatorics or theoretical computing science.
I hope you find it useful.\\[0.5cm]

Daniel Nikpayuk\\[0.5cm]

{\bfseries P.S.} It can probably be optimized.

\end{document}

