% Copyright 2015 Daniel Nikpayuk
\documentclass[twoside]{article}
\usepackage[letterpaper,left=1cm,right=1cm,top=2cm,bottom=2cm]{geometry}
\usepackage{amsmath}
\usepackage{amsfonts}
\usepackage{graphicx}
\usepackage{hyperref}
\usepackage{xcolor}
\usepackage{bm}

\newtheorem{theorem}{Theorem}[section]
\newtheorem{lemma}{Lemma}[section]

\newenvironment{proof}[1][Proof]{\begin{trivlist}
\item[\hskip \labelsep {\bfseries #1}]}{\end{trivlist}}
\newenvironment{definition}[1][Definition]{\begin{trivlist}
\item[\hskip \labelsep {\bfseries Definition (#1):}]}{\end{trivlist}}

\newcommand{\qed}{\nobreak \ifvmode \relax \else
      \ifdim\lastskip<1.5em \hskip-\lastskip
      \hskip1.5em plus0em minus0.5em \fi \nobreak
      \vrule height0.5em width0.5em depth0.25em\fi}

\pagestyle{empty}
\begin{document}

\begin{figure}[h]
\centering
\includegraphics[width=1in]{../../../cc-by-nc.png}\\[0.1in]
\tiny This article is licensed under \\
\href{http://creativecommons.org/licenses/by-nc/4.0/}
{Creative Commons Attribution-NonCommercial 4.0 International.}\\[0.3in]
\end{figure}

This short article provides a quick informal look at the idea that inequality expressions---or intervals---can be viewed
as objects in their own right with their own algebra. As this idea occurred to me recently (at the time of this writing:
September 3rd, 2015) I have additionally googled it and it seems it is a \emph{topological algebra}, something I very much
plan on looking into in the future but as of yet have only heard about. The difference here as stated is this is an informal
treatment, deriving the simplest of results needed for other projects.

Our general inequality expressions are of the following nature:

$$ a\le x\le b $$

In particular, I reserve the lower end of the alphabet (in the expression as: $ a, b $) for constant real numbers $ \mathbb{R} $,
while the upper end of the alphabet (in the expression as: $ x $) is for real number variables. Although we could represent the
same construct with interval notation:

$$ x\in[a,b] $$

for the purpose of this article the interval representation flows better within the narrative of a vector style object.

Informally, the inequality $ a\le x\le b $ has a truth-value: \emph{true} or \emph{false}. If you had need, you could also
give it a partial truth-value ($ a\le x: true $, but $ x\le b $ false for example), but such complicated logics aren't of use
here.  I mention this to emphasize the point that if the whole inequality expression is not \emph{true}, it defaults to
\emph{false}.  Another point to bring up is, although an expression can be false, we are only interested in ones which
are true for our algebra. This is to say, we view the expression to \emph{exist} as an \emph{object} if it is true.

With this in mind, here is our first lemma regarding \emph{componentwise addition}:

\begin{lemma} {$$ a\le b\quad\mbox{and}\quad c\le d\qquad\Longrightarrow\qquad a+c\le b+d $$}
\end{lemma}

\begin{proof}
First we note that
$$ \begin{array}{rclll}
a\le b	&	\Longleftrightarrow	&	0\le b-a	&			&		\\
	&	\Longleftrightarrow	&	0=b-a		&	\mbox{or}	& 	0 < b-a	\\
c\le d	&	\Longleftrightarrow	&	0\le d-c	&			&		\\
	&	\Longleftrightarrow	&	0=d-c		&	\mbox{or}	& 	0 < d-c	\\
\end{array} $$

If $ 0=b-a $ then $ 0\le (d-c)=(d-c)+(b-a) $ which is equivalent to $ a+c\le b+d $. This symmetrically holds for when
$ 0=d-c $. As such, this covers all cases except when $ 0 < b-a $ and $ 0 < d-c $. We take it for granted that a
\emph{positive} number plus a \emph{positive} number equals a \emph{positive} number, so it follows $ 0 < (d-c)+(b-a) $
which is equivalent to $ a+c < b+d $.  Hence, regardless of case, we have at worst $ a+c\le b+d $.\qed

\end{proof}

Moving onto our second lemma, we take a look at subtraction (more or less):

\begin{lemma} $$ a\le b\qquad\Longleftrightarrow\qquad -b\le -a $$
\end{lemma}

\begin{proof}
The proof is as follows:
$$ \begin{array}{rcll}
a\le b	&	\Longleftrightarrow	&	0\le b-a	&		\\
	&	\Longleftrightarrow	&	0\le (-a)-(-b)	&		\\
	&	\Longleftrightarrow	&	-b\le -a	&	\qed	\\
\end{array} $$
\end{proof}

\newpage

As for our next lemma, we move onto multiplication:

\begin{lemma} $$ a\le b\quad\mbox{and}\quad 0\le c\qquad\Longrightarrow\qquad ac\le bc $$
\end{lemma}

\begin{proof}
Similar to the addition lemma, we note that
$$ \begin{array}{rclll}
a\le b	&	\Longleftrightarrow	&	0\le b-a	&			&		\\
	&	\Longleftrightarrow	&	0=b-a		&	\mbox{or}	& 	0 < b-a	\\
0\le c	&	\Longleftrightarrow	&	0=c		&	\mbox{or}	& 	0 < c	\\
\end{array} $$

If $ 0=b-a $ then $ 0\le 0=c(b-a)=cb-ca $ which is equivalent to $ ac\le bc $. This similarly holds for $ 0=c $. As such the
only case left is when $ 0 < b-a $ and $ 0 < c $. We take it for granted that a \emph{positive} number times a \emph{positive}
number is equal to a \emph{positive} number. Hence $ 0 < c(b-a) $ which is equivalent to $ ac < bc $. Thus, regardless of case,
we have at worst $ ac\le bc $.\qed

\end{proof}

For our final lemma, we have division:

\begin{lemma} $$ 0<a\le b\qquad\Longleftrightarrow\qquad 0 < \frac{1}{b}\le\frac{1}{a} $$
\end{lemma}

\begin{proof}
We take for granted that the multiplicative inverse of a \emph{positive} number is a \emph{positive} number.

{\bfseries case: ``$ \Longrightarrow $''}
Since $ 0 < a\le b $ we have both $ 0 < a $ and $ 0 < b $, and so $ 0 < a^{-1}, b^{-1} $. As such, by the multiplication lemma:
$$ \begin{array}{lcrrrr}
a\le b	&	\Longrightarrow	&	aa^{-1}	&	\le	&	ba^{-1}		\\
	&	\Longrightarrow	&	1	&	\le	&	ba^{-1}		\\
	&	\Longrightarrow	&	b^{-1}	&	\le	&	b^{-1}ba^{-1}	\\
	&	\Longrightarrow	&	b^{-1}	&	\le	&	a^{-1}		\\
\end{array} $$

{\bfseries case: ``$ \Longleftarrow $''}
Since $ 0 < b^{-1}\le a^{-1} $ we have both $ 0 < a^{-1} $ and $ 0 < b^{-1} $, and so $ 0 < a, b $. As such, by the multiplication lemma:
$$ \begin{array}{lcrrrrr}
b^{-1}\le a^{-1}	&	\Longrightarrow	&	b^{-1}b	&	\le	&	a^{-1}b		&		\\
			&	\Longrightarrow	&	1	&	\le	&	a^{-1}b		&		\\
			&	\Longrightarrow	&	a	&	\le	&	aa^{-1}b	&		\\
			&	\Longrightarrow	&	a	&	\le	&	b		&	\qed	\\
\end{array} $$

\end{proof}

We have enough now for our theorems:

\begin{theorem}
\color{blue}
$$ a\le x\le b\quad\mbox{and}\quad c\le y\le d\qquad\Longrightarrow\qquad a+c\le x+y\le b+d $$
\end{theorem}

\begin{proof}
$$ \begin{array}{lcrrr}
a\le x\le b	&	\Longleftrightarrow	&	a\le x		&	\mbox{and}	&	x\le b		\\
c\le y\le d	&	\Longleftrightarrow	&	c\le y		&	\mbox{and}	&	y\le d		\\
		&	\Longrightarrow		&	a+c\le x+y	&	\mbox{and}	&	x+y\le b+d	\\
		&	\Longleftrightarrow	&	a+c\le x+y\le b+d	&		&	\qed		\\
\end{array} $$
\end{proof}

\begin{theorem}
\color{blue}
$$ a\le x\le b\qquad\Longleftrightarrow\qquad -b\le -x\le -a $$
\end{theorem}

\begin{proof}
$$ \begin{array}{lcrrr}
a\le x\le b	&	\Longleftrightarrow	&	a\le x		&	\mbox{and}	&	x\le b		\\
		&	\Longleftrightarrow	&	-x\le -a	&	\mbox{and}	&	-b\le -x	\\
		&	\Longleftrightarrow	&	-b\le -x	&	\mbox{and}	&	-x\le -a	\\
		&	\Longleftrightarrow	&	-b\le -x\le -a	&			&	\qed		\\
\end{array} $$
\end{proof}

\newpage

\begin{theorem}
\color{blue}
$$ 0\le a\le x\le b\quad\mbox{and}\quad 0\le c\le y\le d\qquad\Longrightarrow\qquad 0\le ac\le xy\le bd $$
\end{theorem}

\begin{proof}
$$ \begin{array}{lcrrr}
0\le a	&	\mbox{and}	&	0\le c	& \Longrightarrow	&	\textcolor{blue}{0c}\le\textcolor{blue}{ac}	\\
c\le y	&	\mbox{and}	&	0\le a	& \Longrightarrow	&	\textcolor{blue}{ac}\le ay			\\
a\le x	&	\mbox{and}	&	0\le y	& \Longrightarrow	&	ay\le\textcolor{blue}{xy}			\\
y\le d	&	\mbox{and}	&	0\le x	& \Longrightarrow	&	\textcolor{blue}{xy}\le xd			\\
x\le b	&	\mbox{and}	&	0\le d	& \Longrightarrow	&	xd\le \textcolor{blue}{bd}			\\
\end{array} $$
The transitive law assures us that $ 0\le ac\le xy\le bd $\qed
\end{proof}

\begin{theorem}
\color{blue}
$$ 0 < a\le x\le b\qquad\Longleftrightarrow\qquad 0 < \frac{1}{b}\le\frac{1}{x}\le\frac{1}{a} $$
\end{theorem}

\begin{proof}
$$ \begin{array}{lcrrr}
0 < a\le x\le b	&	\Longleftrightarrow	&	0 < a\le x	&	\mbox{and}	&	0 < x\le b		\\
		&				&			&			&				\\
		&	\Longleftrightarrow	&	0 < \frac{1}{b}\le\frac{1}{x}	&	\mbox{and}	&	0 < \frac{1}{x}\le\frac{1}{a}		\\
		&				&			&			&				\\
		&	\Longleftrightarrow	&	0 < \frac{1}{b}\le\frac{1}{x}\le\frac{1}{a}	&	\qed	\\
\end{array} $$
\end{proof}

We finally have enough for some constructive object definitions:

\begin{definition}[additive object]
Let $$ (a\le x\le b)\quad ,\quad (c\le y\le d) $$
be objects, then $$ (a\le x\le b)+(c\le y\le d)\qquad :=\qquad (a+c\le x+y\le b+d) $$
\end{definition}

\begin{definition}[additive inverse object]
Let $$ (a\le x\le b)\qquad $$ be an object, then $$ -(a\le x\le b)\qquad :=\qquad (-b\le -x\le -a) $$
\end{definition}

\begin{definition}[multiplicative object]
Let $$ 0\le(a\le x\le b)\quad ,\quad 0\le(c\le y\le d) $$
be objects, then $$ (a\le x\le b)(c\le y\le d)\qquad :=\qquad (ac\le xy\le bd) $$
\end{definition}

Notice here I used the notation $ 0\le(a\le x\le b) $, though not formalized I mean to intuitively say the object is ``positive'',
which in particular translates as $ 0\le a\le x\le b $.

\begin{definition}[multiplicative inverse object]
Let $$ 0<(a\le x\le b)\qquad $$ be an object, then $$ \frac{1}{(a\le x\le b)}\qquad :=\qquad (\frac{1}{b}\le\frac{1}{x}\le\frac{1}{a}) $$
\end{definition}

A note about these definitions: To be fair, it seems likely with the above definitions you have to take an equivalence
relation to stabilize the objects, but such finer considerations are for an actual theory of intervals as topological algebra.

\newpage

As mentioned in the opening, my intention here is much more the theorems to use for other projects than anything else.
Regarding such other projects: generally \emph{numerical analysis} where one needs to find inequality bounds for error
terms within approximations. The thing is, \emph{real analysis} at its lower levels makes heavy use of inequalities as
well, and although the nature of such ``arithmetic yoga''---as a topology professor of mine once referred to them---is
subtly different, this discrete style of inequality manipulation may be of use there as well.

I'll leave you with a final thought-provoking example. Consider the inequality:
$$ 1\le a\le x\le b $$
Let's say we wanted a bound on $ (x-1)^2 $ instead, we could do this as follows:
$$ \begin{array}{lcrclclcl}
1\le a\le x\le b	& \Longrightarrow & 0 & \le & a-1	& \le & x-1	& \le & b-1	\\
			& \Longrightarrow & 0 & \le & (a-1)^2	& \le & (x-1)^2	& \le & (b-1)^2	\\
\end{array} $$
and so we have our bound already. The thing is, we also know $ (x-1)^2=x^2-2x+1 $,
so we could have also derived our bound in the additional following way:
$$ \begin{array}{lcrclclcl}
1\le a\le x\le b	& \Longrightarrow &	a^2		& \le & x^2		& \le & b^2		\\
			& \Longrightarrow &	-2b		& \le & -2x		& \le & -2a		\\
			& \Longrightarrow &	1		& \le & 1		& \le & 1		\\\hline
			& \Longrightarrow &	a^2-2b+1	& \le & x^2-2x+1	& \le & b^2-2a+1	\\
			& \Longrightarrow &	a^2-2b+1	& \le & (x-1)^2		& \le & b^2-2a+1	\\
\end{array} $$
where we took the top three inequalities and added them together to get the bottom two.

First of all, this alone is thought provoking if you've never really studied it or considered it before---\emph{identity
manipulations don't preserve inequality manipulations!} I mean it makes sense once you really think about it, but when you
spend all your time with identity manipulations you might form the bad habit of expecting preservation. What's more thought
provoking though is this that we have now two differing bounds, so which is better?

To figure this out we note that $ a\le b $ and so $ -2b\le -2a $. In this case then we have
$$ a^2-2b+1\le a^2-2a+1=(a-1)^2 $$
similarly we also know
$$ (b-1)^2=b^2-2b+1\le b^2-2a+1 $$
and so it appears using the factored form $ (x-1)^2 $ gives tighter bounds than the expanded form $ x^2-2x+1 $.
It's also worth noting that if we had instead wondered about $ (x+1)^2 $ our two methods would not have diverged in
the bounds offered (subtraction makes the difference).

What is the hidden pattern here? What are there larger patterns for \emph{best practice} inequality manipulation?
Is there a logical mathematical system of it all? Something to think about\ldots

\end{document}

