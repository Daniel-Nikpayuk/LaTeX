% Copyright 2014 Daniel Nikpayuk
\documentclass[twoside]{article}
\usepackage[letterpaper,left=1in,right=1in,top=0.7in,bottom=0.7in]{geometry}
\usepackage{graphicx}
\usepackage{hyperref}
\usepackage{amsfonts}
\usepackage{amsmath}
\usepackage{amssymb}

\newcommand{\powerset}[2][P]{\ensuremath{\mathbb{#1}(#2)}}
\newcommand{\nthps}[2][P]{\ensuremath{\mathbb{#1}^{#2}}}
\newcommand{\nthus}[2][U]{\ensuremath{\mathbb{#1}^{#2}}}
\newcommand{\psunion}[2][P]{\ensuremath{\bigcup_{#2\ge 0}\mathbb{#1}^{#2}}}
\newcommand{\opsunion}[3][P]{\ensuremath{\bigcup_{#2\ge #3}\mathbb{#1}^{#2}}}
\newcommand{\usunion}[2][U]{\ensuremath{\bigcup_{#2\ge 0}\mathbb{#1}^{#2}}}
\newcommand{\ousunion}[3][U]{\ensuremath{\bigcup_{#2\ge #3}\mathbb{#1}^{#2}}}
\newcommand{\stratified}{\ensuremath{\mathbb{U}^\infty}}
\newcommand{\of}[1]{\ensuremath{(\mathcal{#1})}}

\newcommand{\then}{\ensuremath{\quad\Longrightarrow\quad}}
\newcommand{\equals}{\ensuremath{\quad =\quad}}

\newtheorem{theorem}{Theorem}
\newtheorem{lemma}[theorem]{Lemma}
\newtheorem{proposition}[theorem]{Proposition}
\newtheorem{corollary}[theorem]{Corollary}

\newenvironment{proof}[1][Proof]{\begin{trivlist}
\item[\hskip \labelsep {\bfseries #1}]}{\end{trivlist}}
\newenvironment{definition}[1][Definition]{\begin{trivlist}
\item[\hskip \labelsep {\bfseries #1:}]}{\end{trivlist}}
\newenvironment{example}[1][Example]{\begin{trivlist}
\item[\hskip \labelsep {\bfseries #1}]}{\end{trivlist}}
\newenvironment{remark}[1][Remark]{\begin{trivlist}
\item[\hskip \labelsep {\bfseries #1}]}{\end{trivlist}}

\newcommand{\qed}{\nobreak \ifvmode \relax \else
      \ifdim\lastskip<1.5em \hskip-\lastskip
      \hskip1.5em plus0em minus0.5em \fi \nobreak
      \vrule height0.75em width0.5em depth0.25em\fi}

\title{A Computational Model of Reference by means of Stratified Powersets (Part One)}
\author{Daniel Nikpayuk}
\date{June 14, 2014}

\pagestyle{empty}
\begin{document}
\maketitle
\thispagestyle{empty}

\begin{figure}[h]
\centering
\includegraphics[width=1in]{../../../cc-by-nc.png}\\[0.1in]
\tiny This article is licensed under \\
\href{http://creativecommons.org/licenses/by-nc/4.0/}
{Creative Commons Attribution-NonCommercial 4.0 International.}\\[0.3in]
\end{figure}

\begin{abstract}
Here I introduce a set theoretic model of \emph{data structures} intended to be used within the theory of computation.
I aim to provide a \emph{context space} for which to validate other related specifications of computational design such
as semiotic filters. I further intend to use this contextualization to define and implement a \emph{model of reference}
as an alternative to the standard environment model (frames and bindings) used by interpreters. As this is a three part
article, the second and third will focus on the implementation.

Finally, I will note that there are many computing science situations where a category theoretic lens is more suitable.
I have chosen a set theoretic framework as the focus here because emphasis in this design is placed on context and structure
rather than representation and function.
\end{abstract}

\section*{Motivation}

We start at the beginning.\\[0.1cm]

We start at the beginning and philosophically interrogate what a \emph{data structure} even is, and yet before we can talk about
the structure, we must begin with the data. But for sake of clarity, let's use the word \emph{content}.

So we start from nothing except the content we wish to structure. As it stands, if you give it some
thought, we already have a structure---the content itself. The very idea of ``content'' implies we have grouped together
objects of interest. We might not know how or why they are grouped together, but that they are grouped together. This though,
is exactly already the idea of a \emph{set}. We will be using set theory as our foundation in this article.

In a general sense, how does one represent data structures using set theory?  The nature of the word ``structure''
in a conceptual sense is that of creating \emph{divisions}. Our initial content is a division between what's in this set
and what's \emph{not} in this set---which is to say everything else in the universe. Beyond that, the most basic data
structure is anything that creates a single additional division built on top of our initial division.

Though it is always debatable, I would claim in a general sense the most
basic data structure beyond that of a set is an \emph{ordered pair}:
$$ (a,b) $$
Let's give this a little thought: The ordered pair as an intuitive specification has three properties: First, two objects `$ a $'
and `$ b $' are collected together. Second, one of these objects is by convention assigned the connotation of being ``first''
and the other as being ``last'' (this being the next basic division built upon our initial division). So to start with set theory,
we would first collect together the objects of interest:
$$ \{a,b\} $$
Here we've already run into our first problem: We have a pair, but it's not ordered. How do we add additional structure which
provides enough information that we can infer which is first and which is last? As it turns out there are many different ways
to do this, and what's more there's no reason to say in a general sense that any way is better than any other---so long as they
effectively communicate their intent, they have equivalent potential---they are \emph{equipotent}.

We do have an additional guideline to help us narrow down our design though: Whatever structure we use, it should have the highest
entropy (least possible constraints). In this context it should be as simple as possible (minimalistic)---so as to prevent
additional semantic content being implicitly interpreted which would otherwise have nothing to do with the intuitive idea
of an ordered pair. Here is the classical definition:
$$ (a,b):=\{a,\{a,b\}\} $$
It is worth noting that in our \emph{culture's} value system\ \footnote{As of the time and place this article
was written of course.} we tend to privilege ``first'' over ``last''. Thus, regarding an ordered pair, we can use
the knowledge of this cultural privilege as a convention to remind ourselves of which element is to be interpreted
as ``first'' by including it along with the unordered pair (hence the above definition).  Furthermore, by collecting
'$ a $' together with $ \{a,b\} $ we in fact provide just enough information to represent our data structure, and no more.

It's also worth noting that if in our value system we valued ``last''
before ``first'' we could've implemented an ordered pair equivalently as:
$$ (a,b):=\{b,\{a,b\}\} $$
The main point is not a matter of universal \emph{right} or \emph{wrong}, but a matter of \emph{consistency}.
If we use one implementation within a given context, we either stick to that implementation or have clear
rules as to when to switch---so as to not provide confusion.

So what about a triple? As an example we could implement it like this:
$$ (a,b,c):=(a,(b,c)) $$
which is a pair whose last element is itself a pair. Translating this further into set theoretic terms:
$$ (a,(b,c))=\{a,\{a,\{b,\{b,c\}\}\}\} $$
Or, we might want to define a tree-like structure using ordered pairs:
$$ ((a,b),(c,d))=\{\{a,\{a,b\}\},\{\{a,\{a,b\}\},\{c,\{c,d\}\}\}\} $$
At this level it is easy for critics to attack: Too confusing! Too many left/right curly braces!
Too much repeated use of the symbols `$ a $', `$ b $' , `$ c $', `$ d $' on the right hand side
(especially because they occur only once on the left)! \ldots

All of this is true of course, but such considerations will be explicated upon later in the theory (in part three).
Let's for now focus on \emph{constructive} pattern recognition.

To help simplifiy our discussions, the intuitive concept of \emph{nesting depth} might be of use. For example `$ a $' has
a nesting depth of \emph{zero} while `$ \{a\} $' has a nesting depth of \emph{one} and `$ \{\{a\}\} $' has a nesting depth
of \emph{two}. We could keep going ad infinitum, as such nested sets of the original content would likely be useful
in constructing arbitrary data structures.

This then, is the first motivation in defining a context for data structures: If we have an initial content set $ \mathcal{S} $,
we will need access to its powers of powersets:\ \footnote{The powerset of course being the set of all subsets of $ \mathcal{S} $,
denoted as $ \powerset{S} $.}
\begin{eqnarray*}
\nthps{0}\of{S} & := & \mathcal{S} \\
\nthps{1}\of{S} & := & \powerset{S} \\
\nthps{2}\of{S} & := & \powerset{\powerset{S}} \\
\ldots & := & \ldots \\
\nthps{k}\of{S} & := & \powerset{\nthps{k-1}\of{S}},\quad k\ge 1
\end{eqnarray*}

\newpage

On the other hand, having access to even the full recursive range of a given contents' powersets it turns out is not
sufficient.  For example, to informally construct our ordered pair $ (a,b) $ from set theoretic operations, we would have:
\begin{eqnarray*}
\{a\}\subseteq \mathcal{S} & \mbox{and} & \{\{a,b\}\}\subseteq\powerset{S} \\
 & & \\
(a,b) & = & \{a,\{a,b\}\} \\
 & = & \{a\}\cup\{\{a,b\}\} \\
 & \Longrightarrow & (a,b)\subseteq S\cup\powerset{S} \\
\end{eqnarray*}

A question arises, what if:
$$ (a,b)\not\subseteq\mathcal{S}\mbox{ and }(a,b)\not\subseteq\powerset{S}\quad ? $$
This is to say, what if our ordered pair as a set is inaccessible by means of single powers of powersets alone?
It should be mentioned that although it's possible for $ \nthps{n}, \nthps{n+1} $ to share members for arbitrary
$ n $, in the general case this does not hold. There is a well known proof that a powerset will always have an
element which does not belong to the
original:\ \footnote{The cardinality of a powerset is greater than that of the cardinality of the original.}
$$ \powerset{S}-\mathcal{S}\neq\emptyset $$
The point of all this is that there will always exist---even with our second simplest data structure---ordered
pairs $ (a,b) $, $ a,b\in\mathcal{S} $ that are not subsets or members of any individual $ \nthps{k}\of{S} $ for arbitrary
$ k\ge 1 $. For our interests then, we will need a wider scope.

This provides our second motivation: Not only do we need access to $ \nthps{k}\of{S} $ for arbitrary $ k $
within our context of discourse, but we need to be able to mix and match nesting depths. One might say
$$ \psunion{k}\of{S} $$
is our ideal definition of context, but as it turns out it even it is not large enough to preserve some preferred
set theoretic operations when working with data structures as sets. Such is our final motivation, in what now follows.

\section*{Definitions}

We extend our intuitive motivations to a class of similar structures:
\begin{definition}[Unified Powersets]
\begin{eqnarray*}
\nthus{0}\of{S} & := & \mathcal{S} \\
\nthus{n}\of{S} & := & \psunion{k}(\nthus{n-1}\of{S}),\quad n\ge 1 \\
\end{eqnarray*}
\end{definition}
Even here, for each $ \nthus{n} $ all it takes is its powerset $ \powerset{\nthus{n}} $ to generate never before seen
data structures---likely of interest---which do not belong to our current universe. Individual $ \nthus{n} $ will not suffice
themselves as our universal context.

At this point then we are ready to introduce our discourse for data structures:
\begin{definition}[Stratified Powerset]
$$ \stratified\of{S} := \usunion{n}\of{S} $$
\end{definition}

Oh, by the way---you may have already noticed, but---if it is clear from the context or otherwise no confusion will arise,
we will omit $ \mathcal{S} $ from any of the below notation.

\section*{Lemmata}

This article will pretty much end in a few basic but important theorems, and in order to get us there and to simplify
their proofs a handful of lemmas will be of use:

\begin{lemma}[Monotonicity]
$$ 0\le m\le n\then\nthus{m}\subseteq\nthus{n} $$
\end{lemma}

\begin{proof}

Equivalently
$$ 0\le\ell, m\then\nthus{m}\subseteq\nthus{m+\ell} $$
Our proof then is by induction:
\begin{itemize}
\item case $ \ell=0 $: trivial.
\item case $ \ell\Longrightarrow\ell+1 $:
\begin{eqnarray*}
\nthus{m+(\ell+1)} & = & \psunion{k}(\nthus{m+\ell})=\nthus{m+\ell}\cup\opsunion{k}{1}(\nthus{m+\ell}) \\
& \Longrightarrow & \nthus{m+\ell}\subseteq\nthus{m+(\ell+1)}\qed
\end{eqnarray*}
\end{itemize}
\end{proof}
This lemma shows up in the latter, but is also of interest in that it justifies our notation `$ \stratified $'.
Note also that $ \nthus{n}\subseteq\stratified, n\ge 0 $.

\begin{corollary}[Mobility]
$$ 0\le m\le n,\qquad A\in\nthus{m}\then A\in\nthus{n} $$
\end{corollary}

\begin{proof}
Trivial, by the very definition of a subset and a bit of transitivity.\qed
\end{proof}

\begin{lemma}[Subset to Member]
$$ A\subseteq\nthus{n}\then A\in\nthus{n+1},\quad n\ge 0 $$
\end{lemma}

\begin{proof}
$$ A\subseteq\nthus{n}\then A\in\powerset{\nthus{n}}\subseteq\psunion{k}(\nthus{n})=\nthus{n+1}\qed $$
\end{proof}

\begin{lemma}[Reduction]
$$ A,B\in\stratified\then\exists m\ge 0,\quad A,B\in\nthus{m} $$
\end{lemma}

\begin{proof}
$$ A,B\in\stratified=\usunion{n}\then\exists \ell\ge 0, A\in\nthus{\ell}\quad\mbox{and}\quad\exists m\ge 0, B\in\nthus{m} $$
Without loss of generality, let $ \ell\le m $, then $ \nthus{\ell}\subseteq\nthus{m} $, hence $ A,B\in\nthus{m} $.\qed
\end{proof}

\begin{lemma}[Stratification]
$$ \nthus{n}\equals\nthus{0}\cup\bigcup_{\substack{k\ge 1 \\ 0\le m < n}}\nthps{k}(\nthus{m}),\quad n\ge 1 $$
\end{lemma}

\begin{proof}
First notice our claim is equivalent to:
$$ \nthus{n}\equals\nthus{0}\cup\bigcup_{k\ge 1}\left(\bigcup_{0\le m < n}\nthps{k}(\nthus{m})\right) $$
We use induction.
\begin{itemize}
\item case $ n=1 $:
\begin{eqnarray*}
\nthus{1}\equals\psunion{k}(\nthus{0}) & = & \nthus{0}\cup\opsunion{k}{1}(\nthus{0}) \\
 & = & \nthus{0}\cup\bigcup_{k\ge 1}\left(\bigcup_{0\le m < 1}\nthps{k}(\nthus{m})\right)
\end{eqnarray*}
\item case $ n\Longrightarrow n+1 $:
\begin{eqnarray*}
\nthus{n+1} & = & \psunion{k}(\nthus{n})\equals\nthus{n}\cup\opsunion{k}{1}(\nthus{n}) \\
 & & \\
 & & \mbox{by the inductive step, we have} \\
 & & \\
 & = & \nthus{0}\cup\bigcup_{k\ge 1}\left(\bigcup_{0\le m < n}\nthps{k}(\nthus{m})\right)\cup\opsunion{k}{1}(\nthus{n}) \\
 & = & \nthus{0}\cup\bigcup_{k\ge 1}\left(\bigcup_{0\le m < n+1}\nthps{k}(\nthus{m})\right)\qed
\end{eqnarray*}
\end{itemize}
\end{proof}
With this lemma at hand, we have use for a definition:

\begin{definition}[Collectibles]
Let $ A\in\stratified $, $ A $ is said to be \emph{collectible} if
$$ \exists j\ge 0, \mbox{ and }\exists k\ge 1,\quad\mbox{such that}\quad A\in\nthps{k}(\nthus{j}) $$
\end{definition}
Note that the vast majority\ \footnote{Informally speaking of course.} of $ \stratified $ is collectible, which is to
say for $ A\not\in\nthus{0} $ we have that $ A $ satisfies collectibility.  One might be tempted to redefine collectible
as $ \stratified-\nthus{0} $ but our intention is to define ``collectible'' as a member of $ \stratified $ which acts
like a non-trivial data structure, and all such members tend to belong to the powersets (of powers greater than or equal
to one).  Let's say for example $ A\in\nthps{3}\of{S} $ but also $ A\in\mathcal{S} $. This may just be a coincidence,
but because it belongs to $ \nthps{3}\of{S} $ we have interest in it as a data structure. From the point of view of
it belonging to $ \mathcal{S} $ we have no interest in it, but if we were to use its presence in $ \nthus{0} $ to remove
it from $ \stratified $ we'd lose access to it altogether, even when we're interested in it only from the perspective of
a data structure.

This ends the list of lemmas to be used in our theorems. What remains are ones I've catalogued in my research
which may or may not be of use in further studies of our stratified powerset. I provide these lemmas here in case
they turn out to be useful---it might save others time and effort in reinventing wheels.

\subsection*{Other Lemmas}

As these all have simple proofs, I will leave them as an exercise.
\begin{lemma}\
\begin{eqnarray}
A\subseteq B & \then & \powerset{A}\subseteq\powerset{B} \\[0.5cm]
\nthus{n} & \subseteq & \nthus{n-1}\cup\powerset{\nthus{n}},\quad n\ge 1 \\[0.2cm]
\nthus{n} & \equals & \bigcup_{k\ge 0}\left(\bigcup_{0\le m < n}\nthps{k}(\nthus{m})\right),\quad n\ge 1 \\[0.2cm]
\stratified & \equals & \bigcup_{k,m\ge 0}\nthps{k}(\nthus{m})
\end{eqnarray}
\end{lemma}

\newpage

\section*{Theorems}

The emphasis on the theorems here is that of closure properties:

\begin{theorem}[Closure of Pair]
$$ A,B\in\stratified\then\{A,B\}\in\stratified $$
\end{theorem}

\begin{proof}
Since $ A,B\in\stratified $ we know there exists $ m\ge 0 $:
\begin{eqnarray*}
A,B\in\nthus{m} & \Longrightarrow & \{A\},\{B\}\subseteq\nthus{m} \\
 & \Longrightarrow & \{A\}\cup\{B\}\subseteq\nthus{m}\cup\nthus{m} \\
 & \Longrightarrow & \{A,B\}\subseteq\nthus{m} \\
 & \Longrightarrow & \{A,B\}\in\nthus{m+1}\subseteq\stratified\qed
\end{eqnarray*}
\end{proof}

\begin{theorem}[Near Algebra of Sets]
Let $ A,B\in\stratified $ with $ A,B $ as collectibles:
\begin{enumerate}
\item $ A\cup B\in\stratified $
\item $ A\cap B\in\stratified $
\item $ A-B\in\stratified $
\end{enumerate}
\end{theorem}

\begin{proof}\
Since $ A,B $ are collectible, there exist $ j_1,j_2\ge 0 $ and $ k_1,k_2\ge 1 $ such that
$ A\in\nthps{k_1}(\nthus{j_1}),\ B\in\nthps{k_2}(\nthus{j_2}) $, thus
\begin{eqnarray*}
A\subseteq\nthps{k_1-1}(\nthus{j_1}) & \mbox{and} & B\subseteq\nthps{k_2-1}(\nthus{j_2}) \\
 & & \\
A\subseteq\nthus{j_1+1} & \mbox{and} & B\subseteq\nthus{j_2+1} \\
\end{eqnarray*}
\setcounter{equation}{0}
Let $ j_0:=\mbox{max}(j_1,j_2) $. For our three theorem properties, we then have:
\begin{align}
A\cup B\subseteq\nthus{j_0+1} & \then A\cup B\in\nthus{j_0+2}\subseteq\stratified \\
A\cap B\subseteq A\subseteq\nthus{j_1+1} & \then A\cap B\in\nthus{j_1+2}\subseteq\stratified \\
A-B\subseteq A\subseteq\nthus{j_1+1} & \then A-B\in\nthus{j_1+2}\subseteq\stratified\qed
\end{align}
\end{proof}
With these theorems complete I will comment that for any mathematician who might still be aesthetically
dissatisfied---notably with our ad-hoc inelegant ``collectibles'' concept---I recommend taking
$ \mathcal{S}=\emptyset $, which is to say $ \stratified\of{\emptyset} $. \emph{Why?}

\section*{Conclusion}

Now that we have our theoretical stratified powerset, how do we translate it into a computational model of reference?
That will be the topic of \emph{part two}.

I will leave the reader with a few additional philosophical thoughts on the general nature of sets. Notably:

From an implementation point of view, how does one navigate an arbitrary set? When working with a set, since it is the most
fundamental data structure, it does not provide you ways of navigating elements with respect to other elements. In order to do
that you need known semantic relationships between those elements, for example a ``first'' or ``last'' or ``next'' element
relative to the one at hand, but given an arbitrary set of unknown type you can only know if a given object belongs or not.
Semiotic signifiers such as ``next'' only have meaning if you know that the set in question has additional structure
which supports the idea of ``next''.
\begin{quote}
\emph{If one cannot rely on signifiers to access the elements of a set, how does one traverse the territory carved out by a set?}
\end{quote}
I mention this as a philosophical question because it is one I find too many take for granted. My own opinion is that
in terms of implementation, one needs to privilege a single \emph{identity} data structure---likely an ordering relation---which
otherwise possesses the same content as the set; one which provides just enough information (and no more) in order to access
that content. This is to say, it is taken for granted that in practice, without realizing it, there is already bias in
the system. This area of analysis by the way is centered around one additional concept for which we shall not introduce here:
\emph{the media space}.

Pijariiqpunga.

\end{document}

