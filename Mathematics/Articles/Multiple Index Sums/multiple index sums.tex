% Copyright 2014 Daniel Nikpayuk
\documentclass[twoside]{amsart}
\usepackage[letterpaper,left=1cm,right=1cm,top=1.5cm,bottom=2cm]{geometry}
\usepackage{amssymb}
\usepackage{graphicx}
\usepackage{hyperref}

\newcommand{\bu}[1][u]{\ensuremath{\mathbf #1}}

\title{An interesting class of multiple index sums}
\author{Daniel Nikpayuk}
\date{December 8, 2014}
\begin{document}
\maketitle

\begin{figure}[h]
\centering
\includegraphics[width=1in]{../../../cc-by-nc.png}\\[0.1in]
\tiny This article is licensed under \\
\href{http://creativecommons.org/licenses/by-nc/4.0/}
{Creative Commons Attribution-NonCommercial 4.0 International.}\\[0.3in]
\end{figure}

\section{Overview}

The sums in this article are a generalization of the ones occuring
in the book \emph{Concrete Mathematics} \cite{gkp} of the form:

$$ \sum_{0\le k\le n}f(k) $$

As far as this article is concerned, I have cannibalized portions of an older larger one (of mine)
into what's currently before you; I give only a quick tour of the main ideas and results.

As far as the generalization of sums is concerned, it is worth noting there are many
classes of multiple index sums whose forms quite readily retain many of the original
manipulation laws.  Thus the one I'm interested in preserving in this article is:

\begin{eqnarray*}
\sum_{0\le j\le k\le n}f(j,k)
 & = & \sum_{(0\le j\le n)}\sum_{(j\le k\le n)}f(j,k) \\  
 & = & \sum_{(0\le k\le n)}\sum_{(0\le j\le k)}f(j,k)       
\end{eqnarray*}

which is called the \emph{double sum} or \emph{summand switch law}.

\section{Definitions and Terminology}

It will be easiest to go straight to the multiple index sums I have in mind:

$$ \sum_{0\le u_1\le u_2\le u_3\le\ldots\le u_{k-1}\le u_k\le n}f(u_1, u_2, u_3,\ldots, u_{k-1}, u_k) $$

As you can see, it is not too pretty to look at, write, copy, or work with by hand.  To shortform the various indices,
I will use $ k $-tuples.  In particular I use vector style notation $ \bu $:

$$ \sum_{(0\le\bu\le n)}^k\!\!\!\!\! f(\bu)  $$

Thus
$$ \bu:=(u_1, u_2,\ldots, u_{k-1}, u_k)  $$

Here I use `$ := $' to mean ``by definition equals''.
Also, I take for granted that the coordinates are integers $ \mathbb{Z} $ that are non-decreasing in order.
We write the number $ k $ on top of the Sigma $ \Sigma $ to indicate the length of $ \bu $.
If you wish to refer to a particular coordinate, say $ u_5 $ for example, you simply write
$ \bu_{|5|} $.  If you wish to refer the $ j^{\mbox{\scriptsize{th}}} $ coordinate, you simply write:

$$  \bu_{|j|}  $$

There is one warning I feel I should mention; be careful in avoiding circular definitions:
$$  \bu:\ne(\bu_{|1|},\ldots,\bu_{|k|})  $$

Regarding the notation for the sum above, there is still the issue of order $ \le $.  I will get to that after
I define a \emph{concatenation} operation on the $ k $-tuples of various lengths $ k $.  This is a new operation
that requires new notation.  The symbol I've decided to use is `$  \uplus  $', the process in question then is
to take some $  k  $-tuple, and some $  \ell  $-tuple vectors, and join them to produce a $  (k+\ell)  $-tuple vector:

If $  \bu =(\bu_{|1|},\ldots ,\bu_{|k|})  $, and $  \bu[v] =(\bu[v]_{|1|},\ldots ,\bu[v]_{|\ell|})  $,
then define 

$$  \bu\uplus\bu[v]:=(\bu_{|1|},\ldots ,\bu_{|k|},\bu[v]_{|1|}, \ldots ,\bu[v]_{|\ell|})  $$

Not that it's actually relevant, but if you're interested, this operation \emph{is} associative,
clearly \emph{not} commutative, and most interestingly \emph{has} the following distributive law:
$$  (\bu+\bu[v])\uplus(\bu[w]+\bu[x])=\bu\uplus\bu[w]+\bu[v]\uplus\bu[x]  $$
the usefulness of which, is yet to be found.

As for the ordering property $ 0\le\bu\le n $, since $ \bu $ is non-decreasing, it is never a problem to say $ 0\le\bu $
so long as $ 0\le\bu_{|1|} $.  A symmetrical intuition applies to $ \bu\le n $.  What works out to be most
useful though is that if $ \bu $ is some $ k $-tuple, and $ \bu[v] $ is some $ \ell $-tuple, we can say
that $ \bu\le\bu[v] $ holds as long as the last coordinate of $ \bu $ is less than or equal to the first coordinate
of $ \bu[v] $, which is to say $ \bu_{|k|}\le\bu[v]_{|1|} $.  For example:
$$  (-15,1,3,7,10)\le (30,31)  $$  

Now if one explores this ordering relation in more detail one discovers the following:
\begin{eqnarray*}
(\bu\le\bu[v])\quad\mbox{and}\quad(0\le\bu\uplus\bu[v]\le n)
 & \iff & (0\le\bu\le n)\quad\mbox{and}\quad(\bu\le\bu[v]\le n)                  \\
 & \iff & (0\le\bu[v]\le n)\quad\mbox{and}\quad(0\le\bu\le\bu[v])
\end{eqnarray*} 

This leads to the generalization of the much important summand switch laws:

\begin{eqnarray*}
\sum_{(0\le\bu\uplus\bu[v]\le n)}^{j,k}                                       
       \!\!\!\!\!\!\!\!\! f(\bu\uplus\bu[v])                                 
 & = & \sum_{(0\le\bu\le n)}^j                                           
       \sum_{(\bu\le\bu[v]\le n)}^k\!\!\!\!\!\! f(\bu\uplus\bu[v])     \\  
 & = & \sum_{(0\le\bu[v]\le n)}^k                                          
       \sum_{(0\le\bu\le\bu[v])}^j\!\!\!\!\!\! f(\bu\uplus\bu[v])       
\end{eqnarray*}

Notice on top of the Sigma $ \Sigma $ the pair $ j, k $ to keep track of the lengths of $ \bu, \bu[v] $ respectively.
I will take it for granted here, but if you explore these sums in more detail, you'll discover they parallel the ones
again in \cite{gkp}, regarding manipulation laws.

\section{Applications}

The reason I've spent time exploring these sums at all is that they are intimately related to recursive equations,
also known as discrete initial value problems I believe.  These sums are able to provide discrete closed forms
if you're interested in them, in particular:

\begin{eqnarray*}
\sum_{(0\le\bu\le n)}^k\!\!\!\!\!
	(\bu_{|1|}+1)(\bu_{|2|}+1)\ldots(\bu_{|k|}+1)
 & = & \left\{\!\!\!\begin{array}{c} k+n+1 \\ n+1 \end{array}\!\!\!\right\} \\
\sum_{(0\le\bu\le n)}^k\!\!\!\!\!
	(\bu_{|1|}+1)(\bu_{|2|}+2)\ldots(\bu_{|k|}+k)
 & = & \left[\!\!\!\begin{array}{c} k+n+1 \\ n+1 \end{array}\!\!\!\right]   
\end{eqnarray*}

First of all, the notation:
$  \left\{\!\!\!\begin{array}{c} k \\ n \end{array}\!\!\!\right\}  $, as
used in \cite{gkp} denotes the Stirling Numbers of the Second Kind, whose
combinatorial interpretation is that of the number of ways to partition a
$  k  $-set into $  n  $ non-empty disjoint subsets.  As well, the notation:
$  \left[\!\!\!\begin{array}{c} k \\ n \end{array}\!\!\!\right]  $, also
imported from \cite{gkp}, denotes the Stirling Numbers of the First Kind,
whose combinatorial interpretation is that of the number of permutations of
size $  k  $ (meaning the ones acting on a set of $  k  $ objects), containing 
exactly $  n  $ cycles in their standard unique cycle decomposition.

\newpage

Truthfully though, everything starts with this one:
$$ \sum_{(0\le\bu\le n)}^k\!\!\!\!\! 1={k+n\choose n} $$
whose combinatorial proof is simple; it's actually just ``choice with repetition'': The sum on the
left counts the number of ways to choose $ k $ objects with repetition from a collection of $ n+1 $
objects ($ 0 $ through $ n $). Thus there are $ (n+1)-1 $ ``dividers'' or ``flags'' to position amongst
(or choose from) $ k+(n+1)-1 $ multiset positions, leading to $ {k+n\choose n} $ which is the count on
the right side.

A secondary proof is equally instructive, informative, and worth remembering as a trick: Map the
property of this sum $ (0\le\bu\le n) $ with the following permutation:
$ p(\bu)=(\bu_{|1|}+1, \bu_{|2|}+2, \ldots, \bu_{|k|}+k) $, which if you notice, converts our
non-decreasing sequence of integers to a strictly increasing one\footnote{informally: $ (0 < \bu < n+k+1) $}
which if you notice actually translates out the need for ``with repetition'' leaving us with regular ``choice'':
$$ {n+k\choose k}\quad\mbox{equivalently}\quad {k+n\choose n}\qed $$

As always, combinatorial proofs are invaluable, but algebraic ones in this case are actually more appropriate:

We start with a proof-convenient form of what I believe is termed Pascal's rule:

$$ {k+n\choose n}={k-1+n\choose n}+{k+n-1\choose n-1} $$

and unfold it recursively (using the rightmost as the recusive term):

\begin{eqnarray*}
{k+n\choose n} & = & {k-1+n\choose n}+{k-1+n-1\choose n-1}+{k+n-2\choose n-2} \\
 & = & {k-1+n\choose n}+{k-1+n-1\choose n-1}+{k-1+n-2\choose n-2}+{k+n-3\choose n-3} \\
 & = & {k-1+n\choose n}+{k-1+n-1\choose n-1}+{k-1+n-2\choose n-2}+\ldots
\end{eqnarray*}

Eventually the last term runs out, and with a small adjustment to said last term one ends up with:
$$ {k+n\choose n}=\sum_{0\le u\le n}{k-1+n-u\choose n-u} $$
whose form we adjust as:
$$ {k+n\choose n}=\sum_{0\le u\le n}{k-1+u\choose u} $$

This is where things change from the ordinary and the recursive nature of this style of indexing of sums is expressed;
the next step is the same as before, we unfold:

\begin{eqnarray*}
{k+n\choose n} & = & \sum_{(0\le u\le n)}{k-1+u\choose u} \\
 & = & \sum_{(0\le u\le n)}\sum_{(0\le v\le u)}{k-2+v\choose v} \\
 & = & \sum_{(0\le\bu_{|2|}\le n)}^1\sum_{(0\le\bu_{|1|}\le\bu_{|2|})}^1{k-2+\bu_{|1|}\choose \bu_{|1|}} \\
 & = & \sum_{(0\le\bu\le n)}^2{k-2+\bu_{|1|}\choose \bu_{|1|}} \\
 &   & \\
 & = & \ldots \\
 &   & \\
 & = & \sum_{(0\le\bu\le n)}^k{k-k+\bu_{|1|}\choose \bu_{|1|}} \\
 & = & \sum_{(0\le\bu\le n)}^k 1 \\
\end{eqnarray*}

The proof of the Stirling Numbers is the same.

\subsection{a table of results}

I have accumulated a table of a few of the simpler sums of this kind, of which I have also derived
algebraically.

% make an actual table environment you lazy ass.

\begin{eqnarray}
\sum_{(0\le\bu\le n)}^k\!\!\!\!\! \bu_{|s|}
 & = & s{k+n\choose k+1}                                                    \\
\sum_{(0\le\bu\le n)}^k\!\!\!\!\! 1\cdot\bu  
 & = & \frac{kn}{2} {k+n\choose k}                                          \\ 
\sum_{(0\le\bu\le n)}^k\!\!\!\!\! t^{\bu_{|1|}} 
 & = & \sum_{0\le j\le n}{k+n\choose k+j}(t-1)^j                            \\
\sum_{(0\le\bu\le n)}^k\!\!\!\!\! t^{\bu_{|k|}} 
 & = & \sum_{0\le j\le n}{k+n\choose k+j}t^{n-j}(t-1)^j                     \\
\sum_{(0\le\bu\le n)}^k\!\!\!\!\! t^{1\cdot\bu}           
 & = & \prod_{1\le j\le k}\frac{1-t^{n+j}}{1-t^{j\ \ \ }}                   
\end{eqnarray}  

Here the $ 1\cdot\bu $ is the dot product meaning $ 1\cdot\bu=\bu_{|1|}+\bu_{|2|}+\ldots+\bu_{|k|} $.

\subsection{a table of stronger results}

\begin{eqnarray}
\sum_{(0\le\bu\le n)}^k\!\!\!\!\! \bu_{|s+1|}^m 
 & = & \sum_{1\le j\le m}j!\left\{\!\!\!\begin{array}{c} m \\ j 
       \end{array}\!\!\!\right\}{s+j\choose j}{k+n\choose k+j}              \\
\sum_{(0\le\bu\le n)}^k\!\!\!\!\! (\bu_{|1|}^m+\ldots +\bu_{|k|}^m)
 & = & \sum_{1\le j\le m}j!\left\{\!\!\!\begin{array}{c} m \\ j 
       \end{array}\!\!\!\right\}{k+j\choose 1+j}{k+n\choose k+j}            \\
\sum_{(0\le\bu\le n)}^k\!\!\!\!\! t^{\bu_{|s+1|}}
 & = & \sum_{0\le j\le n}{s+j\choose j}{k+n\choose k+j}(t-1)^j                
\end{eqnarray} 

\subsection{partial proofs}

I have included some partial proofs of some of these results.  I start with (1):

\begin{eqnarray*}
{k+1+n\choose k+1} & = & \sum_{(0\le\bu[w]\le n)}^{k+1}\!\!\!\!\! 1    \\
 & = & \sum_{(0\le\bu[v]\uplus\bu\le n)}^{1,k}\!\!\!\!\!\!\!\! 1       \\
 & = & \sum_{(0\le\bu\le n)}^{k}                                       
       \sum_{(0\le\bu[v]\le\bu)}^1\!\!\!\!\! 1                         \\
 & = & \sum_{(0\le\bu\le n)}^{k}\!\!\!\!\!(\bu_{|1|}+1)                \\
 & = & \sum_{(0\le\bu\le n)}^k\!\!\!\!\! \bu_{|1|}
                               +\sum_{(0\le\bu\le n)}^k\!\!\!\!\! 1    \\      
\end{eqnarray*} 

$$  \Longrightarrow\sum_{(0\le\bu\le n)}^{k}\!\!\!\!\! \bu_{|1|}
     ={k+1+n\choose k+1}-{k+n\choose k}={k+n\choose k+1}\qed  $$

A frequent strategy is the one used in \cite{gkp} where one partitions a particular sum using its index property.
As example I would partition a two index sum with index property:

$$ 0\le j\le k\le n $$

by seperating it into sums with $ j > 0 $ and $ j=0 $:

$$ 0\le j\le k\le n\quad\Longleftrightarrow\quad 0 < j\le k\le n\quad\mbox{ or }\quad 0=j\le k\le n $$

Generalizing this---and skipping a few housekeeping steps---we have:

\begin{eqnarray*}
\sum_{(0\le\bu\le n)}^{k+1}\!\!\!\!\! \bu_{|s|}
 & = & \sum_{(0\le\bu\le n)}^{k}\!\!\!\!\! \bu_{|s-1|}+\!\!\!\!\!\!\!\!\!\!
       \sum_{\ \ \ \ (0\le\bu\le n-1)}^{k+1}
            \!\!\!\!\!\!\!\!\!\!\!\! (\bu_{|s|}+1)                          \\
 & = & \sum_{(0\le\bu\le n)}^{k}\!\!\!\!\! \bu_{|s-1|}+{k+n\choose k+1}
        +\!\!\!\!\!\!\!\!\!\!\!\sum_{\ \ \ \ (0\le\bu\le n-1)}^{k+1}
         \!\!\!\!\!\!\!\!\!\!\!\! \bu_{|s|}                                 
\end{eqnarray*}

Again we apply this same trick, but to the right side of the inequality:

\begin{eqnarray*}
\sum_{(0\le\bu\le n)}^{k+1}\!\!\!\!\! \bu_{|s|}
 & = & \sum_{(0\le\bu\le n)}^{k}\!\!\!\!\! \bu_{|s|}
       +\!\!\!\!\!\!\!\!\!\sum_{\ \ \ \ (0\le\bu\le n-1)}^{k+1}
       \!\!\!\!\!\!\!\!\!\!\!\! \bu_{|s|}                
\end{eqnarray*} 

We now combine the two resulting in a recursive equation:

$$  \Longrightarrow\sum_{(0\le\bu\le n)}^{k}\!\!\!\!\! \bu_{|s|}
     ={k+n\choose k+1}+\sum_{(0\le\bu\le n)}^{k}\!\!\!\!\! \bu_{|s-1|}  $$

Finally we unfold and simplify:

$$  \Longrightarrow\sum_{(0\le\bu\le n)}^{k}\!\!\!\!\! \bu_{|s|}
     =s{k+n\choose k+1}\qed  $$

Next is $  (2)  $, an extension of the inversion property is needed:

$$ 0\le\bu\le n\quad\Longleftrightarrow\quad 0\le n-\hat{\bu}\le n $$

where the $ \hat{\bu} $ is the result from sequentially inverting $ \bu $.

\newpage

Upon verifying this sequential inversion as a permutation, as well as the logic of symmetry,
a generalization of Gauss' classic proof of the arithmetic series demonstrates as follows:

\begin{eqnarray*}
\sum_{(0\le\bu\le n)}^k\!\!\!\!\! 1\cdot\bu  
 & = & \sum_{(0\le n-\hat{\bu}\le n)}^k\!\!\!\!\! 1\cdot(n-\hat{\bu})      \\ 
 & = & \sum_{(0\le\bu\le n)}^k\!\!\!\! 1\cdot(n-\hat{\bu})  \\
 & = & \sum_{(0\le\bu\le n)}^k\!\!\!\!\! (1\cdot n-1\cdot\hat{\bu})  \\
 & = & \sum_{(0\le\bu\le n)}^k\!\!\!\!\! 1\cdot n
       \ -\sum_{(0\le\bu\le n)}^k\!\!\!\!\! 1\cdot\hat{\bu}            \\
 & = & kn{k+n\choose k}-\sum_{(0\le\bu\le n)}^k\!\!\!\!\! 1\cdot\bu        \\
\end{eqnarray*} 
$$  \Longrightarrow\sum_{(0\le\bu\le n)}^k\!\!\!\!\! 1\cdot\bu
    =\frac{kn}{2}{k+n\choose k}\qed  $$

And finally $  (5)  $, but I should explain an interpretation of some notation I use here.
If $ 1\cdot\bu $ is a dot product of two $ k $-tuples, one being $ 1 $ (with all it's entries as $ 1 $)
and the other being $ \bu $, then $ 1\cdot 1 $ would be $ 1+1+\ldots+1+1 $ where there are $ k $ terms
in this sum, hence $ 1\cdot 1=k $ in this instance.

We first partition the sum using the left side of the index property and simplify:

\begin{eqnarray*}
\sum_{(0\le\bu\le n+1)}^k\!\!\!\!\! t^{1\cdot\bu}           
 & = & \sum_{(0\le\bu\le n+1)}^{k-1}\!\!\!\!\! t^{1\cdot(0\uplus\bu)}
       +\sum_{(0\le\bu\le n)}^k\!\!\!\!\! t^{1\cdot(\bu+1)}                \\
 & = & \sum_{(0\le\bu\le n+1)}^{k-1}\!\!\!\!\! t^{0+1\cdot\bu}               
       +\sum_{(0\le\bu\le n)}^k\!\!\!\!\! t^{1\cdot\bu+1\cdot1}            \\
 & = & \sum_{(0\le\bu\le n+1)}^{k-1}\!\!\!\!\! t^{1\cdot\bu}              
       +t^k\sum_{(0\le\bu\le n)}^k\!\!\!\!\! t^{1\cdot\bu}                 \\
\end{eqnarray*} 

We secondly and similarily partition the sum using the right side of the index property and simplify:

\begin{eqnarray*}
\sum_{(0\le\bu\le n+1)}^k\!\!\!\!\! t^{1\cdot\bu}           
 & = & \sum_{(0\le\bu\le n+1)}^{k-1}\!\!\!\!\! t^{1\cdot[\bu\uplus (n+1)]}
       +\sum_{(0\le\bu\le n)}^k\!\!\!\!\! t^{1\cdot\bu}                    \\
 & = & \sum_{(0\le\bu\le n+1)}^{k-1}\!\!\!\!\! t^{1\cdot\bu+(n+1)}         
       +\sum_{(0\le\bu\le n)}^k\!\!\!\!\! t^{1\cdot\bu}                    \\
 & = & t^{n+1}\sum_{(0\le\bu\le n+1)}^{k-1}\!\!\!\!\! t^{1\cdot\bu}        
       +\sum_{(0\le\bu\le n)}^k\!\!\!\!\! t^{1\cdot\bu}                    \\
\end{eqnarray*} 

Now I combine the two:

\begin{eqnarray*}
\Longrightarrow \sum_{(0\le\bu\le n+1)}^{k-1}\!\!\!\!\! t^{1\cdot\bu}
+t^k\sum_{(0\le\bu\le n)}^k\!\!\!\!\! t^{1\cdot\bu}
 & = & t^{n+1}\sum_{(0\le\bu\le n+1)}^{k-1}\!\!\!\!\! t^{1\cdot\bu}       
       +\sum_{(0\le\bu\le n)}^k\!\!\!\!\! t^{1\cdot\bu}                    \\
\Longrightarrow \sum_{(0\le\bu\le n)}^k\!\!\!\!\! t^{1\cdot\bu}
 & = & \frac{1-t^{n+1}}{1-t^k}
       \sum_{(0\le\bu\le n+1)}^{k-1}\!\!\!\!\! t^{1\cdot\bu}
\end{eqnarray*} 

This is a generalization of the common proof to the geometric series:

$$  \Longrightarrow\sum_{(0\le\bu\le n)}^k\!\!\!\!\! t^{1\cdot\bu}           
    =\prod_{1\le j\le k}\frac{1-t^{n+j}}{1-t^{j\ \ \ }}\qed  $$

\section{Conclusion}

Regarding this basic introduction to the main results of this research, I hope you found it somewhat interesting.

A final note: I recognize I did not show a lot of the rigour required in regular math work---not to mention some
of the underlying assumptions and tricks. With respect, my guess is that those who would be interested in such
complicated index sums in the first place would generally be able to derive any of the missing pieces of this article
themselves.

In the off-case that someone new is interested but yet unable to follow the whole of the logic, feel
free to contact me and we can further discuss.

Pijariiqpunga.

\section{Problems}

1) Prove:

$$  \sum_{1\le j\le n}j^m{s+j\choose j}{n+k-s-j\choose k-s}
    =\sum_{1\le j\le m}j!
    \left\{\!\!\!\begin{array}{c} m \\ j \end{array}\!\!\!\right\}
    {s+j\choose j}{k+1+n\choose k+1+j}  $$

2) Prove:

\begin{eqnarray*}
\sum_{0\le j\le n}{s+j\choose j}{n+k-j\choose k}t^j
 & = & \sum_{0\le j\le n}{s+j\choose j}{k+s+1+n\choose k+s+1+j}(t-1)^j    \\
 & = & \sum_{0\le j\le n}\left(\sum_{j\le\ell\le n}{\ell\choose j}
       {s+\ell\choose\ell}{k+s+1+n\choose k+s+1+\ell}(-1)^{\ell-j}\right)t^j
\end{eqnarray*}       

\begin{thebibliography}{99}
\bibitem{gkp} R.L. Graham, D.E. Knuth, O. Patashnik.  Concrete
         Mathematics.  Addison-Wesley Publishing (1994).
\end{thebibliography}

\end{document} 
