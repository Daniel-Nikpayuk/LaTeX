% Copyright 2021 Daniel Nikpayuk
\documentclass[twoside]{article}
\usepackage[letterpaper,left=2cm,right=2cm,top=2.5cm,bottom=2.5cm]{geometry}
\usepackage{asymptote}
\usepackage{amsfonts}
\usepackage{amssymb}
\usepackage{amsmath}
\usepackage{verbatim}
\usepackage{graphicx}
\usepackage{hyperref}

%%%%%%%%%%%%%%%%%%%%%%%%%%%%%%%%%%%%%%%%%%%%%%%%%%%%%%%%%%%%%%%%

% latex symbols:

\newcommand{\vn}{\ensuremath{\varnothing}}

\newcommand{\RA}{\Rightarrow}
\newcommand{\lra}{\longrightarrow}
\newcommand{\then}{\ensuremath{\quad\Longrightarrow\quad}}
\newcommand{\qmapsto}{\ensuremath{\quad \mapsto \quad}}
\newcommand{\mapsfrom}{\mathrel{\reflectbox{\ensuremath{\mapsto}}}}

\newcommand{\defeq}{\ensuremath{\ :=\ }}
\newcommand{\qeq}{\ensuremath{\quad =\quad}}
\newcommand{\qqeq}{\ensuremath{\qquad =\qquad}}
\newcommand{\qdefeq}{\ensuremath{\quad :=\quad}}
\newcommand{\qqdefeq}{\ensuremath{\qquad :=\qquad}}

\newcommand{\quadbar}{\ensuremath{\quad |\quad}}
\newcommand{\quadcomma}{\ensuremath{\quad ,\quad}}
\newcommand{\equals}{\ensuremath{\quad =\quad}}
\newcommand{\defequals}{\ensuremath{\quad :=\quad}}

\newcommand{\qed}{\nobreak \ifvmode \relax \else
      \ifdim\lastskip<1.5em \hskip-\lastskip
      \hskip1.5em plus0em minus0.5em \fi \nobreak
      \vrule height0.75em width0.5em depth0.25em\fi}

% latex markup:

\newcommand{\strong}[1]{{\bfseries #1}}
\newcommand{\bfmbox}[1]{\mbox{\bfseries #1}}

\newtheorem{theorem}{Theorem}
\newtheorem{lemma}[theorem]{Lemma}
\newtheorem{proposition}[theorem]{Proposition}
\newtheorem{corollary}[theorem]{Corollary}

% latex spacing:

\newcommand{\twoquad}{\ensuremath{\quad\quad}}
\newcommand{\threequad}{\ensuremath{\quad\quad\quad}}
\newcommand{\fourquad}{\ensuremath{\quad\quad\quad\quad}}

\newcommand{\twoqquad}{\ensuremath{\qquad\qquad}}
\newcommand{\threeqquad}{\ensuremath{\qquad\qquad\qquad}}
\newcommand{\fourqquad}{\ensuremath{\qquad\qquad\qquad\qquad}}

% essay symbols:

\newcommand{\bv}[1][v]{\ensuremath{\mathbf #1}}
\newcommand{\powerset}[2][P]{\ensuremath{\mathbb{#1}(#2)}}
\newcommand{\nthps}[2][P]{\ensuremath{\mathbb{#1}^{#2}}}
\newcommand{\nthus}[2][U]{\ensuremath{\mathbb{#1}^{#2}}}
\newcommand{\psunion}[2][P]{\ensuremath{\bigcup\limits_{#2\ge 0}\mathbb{#1}^{#2}}}
\newcommand{\usunion}[2][U]{\ensuremath{\bigcup\limits_{#2\ge 0}\mathbb{#1}^{#2}}}
\newcommand{\stratified}{\ensuremath{\mathbb{U}^\infty}}
\newcommand{\of}[1]{\ensuremath{(\mathcal{#1})}}
\newcommand{\ofbb}[1]{\ensuremath{(\mathbb{#1})}}

\newcommand{\readleft}{\ensuremath{\,_{_\leftarrow}\,}}
\newcommand{\readright}{\ensuremath{\,_{_\rightarrow}\,}}

% essay formatting:

\newcommand{\mss}[1]{\ensuremath{\mbox{\scriptsize #1}}}
\newcommand{\bms}[1]{\ensuremath{_{\mbox{\bfseries\tiny #1}}}}
\newcommand{\bnms}[2]{\ensuremath{\bms{#1,}{_#2}}}

\newcommand{\tab}[1][1.125cm]{\hspace{#1}}

\newcommand{\col}[1][0ex]{& \hspace{#1}}
\newcommand{\scol}{\col[0.15cm]}
\newcommand{\lcol}{\col[0.45cm]}

\newcommand{\msbox}[1]{\ensuremath{_{\mbox{\scriptsize #1}}}}
\newcommand{\cbox}[1]{\mbox{// #1}}

\newenvironment{proof}[1][Proof]{\begin{trivlist}
\item[\hskip \labelsep {\bfseries #1}]}{\end{trivlist}}
\newenvironment{definition}[1][Definition]{\begin{trivlist}
\item[\hskip \labelsep {\bfseries #1:}]}{\end{trivlist}}
\newenvironment{example}[1][Example]{\begin{trivlist}
\item[\hskip \labelsep {\bfseries #1}]}{\end{trivlist}}
\newenvironment{remark}[1][Remark]{\begin{trivlist}
\item[\hskip \labelsep {\bfseries #1}]}{\end{trivlist}}

% essay functions:

\newcommand{\subcirc}[1][m]{\ensuremath{\underset{#1}{\circ}}}
\newcommand{\lcirc}{\subcirc[\ell]}
\newcommand{\rcirc}{\subcirc[r]}

\newcommand{\subfunc}[1]{\ensuremath{\langle #1\rangle}}

\newcommand{\id}{\mbox{id}}

\newcommand{\compose}{\mbox{compose}}
\newcommand{\precompose}[1]{\ensuremath{\mbox{precompose}_{#1}}}
\newcommand{\postcompose}[1]{\ensuremath{\mbox{postcompose}_{#1}}}

\newcommand{\ndopose}{\mbox{endopose}}
\newcommand{\prendopose}[1]{\ensuremath{\mbox{preendopose}_{#1}}}
\newcommand{\postndopose}[1]{\ensuremath{\mbox{postendopose}_{#1}}}

\newcommand{\lift}{\mbox{lift}}
\newcommand{\stem}{\mbox{stem}}
\newcommand{\costem}{\mbox{costem}}
\newcommand{\distem}{\mbox{distem}}

%%%%%%%%%%%%%%%%%%%%%%%%%%%%%%%%%%%%%%%%%%%%%%%%%%%%%%%%%%%%%%%%

\begin{asydef}
// this comment prevents a compilation bug.

real direction(bool value)
{
	return value ? 1 : -1;
}

pair offset(pair current, string align, real x = 0.03)
{
	pair shifter =	(align == "E") ? (x, 0) :
			(align == "N") ? (0, x) :
			(align == "W") ? (-x, 0) :
			(align == "S") ? (0, -x) : (0,0);

	return shift(shifter)*current;
}

pair segment(pair current = (0,0), string align, real projection, real angle = 0)
{
	align = (align == "E") ? "EU" :
		(align == "N") ? "NL" :
		(align == "W") ? "WD" :
		(align == "S") ? "SR" : align;

	real base          = direction(substr(align, 0, 1) == "E" || substr(align, 0, 1) == "N");
	real adjacent      = direction(substr(align, 1, 1) == "R" || substr(align, 1, 1) == "U") * Tan(angle);
	pair initialVertex = (substr(align, 0, 1) == "E" || substr(align, 0, 1) == "W") ? (base, adjacent) : (adjacent, base);
	pair scaledVertex  = scale(projection) * initialVertex;
	pair shiftedVertex = shift(current)    * scaledVertex;

	return shiftedVertex;
}

//

void drawpath(path p)
{
	draw(p);

	for (int k=0; k < size(p); ++k)
	{
		dot(point(p, k));
	}
}

void safeLabel(picture pic = currentpicture, Label L, pair position, real width, real height,
	align align = NoAlign, pen textpen = currentpen, pen borderpen = currentpen,
	pen fillpen = white, filltype filltype = NoFill, string bordertype = "round")
{
	if (bordertype == "round")
	{
		pair w = position + (-width, 0);
		pair e = position + ( width, 0);
		pair n = position + (0, height);
		pair s = position + (0,-height);

		filldraw(pic, w{up}::n{right}::e{down}::s{left}::cycle, fillpen, borderpen);
	}
	else if (bordertype == "box")
	{
		pair sw = position + (-width,-height);
		pair se = position + (-width, height);
		pair nw = position + ( width,-height);
		pair ne = position + ( width, height);

		filldraw(pic, sw--se--ne--nw--cycle, fillpen, borderpen);
	}

	label(pic, L, position, align, textpen, filltype);
}


\end{asydef}

%%%%%%%%%%%%%%%%%%%%%%%%%%%%%%%%%%%%%%%%%%%%%%%%%%%%%%%%%%%%%%%%

\title{Typed Assembly: From Function to Text}
\author{Daniel Nikpayuk}
\date{August 5, 2021}
\pagestyle{empty}
\begin{document}
\maketitle
\thispagestyle{empty}

\begin{figure}[h]
\centering
\includegraphics[width=1in]{../../../cc-by-nc.png}\\[0.1in]
\tiny This article is licensed under \\
\href{http://creativecommons.org/licenses/by-nc/4.0/}
{Creative Commons Attribution-NonCommercial 4.0 International.}\\[0.3in]
\end{figure}

What is typed assembly? The easiest way to explain is it's a deconstruction of the idea of function \strong{composition}.
With composition, you take functions $ f_0, f_1, \ldots, f_{n-1}, f_n $ and compose them together:
$$ f_n \circ f_{n-1} \circ \ldots \circ f_1 \circ f_0 $$

Simple enough, but this only works if the functions have composable domains/codomains:
$$ \begin{array}{rllll}
f_0	& :	& X_0		& \to		& X_1		\\
f_1	& :	& X_1		& \to		& X_2		\\
	&	&		& \tab[0.75ex]\vdots		\\
f_{n-1}	& :	& X_{n-1}	& \to		& X_n		\\
f_n	& :	& X_n    	& \to		& X_{n+1}	\\
\end{array} $$

What if your functions instead have arbitrary domains?
$$ \begin{array}{rllll}
f_0	& :	& X_0		& \to		& Y_0		\\
f_1	& :	& X_1		& \to		& Y_1		\\
	&	&		& \tab[0.75ex]\vdots		\\
f_{n-1}	& :	& X_{n-1}	& \to		& Y_{n-1}	\\
f_n	& :	& X_n		& \to		& Y_n		\\
\end{array} $$

\newpage

From a category theory perspective, you \emph{lift} these functions from their home category
into a \emph{typed assembly} category:
$$ \begin{array}{rllll}
\mbox{lift}(f_0)	& :	& Z	& \to		& Z	\\
\mbox{lift}(f_1)	& :	& Z	& \to		& Z	\\
			&	&  	& \tab[0.75ex]\vdots	\\
\mbox{lift}(f_{n-1})	& :	& Z	& \to		& Z	\\
\mbox{lift}(f_n)	& :	& Z	& \to		& Z	\\
\end{array} $$
where our typed assembly category object $ Z $ in effect can be defined as:
$$ Z \qdefeq X_0 \times \ldots \times X_n \times Y_0 \times \ldots \times Y_n $$

By \emph{refactoring} the domains and codomains, we can now compose our lifted functions:
$$ \mbox{lift}(f_n) \circ \mbox{lift}(f_{n-1}) \circ \ldots \circ \mbox{lift}(f_1) \circ \mbox{lift}(f_0) $$

This idea is borrowed from \emph{register machine theory}, and in fact is how register machines are created
if we were to explain how they work to an audience of mathematicians. The refactored common \emph{signature}
forms the registers, and \emph{lifted composition} in effect is the continuation passing monad. Each lifted
function represents an instruction. Finally, as we may desire \emph{recursion} we extend the register space
$ Z $ such that each \emph{typed} register is given its own \emph{stack} so you can save and restore values.

\newpage

\ \\
%\begin{comment}
%%%%%%%%%%%%%%%%%%%%%%%%%%%%%%%%%%%%%%%%%%%%%%%%%%%%%%%%%%%%%%%%%%%%%
                                  %
                                  %
\begin{center}
 \begin{asy}
 unitsize(1cm);
 
 //
 
 pair p00 = (0,0);
 
 pair p10 = segment(p00, "SL", 1, 20);
 pair p11 = segment(p00, "SR", 1, 40);
 pair p12 = segment(p00, "SR", 1, 60);
 pair p13 = segment(p00, "SR", 1, 70);
 
 pair p20 = segment(p12, "SL", 1, 20);
 pair p21 = segment(p12, "SR", 1, 50);
 
 //
 
 draw(p00--p10);
 draw(p00--p11);
 draw(p00--p12);
 draw(p00--p13);
 
 draw(p12--p20);
 draw(p12--p21);
 
 //
 
 label("$y_f=f(x_1,g(w),x_3)$:", (0,0.75), N);
 
 safeLabel("$f$", p00, 0.25, 0.25);
 
 safeLabel("$y_f$", p10, 0.25, 0.25);
 safeLabel("$x_1$", p11, 0.25, 0.25);
 safeLabel("$g$", p12, 0.25, 0.25);
 safeLabel("$x_3$", p13, 0.25, 0.25);
 
 safeLabel("$x_2$", p20, 0.25, 0.25);
 safeLabel("$w$", p21, 0.25, 0.25);
 
 \end{asy}
\end{center}
                                  %
                                  %
%%%%%%%%%%%%%%%%%%%%%%%%%%%%%%%%%%%%%%%%%%%%%%%%%%%%%%%%%%%%%%%%%%%%%
%\end{comment}

\end{document}
