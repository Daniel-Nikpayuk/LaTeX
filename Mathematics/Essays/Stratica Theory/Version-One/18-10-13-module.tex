% Copyright 2018 Daniel Nikpayuk
\documentclass[twoside]{article}
\usepackage[letterpaper,left=1cm,right=1cm,top=2cm,bottom=2cm]{geometry}
\usepackage{asymptote}
\usepackage{amsfonts}
\usepackage{graphicx}
\usepackage{hyperref}
\usepackage{amsmath}

\begin{asydef}
// this comment prevents a compilation bug.

pair p00 = (0,0);

pair p10 = (-4.5,-2);
pair p11 = (1.75,-2);
pair p12 = (10.50,-2);

pair p20 = (-6.5,-4);
pair p21 = (-3.5,-4);
pair p22 = (-0.5,-4);
pair p23 = (3.10,-4);
pair p24 = (7.25,-4);
pair p25 = (10.50,-4.30);

pair p30 = (-6.5,-6);
pair p31 = (-3.5,-6);
pair p32 = (-0.5,-6);
pair p33 = (3.10,-6);
pair p34 = (7.25,-6);

//

string s00 = "module";

string s10 = "interface";
string s11 = "perspective";
string s12 = "model";

string s20 = "structure";
string s21 = "navigator";
string s22 = "identity";
string s23 = "proximity";
string s24 = "functor";
string s25 = "\parbox{2.20cm}{submodules,\\branches}";

string s30 = "\parbox{2.10cm}{atoms,\\constructive\\grammar}";
string s31 = "\parbox{2.10cm}{arithmetics,\\monoids,\\algebras}";
string s32 = "\parbox{1.75cm}{equality,\\inequality}";
string s33 = "\parbox{4.05cm}{less than (or equal),\\greater than (or equal)}";
string s34 = "\parbox{2.65cm}{print, map,\\morph, repeat,\\assign}";

//

void safeLabel(picture pic = currentpicture, Label L, pair position, real width, real height,
		align align = NoAlign, pen p = currentpen, filltype filltype = NoFill)
{
	pair sw = position + (-width,-height);
	pair se = position + (-width, height);
	pair nw = position + ( width,-height);
	pair ne = position + ( width, height);

	filldraw(pic, sw--se--ne--nw--cycle, white, p);
	label(pic, L, position, align, p, filltype);
}

\end{asydef}

\title{Module Theory}
\author{Daniel Nikpayuk}
\date{October 13, 2018}
\pagestyle{empty}
\begin{document}
\maketitle
\thispagestyle{empty}

\begin{figure}[h]
\centering
\includegraphics[width=1in]{../../../cc-by-nc.png}\\[0.1in]
\tiny This article is licensed under \\
\href{http://creativecommons.org/licenses/by-nc/4.0/}
{Creative Commons Attribution-NonCommercial 4.0 International.}\\[0.3in]
\end{figure}

The following is my most practiced paradigm for designing a code module for a specified \emph{type}:\\[0.25cm]

\hspace{-0.75cm}\begin{asy}
unitsize(1cm);

pen darkgray = gray(0.35);

//

draw(p00--p10);
draw(p00--p11);
draw(p00--p12);

draw(p10--p20);
draw(p10--p21);

draw(p11--p22);
draw(p11--p23);
draw(p11--p24);

draw(p12--p25, darkgray);

draw(p20--p30, darkgray);
draw(p21--p31, darkgray);
draw(p22--p32, darkgray);
draw(p23--p33, darkgray);
draw(p24--p34, darkgray);

//

safeLabel(s00, p00, 0.75, 0.25);

safeLabel(s10, p10, 0.90, 0.25);
safeLabel(s11, p11, 1.15, 0.25);
safeLabel(s12, p12, 0.70, 0.25);

safeLabel(s20, p20, 0.95, 0.25);
safeLabel(s21, p21,    1, 0.25);
safeLabel(s22, p22, 0.85, 0.25);
safeLabel(s23, p23,    1, 0.25);
safeLabel(s24, p24, 0.80, 0.25);
safeLabel(s25, p25, 1.30, 0.55, darkgray);

safeLabel(s30, p30, 1.30, 0.85, darkgray);
safeLabel(s31, p31, 1.25, 0.85, darkgray);
safeLabel(s32, p32, 1.05, 0.55, darkgray);
safeLabel(s33, p33, 2.20, 0.55, darkgray);
safeLabel(s34, p34, 1.55, 0.85, darkgray);

\end{asy}

\vspace{0.5cm}

\subsection*{intuition}

The idea is that the best types to design are the ones that have multitudes of subtypes. To achieve this, our type
would need specifics for construction, and ways to navigate its instances. Once able to reach all such instances,
you then look to compare: You should be able to group and regroup instances to form subtypes of interest. Doing
it this way assures you can also navigate subtype instances, not to mention subtyping your subtypes---all
because you already have access to those operations within the root type.

This paradigm is inspired from the patterns I see within \emph{mathematical spaces}. I ask meta-mathematical questions
such as: Why is $ \mathbb{R}^3 $ so successful? Why can you model spheres and toruses and cubes and dodecahedrons and
so many other spaces with this one single space? Physicists and engineers and others use it and its subspaces for all
sorts of applications. These sorts of questions guide this design.

\newpage

\subsection*{details}

To get the ball rolling, I introduce the idea of a \emph{highly navigable} space. Keeping things at an intuitive
level, such a space would imply

\begin{enumerate}
\item {\bf navigability}: If you can't navigate the internal elements of a set-oriented structure, what's it's value?
			  Extending this, if you can't navigate it easily or freely, this also limits its value as
			  a toolset.
\item {\bf multiplicity}: Highly navigable doesn't just mean easily navigable, it means you can navigate it in many
			  different ways. It becomes highly interpretable, where it has many subspaces which themselves
			  are navigable. This allows for the space as a whole the ability to model many different
			  applications.
\end{enumerate}

With this said, it is not unnatural from this perspective that a {\bf module} here focuses on its navigability, as well
as its capacity to generate subtypes. Of course, as with anything else, we still need to start with its constructive
grammar, which is to say its {\bf interface}. Don't take this for granted, as it is often the case its navigational
features derive from its constructive grammars.

I'll explain this with my classic example of a $ 16\ \!\! $-bit \emph{word}. As a type, such a word has an interface,
which is to say it has a \emph{structure} (composed of bits) as well as a means to \emph{navigate} to those bits
and identify their value (or mutate them if such operations are permitted).

\begin{itemize}
\item {\bf structure}:	Identify the atoms, and the constructive grammar that make up the internals of this type.
\item {\bf navigator}:	Navigators often start with arithmetics, which build into monoids, and algebras. For example
			when navigating the natural numbers, you start with the successor function, from there
			you can build addition, then multiplication, and exponentiation. These later operators
			don't give you greater access to the natural numbers in their potential, but they certainly
			give you faster access in application. The key thing to note here is that your navigational
			system will effectively be a \emph{universal coordinate system} in the sense that you can use
			it to access all components that make up the interface structure.
\end{itemize}

Beyond the interface of a type is its {\bf perspective}. This leads up to generating our desired subtypes.
Just as a structure needs a universal coordinate system for navigation, substructures need a universal language
for separating and joining a type's internal instances. In a meta-mathematical sense, the common pattern in dividing
and uniting is the broad idea of a \emph{comparison}. If we can compare the internals of a type against a particular
value system, we can decide to split them, or concatenate them. How do we compare things in a general sense though?
The approach I offer here is an outward expanding collection of concentric circles of \emph{similarity}:

\begin{itemize}
\item {\bf identity}:	The first step in determining similarity is testing for equality, or inequality.
\item {\bf proximity}:	If two objects aren't equal, the next best measure of similarity is the idea of \emph{nearness},
			which generally translates to metrics, ordering relations, and otherwise general topological
			nearness. In practice though we do often have strong tests such as less than, less than or equal,
			greater than, greater than or equal.
\item {\bf functor}:	If we cannot identify two objects, and there is no desirable measure of nearness, we tend then
			to compare by indirectly through the use of another type. 

			At the root (module) level, the most powerful functors tend to be the well known \emph{print, map}
			operators. Printing is common for computational types. Mapping produces its own specializations:
			\emph{morph, repeat, assign}. Here I'm using ``morph'' in the special case that \emph{map} is an
			endofunctor (maps back to itself); ``assign'' being a special case of morph. My use of ``repeat''
			is the common one. In code these specializations are often worth distinguishing as they are
			used frequently enough to warrant cycle optimization.

			Finally, within submodules, functors tend to be more type specific, and so their patterns tend
			to be harder to predict in advance, but at least we can reserve space for them here.
\end{itemize}

Following the perspective of a type we can begin to look at its {\bf model}. With models, we are interpreting
and reinterpreting subtypes which offers a level of closure.~\footnote{Given that the non-recursive
aspects of this module are used to generate the self-similar nature, this concept in a sense is \emph{complete}.}

My advice to you: Don't be too strict when it comes to typology. Type safety is important but many of the best
innovations come from being able to reinterpret the purpose of anything and everything around us.~\footnote{This idea
I borrow both from mathematics as well as my own Inuit culture} In my culture, in the Arctic, we reinterpret water
(snow) and many other materials for greater reuse. We're not fixated on a type being limited to any one kind of thing.
With our current discussion, a $ 16\ \!\! $-bit word can be reinterpreted as a $ 16\ \!\! $-bit unsigned integer,
or a utf-16 unicode character, for example.  The idea of a model is narrow our type instance focus and to be able
to equip such instances with additional functions.~\footnote{If you're concerned about corrupting the underlying
type system, you can create a \emph{contextual scope} to mitigate any complications introduced.}

\subsection*{generality}

Modules not only help us design types, they help us design narratives:\\[0.25cm]

\hspace{-0.75cm}\begin{asy}
unitsize(1cm);

//

pair m00 = (0,0);
pair m01 = (1,0);
pair m02 = (2,0);

pair m10 = (  -1, 1);
pair m11 = ( 1.5, 1);
pair m12 = (2.75, 1);

pair m20 = (-1.5, 2);
pair m21 = (0.25, 2);
pair m22 = (   1, 2);
pair m23 = (   3, 2);

pair m30 = (  -2, 3);
pair m31 = (  -1, 3);
pair m32 = (   0, 3);
pair m33 = (   1, 3);
pair m34 = (   2, 3);
pair m35 = (   3, 3);

pair m40 = (-2.5, 4);
pair m41 = (  -1, 4);
pair m42 = ( 0.5, 4);
pair m43 = ( 1.5, 4);
pair m44 = ( 2.5, 4);
pair m45 = (   3, 4);
pair m46 = (   4, 4);

pair m50 = (  -3, 5);
pair m51 = (-1.5, 5);
pair m52 = (-0.2, 5);
pair m53 = (   1, 5);
pair m54 = (   2, 5);
pair m55 = (   4, 5);
pair m56 = (   5, 5);

pair m60 = (  -3, 6);
pair m61 = (-1.5, 6);
pair m62 = (-0.2, 6);
pair m63 = (   1, 6);
pair m64 = (   2, 6);
pair m65 = (   4, 6);
pair m66 = (   5, 6);

draw(m00--m10, gray, MidArrow);
draw(m01--m21, gray, Arrow(Relative(0.32)));
draw(m01--m11, gray, MidArrow);
draw(m02--m12, gray, MidArrow);

draw(m10--m20, gray, MidArrow);
draw(m10--m21, gray, MidArrow);
draw(m11--m22, gray, MidArrow);
draw(m12--m23, gray, MidArrow);

draw(m20--m30, gray, MidArrow);
draw(m20--m31, gray, MidArrow);
draw(m21--m32, gray, MidArrow);
draw(m22--m33, gray, MidArrow);
draw(m22--m34, gray, MidArrow);
draw(m23--m34, gray, MidArrow);
draw(m23--m35, gray, MidArrow);

draw(m30--m40, lightblue, MidArrow);
draw(m30--m41, lightblue, MidArrow);
draw(m31--m42, lightblue, MidArrow);
draw(m32--m43, gray, MidArrow);
draw(m33--m43, gray, MidArrow);
draw(m34--m43, gray, MidArrow);
draw(m34--m44, gray, MidArrow);
draw(m35--m45, lightgreen, MidArrow);
draw(m35--m46, lightgreen, MidArrow);

draw(m40--m50, lightblue, MidArrow);
draw(m41--m51, lightblue, MidArrow);
draw(m41--m52, lightblue, MidArrow);
draw(m42--m52, lightblue, MidArrow);
draw(m43--m53, gray, MidArrow);
draw(m44--m54, gray, MidArrow);
draw(m45--m55, lightgreen, MidArrow);
draw(m46--m56, lightgreen, MidArrow);

draw(m50--m60, paleblue, MidArrow);
draw(m51--m61, paleblue, MidArrow);
draw(m52--m62, paleblue, MidArrow);
draw(m53--m63, lightgray, MidArrow);
draw(m54--m64, lightgray, MidArrow);
draw(m55--m65, palegreen, MidArrow);
draw(m56--m66, palegreen, MidArrow);

dot(m00);
dot(m01);
dot(m02);

dot(m10);
dot(m11);
dot(m12);

dot(m20);
dot(m21);
dot(m22);
dot(m23);

dot(m30, blue);
dot(m31, blue);
dot(m32);
dot(m33);
dot(m34);
dot(m35, green);

dot(m40, blue);
dot(m41, blue);
dot(m42, blue);
dot(m43);
dot(m44);
dot(m45, green);
dot(m46, green);

dot(m50, blue);
dot(m51, blue);
dot(m52, blue);
dot(m53);
dot(m54);
dot(m55, green);
dot(m56, green);

\end{asy}

\vspace{0.5cm}

This graphic can be interpreted as a \emph{narrative} module which can help you design libraries among other things.
Each dot is a local module, and the arrows connecting them are dependencies. The dots (local modules) are instances
of this narrative type, and they are navigated through their storied relationships. You can then compare the dots
respectively, and use these comparisons to create subnarratives. In the above the subnarratives are in blue and green.
This might be a bit abstract, so I'll give two examples.

The first example is none other than \emph{axiomatic set theory} itself! The dots at the bottom are axioms,
where definitions and theorems are derived from them as higher level dots. This derivation process continues
fanning upward and outward as axiomatic systems tend to do. You create the subnarratives to be other branches
of math derived from set theory such as group theory, graph theory, etc.

The second example is for programming in particular, in the case you might want to create narratives of
\emph{hardware} types. The major narrative then has local hardware type modules as dots, and source dependencies
as the arrows. You create subnarratives when you concentrate on specific hardware abstractions such as screen,
memory, touchpad, etc.

\subsection*{conclusion}

I will finish by saying the value in this paradigm is that when the relationships are identified, any module that
adheres to this design is well suited for composition. Structure composes with structure, navigators compose with
navigators, identity, proximity, functor perspectives compose respectively with their corresponding perspectives.

If building a library, you can then focus on building your narrative design. Furthermore, if you have a narrative
that's stabilized, a common optimization translates quite readily: For any given stratification of composed
layers (structure, navigator, function), you unroll, reduce overhead, compress.~\footnote{I would warn, this
optimization is mostly relevant when you're frequently passing back and forth between layers of abstraction.}

\end{document}
