% Copyright 2014 Daniel Nikpayuk
\documentclass[twoside]{article}
\usepackage[letterpaper,left=1cm,right=1cm,top=6.5cm,bottom=2cm]{geometry}
\usepackage{asymptote}
\usepackage{graphicx}
\usepackage{hyperref}

\title{Technology Space}
\author{Daniel Nikpayuk}
\date{December 9, 2014}
\pagestyle{empty}
\begin{document}
\maketitle
\thispagestyle{empty}

\begin{figure}[h]
\centering
\includegraphics[width=1in]{cc-by-nc.png}\\[0.1in]
\tiny This article is licensed under \\
\href{http://creativecommons.org/licenses/by-nc/4.0/}
{Creative Commons Attribution-NonCommercial 4.0 International.}\\[0.3in]
\end{figure}

\newpage

\begin{asydef}
import graph;
unitsize(1cm);

pair [] z=new pair[15];
z[0]=(-7.5,0);
z[1]=(-5.5,0.175);
z[2]=(-4,-0.4);
z[3]=(-3.5,0.35);
z[4]=(-2.5,0.05);
z[5]=(-1.3,0.7);
z[6]=(-0.9,-0.8);
z[7]=(0,-0.9);
z[8]=(0,0);
z[9]=(0.3,0.75);
z[10]=(2,1);
z[11]=(3,-0.9);
z[12]=(3,0);
z[13]=(4,-0.1);
z[14]=(6.25,0.27);

real [] o=new real[4];
o[0]=0;
o[1]=0;
o[2]=9;
o[3]=1.5;

real [] lEllipse(real [] e)
{
	real [] le;
	le[0]=e[0]-(5*e[2])/10;
	le[1]=e[1]+(4*e[2])/10;
	le[2]=e[2]/4;
	le[3]=e[3]/4;

	return le;
}

real [] mEllipse(real [] e)
{
	real [] me;
	me[0]=e[0];
	me[1]=e[1]+(5*e[2])/10;
	me[2]=e[2]/5;
	me[3]=e[3]/5;

	return me;
}

real [] rEllipse(real [] e)
{
	real [] re;
	re[0]=e[0]+(4*e[2])/10;
	re[1]=e[1]+(4*e[2])/10;
	re[2]=e[2]/3;
	re[3]=e[3]/3;

	return re;
}

real [] fEllipse(real [] e)
{
	real [] fe;
	fe[0]=e[0];
	fe[1]=e[1]+(2*e[2])/5;
	fe[2]=(7*e[2])/10;
	fe[3]=(7*e[3])/10;

	return fe;
}

void drawEllipse(real [] e)
{
	draw(ellipse((e[0], e[1]), e[2], e[3]), black);
}

void drawDottedEllipse(real [] e)
{
	draw(ellipse((e[0], e[1]), e[2], e[3]), dashed+lightblue);
}

\end{asydef}

\begin{center}
\noindent\hspace*{-0.8cm}\begin{asy}
//Context Space Visual:

drawEllipse(o);
	real [] l=lEllipse(o);
	drawEllipse(l);
		real [] ll=lEllipse(l);
		drawEllipse(ll);

		real [] lr=rEllipse(l);
		drawEllipse(lr);
			real [] lrl=lEllipse(lr);
			drawEllipse(lrl);

			real [] lrm=mEllipse(lr);
			drawEllipse(lrm);

	real [] r=rEllipse(o);
	drawEllipse(r);
		real [] rl=lEllipse(r);
		drawEllipse(rl);
			real [] rlr=rEllipse(rl);
			drawEllipse(rlr);

		real [] rm=mEllipse(r);
		drawEllipse(rm);

		real [] rr=rEllipse(r);
		drawEllipse(rr);
			real [] rrr=rEllipse(rr);
			drawEllipse(rrr);
				real [] rrrr=rEllipse(rrr);
				drawEllipse(rrrr);

//Ambiguity:

draw(z[0]..z[1]---z[2], red);
draw(z[0]{NE}..{E}z[3], red);
draw(z[0]{E}..{N}z[3], red);
draw(z[1]{NE}..z[3], red);
draw(z[8]{SW}..z[4], red);
draw(z[8]{W}..{N}((z[5]+z[8])/2){N}..{W}z[5], red);
draw(z[8]..z[6], red);
draw(z[8]{NE}..z[9]{NW}..z[5], red);
draw(z[8]..z[12]..z[13], red);
draw(z[7]{E}..{N}((z[7]+z[11])/2){N}..{E}z[11], red);
draw(z[10]{E}..{S}((z[10]+z[14])/2){S}..{E}z[14], red);
draw(z[10]{E}..{S}z[12], red);
draw(z[12]{NW}..z[11], red);
draw(z[9]{S}..{E}z[11], red);
draw(z[2]{S}..{N}z[5], red);
draw(z[11]{E}..{W}z[14], red);
draw(z[14]{W}..{S}((z[14]+z[2])/2){S}..{W}z[2], red);
draw(z[9]{W}..{S}((z[9]+z[1])/2){S}..{N}z[1], red);

dot(z);

//Subtext:

label("Context Space", (0, -3));

real xLabel=9.25;
label("- 0 (ambiguity)", (xLabel, 0), E);
label("- 1 (context)", (xLabel, 3.4), E);
label("- 2 (subtext)", (xLabel, 4.5), E);
label("- 3 (\ldots)", (xLabel, 5), E);

\end{asy}
\end{center}

One starts with the Context Space:\vspace*{0.65cm}

First there is \emph{ambiguity}! It's not that one starts with no ``meaning'' in this universe and creates it from nothing,
rather there is too much meaning and we carve it from everything.  There's equal possibility for all meanings, and thus no
way to know if one meaning is privileged over another.

From ambiguity comes order, but for each pattern of order introduced, much is still left unordered and ambiguous,
allowing for many many possible orders still.  The first and lowest patterning of order is called an \emph{infrastructure};
or framework; or discourse; or semantics; or just plain context. As humans we come together by means of common experience.
There are many experiences in our lives which bring us so much joy, or so much pain; many of these experiences are common
and cannot be avoided---we must bring meaning to them. The Context Space is the first level above complete ambiguity
in establishing common and individual meanings for our lives.

Once an infrastructure of communication is established, one can further restrict the possibility for ambiguity,
resulting in sub-contexts, or \emph{subtexts}. And from subtexts themselves one can further restrict out subtexts
(subsubtexts).  This may continue many times.

It is possible to reach a point of restriction when only one unique possibility remains; an unique meaning.

There is also \emph{intertextuality} to consider. Comparing contexts with their related subtexts.

Keep note though, a common human error is to assume two contexts are comparable when in fact they do not even
belong to the same infrastructure. Maybe a metric of comparison can be designed to allow a comparision, but one
should not assume from the start that all things are naturally comparable. At times, justification is needed.

\newpage

\begin{center}
\noindent\begin{asy}
//Semiotic Space Visual:

drawDottedEllipse(o);

label("``The rabbit heard a noise and ran.''", (0, 3));
label("Semiotic Space", (0, -3));

label("The", (0,0), blue);
label("rabbit", (-6,0), blue);
label("heard", (3,1), blue);
label("a", (-2.7,-0.4), blue);
label("noise", (5,-0.2), blue);
label("and", (3.2,0.1), blue);
label("ran", (-4,0.23), blue);
label("\ldots", (-2.9,0.41), blue);
label("\ldots", (-1,1.1), blue);
label("\ldots", (2.4,-0.72), blue);

\end{asy}
\end{center}

Next comes the Semiotic Space:\vspace*{0.65cm}

Signs, signifiers, signifieds. Fancy words, which just mean \emph{representation}. With so much diversity and depth and complexity
in infinitely many ways in infinitely many contexts it helps to build tools---or technology---to navigate, to simplify.

The most common tools to move around and navigate are \emph{words} themselves. Any individual word captures a portion of context:
On the one hand, that portion is still highly ambiguous, but on the other it is also embedded with much meaning---so much meaning
in a single word.

One can save a lot of work in interacting with a context by using word tools to make the process easier. But be warned, how much is missed?
how much is ``lost in translation'' when one becomes too comfortable with their tools? their technologies?
When one lets meaning become routine\ldots

\newpage

\begin{center}
\noindent\hspace*{-0.75cm}\begin{asy}
//Media Space Visual:

real [] f=fEllipse(fEllipse(o));
f[1]=0;
drawEllipse(f);
	real [] ff=fEllipse(f);
	drawEllipse(ff);
		real [] fff=fEllipse(ff);
		drawEllipse(fff);
			real [] ffff=fEllipse(fff);
			drawEllipse(ffff);
				real [] fffff=fEllipse(ffff);
				drawEllipse(fffff);

real h=5;

label("``The rabbit heard a noise and ran.''", (-12, h+1));
label("Media Space", (-7, -2));

draw((-12,h)--(-13,h-1), lightred);
	draw((-13,h-1)--(-14,h-2), lightred);
	draw((-13,h-1)--(-12,h-2), lightred);
		draw((-14,h-2)--(-15,h-3), lightred);
		draw((-12,h-2)--(-13,h-3), lightred);
draw((-12,h)--(-11,h-1), lightred);

real ew=6.3, eh=0;
draw((-14,h-2){S}..{E}(-ew,eh), dashed+lightblue, Arrow);

ew=(7*ew)/10; eh+=(2*ew)/5;
draw((-12,h-2){S}..{E}(-ew,eh), dashed+lightblue, Arrow);

ew=(7*ew)/10; eh+=(2*ew)/5;
draw((-13,h-1){ESE}..{E}(-ew,eh), dashed+lightblue, Arrow);

ew=(7*ew)/10; eh+=(2*ew)/5;
draw((-11,h-1){E}..{E}(-ew,eh), dashed+lightblue, Arrow);

ew=(7*ew)/10; eh+=(2*ew)/5;
draw((-12,h){E}..{E}(-ew,eh), dashed+lightblue, Arrow);

real few=1, feh=0.25;
fill(ellipse((-12,h), few, feh), white);
	fill(ellipse((-13,h-1), few, feh), white);
		fill(ellipse((-14,h-2), few, feh), white);
			fill(ellipse((-15,h-3), few, feh), white);
		fill(ellipse((-12,h-2), few, feh), white);
			fill(ellipse((-13,h-3), few, feh), white);
	fill(ellipse((-11,h-1), few, feh), white);

label("and", (-12,h), orange);
	label("heard", (-13,h-1), orange);
		label("rabbit", (-14,h-2), orange);
			label("The", (-15,h-3), orange);
		label("noise", (-12,h-2), orange);
			label("a", (-13,h-3), orange);
	label("ran", (-11,h-1), orange);

\end{asy}
\end{center}

Finally the dynamic; the Media Space:\vspace*{0.65cm}

It is the place where context and semiotics interact, where we produce and imagine and create unique meaning!

One \emph{parses} a sequence of words, a \emph{sentence}. One parses these words with a syntax, with a grammar,
which in simpler terms is the way in which one gives order to the process of sifting out meaning---it is the
way in which one applies each word successively to a context space to form an understanding.

In this example, one starts with ``rabbit'' and from the infrastructure we raise, we lift, we sift;
we use the word as a \emph{context filter}. The word itself narrows the infrastructure,
resulting in a narrower but still vague context.

On top of that filter, one adds another filter forming ``The rabbit'' and then ``The rabbit heard''
etc.. We continue this process until we've applied all possible filters, narrowing the meaning each time until
the ideal result: Once the final filter has been applied, a unique meaning, or a unique \emph{interpretation} is established.

What an amazing technology! so much can be said by reusing and recycling a far fewer number words than contexts.
How efficient but also flexible, adaptable, modular, extensible!

But do not take this for granted! So much can go wrong. What if you're inexperienced and parse the same sentence
but in a different order? What if you apply the same context filters but in a different order? Sometimes a subtly
different unique meaning is derived; sometimes a very different unique meaning is obtained; sometimes one arrives
at a context in which there is still much ambiguity and yet no unique meaning.

We use this Technology Space called \emph{language} all the time and yet take
for granted what it \emph{is}; what its potential is; and what could go wrong in its use.

How can we really get to know someone else, or even ourselves if we take for granted one of our main interfaces
to understanding that person? How can we really contribute back to our families and our communities if we lack
a mature understanding of the tools we use to read, write, navigate? to decode, encode, translate? to maintain our relations
in an ever changing world?

P.S. If you're only interested in linear parsing, all you need are context filters; otherwise for non-linear parsing you need
both filters and projectors (functional inverses of each other).

\end{document}
//label("("+xCoor+", "+yCoor+")", (0,0));
//label("", (, ));


