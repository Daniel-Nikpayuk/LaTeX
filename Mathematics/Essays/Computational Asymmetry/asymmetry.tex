% Copyright 2018 Daniel Nikpayuk
\documentclass[twoside]{article}
\usepackage[letterpaper,left=1cm,right=1cm,top=1.5cm,bottom=1.5cm]{geometry}
\usepackage{amsmath}
\usepackage{graphicx}
\usepackage{hyperref}

\newcommand{\then}{\ensuremath{\quad\Longrightarrow\quad}}
\newcommand{\equals}{\ensuremath{\quad =\quad}}

\newcommand{\fourqquad}{\ensuremath{\qquad\qquad\qquad\qquad}}
\newcommand{\cat}{\ensuremath{c}}
\newcommand{\chat}{\ensuremath{\hat{c}}}

\pagestyle{empty}
\begin{document}

\begin{figure}[h]
\centering
\includegraphics[width=1in]{../../../cc-by-nc.png}\\[0.1in]
\tiny This article is licensed under \\
\href{http://creativecommons.org/licenses/by-nc/4.0/}
{Creative Commons Attribution-NonCommercial 4.0 International.}\\[0.3in]
\end{figure}

Recently, I was looking to solve the combinatorial sum:
$$ S_n \equals \sum_{0\le k\le n}{2k \choose k}{2n-2k \choose n-k} $$
by means of generating functions. This style of approach ends up being a lot of algebraic work by hand and is best solved by
the symbolic packages of a computer, but otherwise what lessons can be learned if we had to do it ourselves?

The object of this article is to apply the idea of \emph{interface theory}, where we start with a space, take an inventory
of its subspaces, and then for any given subspace create a compressed representation which maintains its own navigational algebra.
This is an \emph{interface layer}. Solving the moments of the above generating function offers a quality example of this process,
showing of the merits of interface design toward general problem solving.

To begin, when we're looking to solve for the individual terms of a rational generating function such as the above, it helps to decompose
it into its roots:

Converting to its generating function starts us off with:

\begin{align*}
S(z)		& \equals \sum_{n\ge 0}\left(\sum_{0\le k\le n}{2k \choose k}{2n-2k \choose n-k}\right)z^n		\\
															\\
		& \equals \left(\sum_{n\ge 0}{2n \choose n}z^n\right)\left(\sum_{n\ge 0}{2n \choose n}z^n\right)	\\
															\\
		& \equals \left(\sum_{n\ge 0}{2n \choose n}z^n\right)^2							\\
\end{align*}

It follows that the next step in our strategy is to find the generating function for the derived series:
$$ \sum_{n\ge 0}{2n \choose n}z^n $$
and although we don't immediately know its generating function, we do know the generating function for a similar series:
$$ \cat(z) \equals \sum_{n\ge 0}\frac{1}{n+1}{2n \choose n}z^n $$
which are the \emph{catalan numbers}, and compress to
$$ \cat(z) \equals \frac{1-\sqrt{1-4z}}{2z} $$

This is useful to us as:
\begin{align*}
\cat(z)			& \equals \sum_{n\ge 0}\frac{1}{n+1}{2n \choose n}z^n				\\
													\\
z\cat(z)		& \equals \sum_{n\ge 0}\frac{1}{n+1}{2n \choose n}z^{n+1}			\\
													\\
\frac{d}{dz}(z\cat(z))	& \equals \frac{d}{dz}\sum_{n\ge 0}\frac{1}{n+1}{2n \choose n}z^{n+1}		\\
													\\
z\cat'(z)+\cat(z)	& \equals \sum_{n\ge 0}{2n \choose n}z^n					\\
\end{align*}
so then it's a matter of finding $ \cat'(z) $. Although we can derive this directly from $ c(z) $, with a bit of
foresight---not to mention past experience manipulating generating functions---it's worth using an interface
which will allow us to avoid direct use of the square root function. Mostly, it offers a case study in how to
use interfaces.

Our interface starts with the \emph{conjugate} of our catalan function:
$$ \chat(z) \equals \frac{1+\sqrt{1-4z}}{2z} $$

The value in this conjugate is it offers an algebra where we can hide the square root:
$$ \cat+\chat \equals \frac{1}{z} \fourqquad \cat\chat \equals \frac{1}{z} $$

We already have our compressions: $ \cat, \chat $, so we only need derive algebraic rules. The main thing with square
roots is we would prefer to always have them in the numerators of our rational functions, and this will inform our algebra
as well. Regardless, it is natural to run through the arithmetic operators of \emph{addition} and \emph{multiplication} first:
these derive quite readily from the definitions.

Next, it we derive the \emph{additive} and \emph{multiplicative} inverses:
\begin{align*}
-\cat		& \equals -\frac{1}{z}+\chat		& \frac{1}{\cat}	& \equals z\chat	\\
													\\
-\chat		& \equals -\frac{1}{z}+\cat		& \frac{1}{\chat}	& \equals z\cat		\\
\end{align*}

With these, we have the basis needed to derive pretty much any other form we'd be interested in, for example:
\begin{align*}
\cat-\chat		& \equals \cat+\left(-\frac{1}{z}+\cat\right)			\\
											\\
			& \equals -\frac{1}{z}+2\cat					\\
											\\
			& \quad\mbox{or}						\\
											\\
\cat^2			& \equals (-\cat)(-\cat)					\\
											\\
			& \equals -\cat\left(-\frac{1}{z}+\chat\right)			\\
											\\
			& \equals \cat\frac{1}{z}-\cat\chat				\\
											\\
			& \equals -\frac{1}{z}+\frac{1}{z}\cat				\\
											\\
			& \quad\mbox{or}						\\
											\\
\frac{\cat}{\chat}	& \equals z\cat^2						\\
											\\
			& \equals z\left(-\frac{1}{z}+\frac{1}{z}\cat\right)		\\
											\\
			& \equals -1+\cat						\\
\end{align*}

We still need to derive $ \cat'(z) $ of course, and this can be done as follows:
\begin{align*}
\cat^2			& \equals -\frac{1}{z}+\frac{1}{z}\cat				\\
											\\
z\cat^2			& \equals -1+\cat						\\
											\\
\frac{d}{dz}(z\cat^2)	& \equals \frac{d}{dz}(-1+\cat)					\\
											\\
\cat^2+2z\cat\cat'	& \equals \cat'							\\
											\\
\cat^2			& \equals (1-2z\cat)\cat'					\\
											\\
\cat'			& \equals \frac{\cat^2}{1-2z\cat}				\\
\end{align*}

This particular form isn't as simple as could be, and so we should simplify further:
\begin{align*}
\cat'			& \equals \frac{\cat^2}{1-2z\cat}						\\
													\\
			& \equals \frac{\cat^2}{1-2z\cat}\cdot\frac{1-2z\chat}{1-2z\chat}		\\
													\\
			& \equals \frac{\cat^2(1-2z\chat)}{(1-2z\cat)(1-2z\chat)}			\\
													\\
			& \equals \frac{\cat(\cat-2z\cat\chat)}{1-2z(\cat+\chat)+4z^2\cat\chat}		\\
													\\
			& \equals \frac{\cat^2-2\cat}{-1+4z}						\\
													\\
			& \equals \frac{-\frac{1}{z}+\frac{1}{z}\cat-2\cat}{-1+4z}			\\
													\\
			& \equals \frac{1}{z} \cdot \frac{-1+(1-2z)\cat}{-1+4z}				\\
													\\
			& \equals \frac{1+(2z-1)\cat}{z(1-4z)}						\\
\end{align*}

It's useful to put this into a \emph{normal form} which in this case means separating the partial fractions now:
$$ \frac{1}{z(1-4z)} \equals \frac{1}{z}+\frac{4}{1-4z} $$
leaving us with:
$$ \cat'(z) \equals \frac{1}{z}+\frac{4}{1-4z}+\left(\frac{2z-1}{z}+\frac{8z-4}{1-4z}\right)\cat $$

So now:
\begin{align*}
\sum_{n\ge 0}{2n \choose n}z^n	& \equals z\cat'(z)+\cat(z) 								\\
															\\
				& \equals 1+\frac{4z}{1-4z}+\left(2z-1+\frac{8z^2-4z}{1-4z}\right)\cat+\cat		\\
															\\
				& \equals \frac{1-4z}{1-4z}+\frac{4z}{1-4z}+\left(\frac{(2z-1)(1-4z)}{1-4z}
				  +\frac{8z^2-4z}{1-4z}+\frac{1-4z}{1-4z}\right)\cat					\\
															\\
				& \equals \frac{1}{1-4z}+\frac{(2z-8z^2-1+4z)+8z^2-4z+1-4z}{1-4z} \cdot \cat		\\
															\\
				& \equals \frac{1}{1-4z}-\frac{2z}{1-4z}\cat						\\
\end{align*}

\begin{align*}
S(z)			& \equals \left(\sum_{n\ge 0}{2n \choose n}z^n\right)^2							\\
																\\
			& \equals \left(\frac{1}{1-4z}-\frac{2z}{1-4z}\cat\right)^2						\\
																\\
			& \equals \frac{1}{(1-4z)^2}-2\frac{2z}{(1-4z)^2} \cdot \cat+\frac{(2z)^2}{(1-4z)^2} \cdot \cat^2	\\
																\\
			& \equals \frac{1}{(1-4z)^2}-\frac{4z}{(1-4z)^2} \cdot \cat
				  +\frac{4z^2}{(1-4z)^2}\left(-\frac{1}{z}+\frac{1}{z}\cat\right)				\\
																\\
			& \equals \frac{1}{(1-4z)^2}-\frac{4z}{(1-4z)^2} \cdot \cat+\frac{4z}{(1-4z)^2}(-1+\cat)		\\
																\\
			& \equals \frac{1}{(1-4z)^2}-\frac{4z}{(1-4z)^2} \cdot \cat
				  -\frac{4z}{(1-4z)^2}+\frac{4z}{(1-4z)^2} \cdot \cat						\\
																\\
			& \equals \frac{1}{1-4z}
\end{align*}

And so
$$ \sum_{0\le k\le n}{2k \choose k}{2n-2k \choose n-k} \equals 4^n $$

Computational asymmetry in problem solving!

\end{document}

