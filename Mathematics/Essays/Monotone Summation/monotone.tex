% Copyright 2019 Daniel Nikpayuk
\documentclass[twoside]{article}
\usepackage[letterpaper,left=2.5cm,right=2.5cm,top=2.5cm,bottom=2.5cm]{geometry}
\usepackage{amsfonts}
\usepackage{graphicx}
\usepackage{hyperref}
\usepackage{amsmath}

\newcommand{\bu}[1][u]{\ensuremath{\mathbf #1}}
\newcommand{\eq}{\ensuremath{\quad=\quad}}

\title{Monotone Summation Identities}
\author{Daniel Nikpayuk}
\date{December 8, 2019}
\begin{document}
\maketitle

\begin{figure}[h]
\centering
\includegraphics[width=1in]{../../../cc-by-nc.png}\\[0.1in]
\tiny This article is licensed under \\
\href{http://creativecommons.org/licenses/by-nc/4.0/}
{Creative Commons Attribution-NonCommercial 4.0 International.}\\[0.3in]
\end{figure}

\section{Terminology}

Monotone sums are of the following form:
$$ \sum_{0\le u_1\le u_2\le u_3\le\ldots\le u_{k-1}\le u_k\le n}f(u_1, u_2, u_3,\ldots, u_{k-1}, u_k) $$
As is, this is quite tedious to parse so we shorten our notation as follows:
$$ \sum_{(0\le\bu\le n)}^k\!\!\!\!\! f(\bu)  $$
where
$$ \bu\ :=\ (u_1, u_2,\ldots, u_{k-1}, u_k)  $$
belongs to the montone $ k $-space: The set of monotonically increasing sequences of integers of length $ k $.

This inspires further notation: In particular, if you wish to refer the $ j^{\mbox{\scriptsize th}} $ coordinate
of $ \bu $, you simply write:
$$  \bu_{/j}  $$
Such notation is analogous to being able to access the \emph{real} and \emph{imaginary} components of a complex number:
$$ \mathcal{R}e(z),\ \mathcal{I}m(z) $$

We will also want a \emph{concatenation} operator:
$$  \bu \mid \bu[v]\ :=\ (\bu_{/1},\ldots ,\bu_{/k},\bu[v]_{\!/1}, \ldots ,\bu[v]_{\!/\ell})  $$

We can now express our summand switch laws:
\begin{align*}
\sum_{(0\le\bu\mid\bu[v]\le n)}^{j,k}                                       
       \!\!\!\!\!\!\! f(\bu\mid\bu[v])                                 
 &\eq \sum_{(0\le\bu\le n)}^j                                           
       \sum_{(\bu\le\bu[v]\le n)}^k\!\!\!\!\! f(\bu\mid\bu[v])     \\  
 &\eq \sum_{(0\le\bu[v]\le n)}^k                                          
       \sum_{(0\le\bu\le\bu[v])}^j\!\!\!\!\! f(\bu\mid\bu[v])      \\
\end{align*}

\section{Identities}

These sums are able to provide discrete closed forms for well known combinatorial numbers:
\begin{align*}
{k+n\choose n}
 &\eq \sum_{(0\le\bu\le n)}^k \!\!\! 1						\\
\left\{\!\!\!\begin{array}{c} k+n+1 \\ n+1 \end{array}\!\!\!\right\}
 &\eq \sum_{(0\le\bu\le n)}^k \!\!\!\!
	(\bu_{/1}+1)(\bu_{/2}+1)\ldots(\bu_{/k}+1)				\\
\left[\!\!\!\begin{array}{c} k+n+1 \\ n+1 \end{array}\!\!\!\right]   
 &\eq \sum_{(0\le\bu\le n)}^k \!\!\!\!
	(\bu_{/1}+1)(\bu_{/2}+2)\ldots(\bu_{/k}+k)				\\
\end{align*}

As for general identities, if we denote $ 1\cdot\bu := \bu_{/1} + \ldots + \bu_{/k} $, then:
\begin{align*}
\sum_{(0\le\bu\le n)}^k\!\!\!\!\! \bu_{\!/s}
 &\eq s{k+n\choose k+1}                                                    \\
\sum_{(0\le\bu\le n)}^k\!\!\!\!\! 1\cdot\bu  
 &\eq \frac{kn}{2} {k+n\choose k}                                          \\ 
\sum_{(0\le\bu\le n)}^k\!\!\!\!\! t^{\bu_{\!/1}} 
 &\eq \sum_{0\le j\le n}{k+n\choose k+j}(t-1)^j                            \\
\sum_{(0\le\bu\le n)}^k\!\!\!\!\! t^{\bu_{\!/k}} 
 &\eq \sum_{0\le j\le n}{k+n\choose k+j}t^{n-j}(t-1)^j                     \\
\sum_{(0\le\bu\le n)}^k\!\!\!\!\! t^{1\cdot\bu}           
 &\eq \prod_{1\le j\le k}\frac{1-t^{n+j}}{1-t^{j\ \ \ }}                   
 &									   \\
\sum_{(0\le\bu\le n)}^k\!\!\!\!\! \bu_{\!/s+1}^m 
 &\eq \sum_{1\le j\le m}j!\left\{\!\!\!\begin{array}{c} m \\ j 
       \end{array}\!\!\!\right\}{s+j\choose j}{k+n\choose k+j}              \\
\sum_{(0\le\bu\le n)}^k\!\!\!\!\! (\bu_{\!/1}^m+\ldots +\bu_{\!/k}^m)
 &\eq \sum_{1\le j\le m}j!\left\{\!\!\!\begin{array}{c} m \\ j 
       \end{array}\!\!\!\right\}{k+j\choose 1+j}{k+n\choose k+j}            \\
\sum_{(0\le\bu\le n)}^k\!\!\!\!\! t^{\bu_{\!/s+1}}
 &\eq \sum_{0\le j\le n}{s+j\choose j}{k+n\choose k+j}(t-1)^j                
\end{align*} 

\begin{thebibliography}{99}
\bibitem{gkp} R.L. Graham, D.E. Knuth, O. Patashnik.  Concrete
         Mathematics.  Addison-Wesley Publishing (1994).
\end{thebibliography}

\end{document} 
