% Copyright 2014 Daniel Nikpayuk
\documentclass[twoside]{article}
\usepackage[letterpaper,left=1cm,right=1cm,top=2cm,bottom=2cm]{geometry}
\usepackage{asymptote}
\usepackage{graphicx}
\usepackage{hyperref}

\pagestyle{empty}
\begin{document}

\begin{asydef}
import graph;
unitsize(1cm);

//My choice of "semi" and "auto " do not imply "semi-auto".

//base components:

	path semi0=(1,4){N}..{S}(-1,4);
	path semi45=rotate(45)*semi0;
	path semi120=rotate(120)*semi0;
	path semi135=rotate(135)*semi0;
	path semi180=rotate(180)*semi0;
	path semi240=rotate(240)*semi0;

	path auto3=semi120{NE}..{E}(0,-1){E}..{SE}semi240;
	path auto4=semi45{SE}..{S}(-1,0){S}..{SW}semi135;

//bi-cognite:
	picture bicog;
	draw(bicog, semi0--semi180--cycle);

//curved bi-cognite:
	picture curbicog;
	draw(curbicog, auto3{NW}..{SW}cycle);

//tri-cognite:
	path auto120=rotate(120)*auto3;
	path auto240=rotate(240)*auto3;

	picture tricog;
	draw(tricog, auto3);
	draw(tricog, auto120);
	draw(tricog, auto240);

//quad-cognite:
	path auto90=rotate(90)*auto4;
	path auto180=rotate(180)*auto4;
	path auto270=rotate(270)*auto4;

	picture quadcog;
	draw(quadcog, auto4);
	draw(quadcog, auto90);
	draw(quadcog, auto180);
	draw(quadcog, auto270);

\end{asydef}

\begin{center}
\bfseries\Large Cognate Theory \\ \normalsize Daniel Nikpayuk \\ December 8, 2014
\end{center}

\noindent\hspace*{0cm}\begin{asy}
//draw:

picture cogpic;
add(cogpic, tricog);
add(cogpic, shift(-6.6,-5.6)*rotate(145)*scale(3/4)*bicog);
add(cogpic, shift(-5.4,1.4)*rotate(3)*scale(3/4)*bicog);
add(cogpic, shift(5.35,2.85)*rotate(-60)*curbicog);
add(cogpic, shift(-2.525, 8.95)*rotate(-1.5)*quadcog);

picture pic;
add(pic, rotate(90)*scale(1/2)*cogpic);

real sc=1/2;
real fs=sc*18pt;
label(pic, "0", (0,0), fontsize(fs));
label(pic, "1", (-1.4,2.5), fontsize(fs));
label(pic, "2", (-0.7,-2.7), fontsize(fs));
label(pic, "3", (-4.455,-1.23), fontsize(fs));
label(pic, "4", (2.845,-3.35), fontsize(fs));

add(scale(sc)*pic);

\end{asy}
\noindent\hspace*{2cm}\begin{tabular}{r|rrrrrrr}
0 & \{1\} & , & \{1, 3\} & , & \{2, 4\} & & \\
1 & \{0\} & , & \{0, 3\} & & & & \\
2 & \{0, 4\} & , & \{3\} & & & & \\
3 & \{0, 1\}  & , & $ \emptyset\,\ $ & , & $ \emptyset\,\ $ & , & \{2\} \\
4 & \{0, 2\} & , & $ \emptyset\,\ $ & & & & \\
\end{tabular}\\[0.5cm]

The general idea here is to replace the existing \emph{graph theory} with improved definitions:\\[0.1cm]

It can be argued the choice of representation towards the semantics of graph theory grew out various Western
cultures that privilege/d an \emph{object over relationship} model. I'm not arguing the merits of ``individualism''
as many great contributions have been made essentializing its value as an ideology---I do argue it is
\emph{one way among many} and it is thus worth considering the contributions of other worldviews. As culture shapes
definitions even within something as ``pure'' as mathematics one may notice within graph theory a privileging of objects
(vertices) and only by assuming their existence first do we then construct relationships (edges). An example of the
problematic nature of this approach is in considering a ``3-way relationship'':

Is it fair to use graph theory to say a 3-way relationship is a graph between 3 objects?  (i.e.~it consists of a collection
of 2-way relationships between 3 objects) What of \emph{emergence}? Is a relationship not more than the sum of its parts?
Given the existing axioms and definitions of graph theory these sorts of considerations are an after-thought. I should
again emphasize that I do not present this theory as a matter of ``fairness'' rather as a need to point out potentially
poor design of definition; especially given the need to better understand relational structures in this ever complex world.
To be {\bfseries bold}, I don't believe the language of graph theory is up to the challenge.

A \emph{cognate} is a relationship that is priviliged first. This is to say it is assumed to exist first and only
then can objects be represented as the \emph{gaps} between these cognates. Part of choice of terminology---aside from
a cognate being a type of relationship---is the word `cognate' is similar to `cognite' (cognition) and when looking
at the brain one has things like \emph{synaptic clefts} which aren't objects in-and-of-themselves yet graph theory
would represent them as such. And this idea of a `gap' isn't new to math either as the more elegant of the two constructions
(the other being Cauchy sequences) of the real line uses \emph{dedekind cuts}: The unique absence of a number is its very
definition---such an absence observed as a gap in the rational line.

Now if one were to represent a \emph{cognation} as a structuring of cognates using set theory it would be as a mapping from
the cognate set to various sequences of subsets of those cognates (with a few additional restrictions). For example $ c_0 $
(cognate $ 0 $ in the above figure) has 3 \emph{orientations} (relational orientations), the first of which is with $ c_1 $;
the second with $ c_1 $ and $ c_3 $, etc.

That's the intuitive understanding anyway. As such it would be easy enough to migrate the existing definitions and theorems of graph
theory over to this cognate theory. What's more: one can actually properly represent things like ``multi-graphs'' and ``directed-graphs''
all within a single definition. This elegance is in opposition to current graph theory where one must use similar but different
definitions to represent such similar but different structures. Finally, this broader definition allows for simpler representations
of more interesting directional relationships like if at a given \emph{junction} between cognates there was a one-directional
flow between some cognates but not others. Such a thing isn't even easily perceived of within the frames of graph theory.

\begin{figure}[h]
\centering
\includegraphics[width=1in]{../../../cc-by-nc.png}\\[0.1in]
\tiny This article is licensed under \\
\href{http://creativecommons.org/licenses/by-nc/4.0/}
{Creative Commons Attribution-NonCommercial 4.0 International.}\\[0.3in]
\end{figure}

\end{document}

