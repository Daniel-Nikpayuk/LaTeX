% Copyright 2015 Daniel Nikpayuk
\documentclass[twoside]{article}
\usepackage[letterpaper,left=1cm,right=1cm,top=2cm,bottom=2cm]{geometry}
\usepackage{amsmath}
\usepackage{amsfonts}
\usepackage{graphicx}
\usepackage{hyperref}
\usepackage{xcolor}
\usepackage{bm}

\newcommand{\st}{$ ^{\mbox{\scriptsize st}} $ }
\newcommand{\nd}{$ ^{\mbox{\scriptsize nd}} $ }
\newcommand{\rd}{$ ^{\mbox{\scriptsize rd}} $ }

\renewcommand{\th}{$ ^{\mbox{\scriptsize th}} $ }
%\th command exists, it gives the old english symbol used for `theta' IPA (voiceless)

\newcommand{\seq}[1][u]{\ensuremath{<\!#1\!>}}
\newcommand{\bseq}[1][u]{\ensuremath{<\!\!\bm{#1}\!\!>}}
\newcommand{\underseq}[2][u]{\ensuremath{\underset{#2}{<\!#1\!>}}}
\newcommand{\bunderseq}[2][u]{\ensuremath{\underset{#2}{<\!\!\bm{#1}\!\!>}}}

\newcommand{\radix}[2][u]{\ensuremath{\underset{#2}{(#1)}}}
\newcommand{\radixp}[3][u]{\ensuremath{\underset{#2}{(#1)^{#3}}}}
\newcommand{\bradix}[2][u]{\ensuremath{\underset{#2}{(\bm{#1})}}}
\newcommand{\bradixp}[3][u]{\ensuremath{\underset{#2}{(\bm{#1})^{#3}}}}

\newtheorem{theorem}{Theorem}[section]
\newtheorem{lemma}{Lemma}[section]

\newenvironment{proof}[1][Proof]{\begin{trivlist}
\item[\hskip \labelsep {\bfseries #1}]}{\end{trivlist}}
\newenvironment{definition}[1][Definition]{\begin{trivlist}
\item[\hskip \labelsep {\bfseries Definition (#1):}]}{\end{trivlist}}

\newcommand{\qed}{\nobreak \ifvmode \relax \else
      \ifdim\lastskip<1.5em \hskip-\lastskip
      \hskip1.5em plus0em minus0.5em \fi \nobreak
      \vrule height0.5em width0.5em depth0.25em\fi}

\pagestyle{empty}
\begin{document}

\begin{figure}[h]
\centering
\includegraphics[width=1in]{../../../cc-by-nc.png}\\[0.1in]
\tiny This article is licensed under \\
\href{http://creativecommons.org/licenses/by-nc/4.0/}
{Creative Commons Attribution-NonCommercial 4.0 International.}\\[0.3in]
\end{figure}

This short article provides a notation for the positional (radix) representation of natural numbers with the intention
of formula manipulation regarding numeric analytic as well as number theoretic proofs.

I'll get straight to it. Let $ \mathbb{N}^\mathbb{Z} $ be the bidirectional sequences of natural numbers:
$$ \bm{u}:=<\ldots, u_{-2}, u_{-1}, u_0, u_{1}, u_{2},\ldots> $$
Given we're working with sequences as our data structures, we will want to be able to \emph{filter} as means
of navigating and accessing the data. As such, we begin with subsequences:
$$ \underset{\mathbb{P}_2(k)}{\seq[\mathbb{P}_1(\bm{u})]},\quad k\in\mathbb{Z} $$
Currently this notation is too vague and might not make much sense. As an example to help clarify, take the property:
$$ \seq[a\le\bm{u}\le b]\qquad:\Longleftrightarrow\qquad <u_k\ |\ a\le u_k\le b,\ k\in\mathbb{Z}> $$
to mean the subsequence of $ \bm{u} $ satisfying the given bounds.  With such a property $ \mathbb{P}_1(\bm{u})=a\le\bm{u}\le b $,
we've constrained and thus \emph{filtered} the possible sequences in a \emph{weak} sense, meaning we haven't isolated a
unique case but still have many sequences satisfying the filtering condition.\footnote{As a way of thinking, weak specifications
are in fact just fine---and used all the time. This relates to a way of thinking called concurrent logic.}
In addition to this type of filtering, we can further constrain possible sequences by filtering on the index set instead:
$$ \underset{c\le k\le d}{\bseq},\quad k\in\mathbb{Z} $$
which means the subsequence of $ \bm{u} $ with indices $ [c\ldots d] $.

It is also possible to filter sequences using both filtering styles:
$$ \underset{c\le k\le d}{\seq[a\le\bm{u}\le b]},\quad k\in\mathbb{Z} $$
though in practice this exact notation will seldom be used---it is powerful, but a bit ugly.
In the very special but important case where $ c\le k\le c $ (or $ k=c $), we simplify our notation:
$$ \bseq_c\quad:=\quad u_c,\quad c\in\mathbb{Z} $$
to mean the $ c $\th ordered/indexed element of the sequence $ \bm{u} $.

As the filtering properties can be anything, they can also be constructive in nature:
$$ \underset{c\le k\le d}{\seq[b^k]} $$
which would be interpreted as:
$$ <b^c,b^{c+1},\ldots,b^d> $$
the finite subsequence of powers of $ b $ between $ c,d $.

You can also have constants:
$$ \bseq[1]:=<\ldots,1,1,1,\ldots> $$

Given our sequences can be interpreted as \emph{vectors}, we also have \emph{componentwise operators} and the \emph{dot-product}.

Finally, as our \emph{stream} sequence is infinite in both directions, we are also able to \emph{shift}:

$$ \begin{array}{rcl}
\seq[(\bm{u}\ll s)] & := & <u_{k-s}\ |\ k\in\mathbb{Z}> \\
\seq[(\bm{u}\gg s)] & := & <u_{k+s}\ |\ k\in\mathbb{Z}> \\
\end{array} $$

\newpage

This brings us to our main notation:

$$ \bradixp{0\le k\le n}{b}\quad:=\quad\underseq[0\le\bm{u} < b]{0\le k\le n}\cdot\underseq[b^k]{0\le k\le n} $$
where $ (\cdot) $ is the dot-product. Note, when expanded out this is our positional notation:
$$ \bradixp{0\le k\le n}{b}\quad =\quad u_0+u_1b+u_2b^2+\ldots+u_nb^n $$
when the context is clear (and frankly it often is), we will omit the base $ b $:
$$ \bradix{0\le k\le n} $$
Finally, keep in mind it is implicit in the above definition that $ b > 0 $.

We will prove a few basic manipulation properties of this notation.
The first thing to realize is our notation holds true under weaker constraints:
$$ \bradixp{m\le k\le n}{b}\quad:=\quad\underseq[0\le\bm{u} < b]{m\le k\le n}\cdot\underseq[b^k]{m\le k\le n} $$
which follows simply from the notational expansion above.

\begin{lemma}
\color{blue}
$$ \bradixp{j\le k\le\ell}{b}\quad =\quad \radixp[\bm{u}\ll j]{\!\!0\le k\le\ell-j}{b}\, b^j $$
\end{lemma}

\begin{proof}
$$ \begin{array}{rrrcl}
\bradixp{j\le k\le\ell}{b}
	& = &	\bunderseq{j\le k\le\ell}		& \cdot & \underseq[b^k]{j\le k\le\ell}			\\
														\\
	& = &	\underseq[\bm{u}\ll j]{0\le k\le\ell-j} & \cdot & \underseq[b^{k+j}]{\!\!0\le k\le\ell-j}	\\
														\\
	& = &	\underseq[\bm{u}\ll j]{0\le k\le\ell-j} & \cdot & \underseq[b^k]{\!\!0\le k\le\ell-j}\, b^j	\\
														\\
	& = &	\radixp[\bm{u}\ll j]{0\le k\le\ell-j}{b}\, b^j	& \qed	\\
\end{array} $$
\end{proof}

\begin{lemma}
\color{blue}
$$ \bradixp{0\le k\le\ell}{b}\quad <\quad b^{\ell+1} $$
\end{lemma}

\begin{proof}
$$ \begin{array}{rcl}
\bradixp{0\le k\le\ell}{b}
	& = &	\underseq[0\le\bm{u}\le b-1]{0\le k\le\ell}\cdot\underseq[b^k]{0\le k\le\ell}	\\
												\\
	& \le &	(b-1)\underseq[b^k]{0\le k\le\ell}						\\
												\\
	& = &	(b-1)\frac{b^{\ell+1}-1}{b-1}							\\
												\\
	& = &	b^{\ell+1}-1									\\
	& < &	b^{\ell+1}\quad\qed								\\
\end{array} $$
\end{proof}

\newpage

\begin{lemma}
\color{blue}
$$ \bradixp{j\le k\le\ell}{b}\quad\le\quad\bradixp{0\le k\le\ell}{b}\quad <\quad \bradixp{j\le k\le\ell}{b}+b^j $$
\end{lemma}

\begin{proof}
From the lemma above, the fact that $ b > 0 $ as well as the constraint that $ 0\le\bm{u} < b $, we have:
$$ 0\quad\le\quad\bradixp{0\le k < j}{b}\quad <\quad b^j $$
Adding $ \bradixp{j\le k\le\ell}{b} $ componentwise to all three sides leaves us with:
$$ \begin{array}{rrcrl}
\bradixp{j\le k\le\ell}{b} & \le & \bradixp{0\le k < j}{b}+\bradixp{j\le k\le\ell}{b} & < & \bradixp{j\le k\le\ell}{b}+b^j	\\
																\\
\bradixp{j\le k\le\ell}{b} & \le & \bradixp{0\le k\le\ell}{b} & < & \bradixp{j\le k\le\ell}{b}+b^j\quad\qed			\\
\end{array} $$
\end{proof}

\end{document}

